\begin{table}[!htbp]
\centering
\caption{Table reporting an example of original data and the respective emotion carriers and valences. On the first column the fields and on the second the respective values. Emotion carriers, reported in the third column, are particular tokens chunks that convey narrators’ emotions in the text. Valence is a number ranging from -2 to +2 that represents if the highlighted section of the text induced sadness or happiness in a different listener, who is not the narrator.}
\label{tab:dataset-coadapt-example-ec-valence}
    \centering
    \begin{tabularx}{\linewidth}{ l | X | p{2cm} | c}
        \toprule
        \multicolumn{4}{c}{ \thead{Coadapt Original Data}}\\
        \midrule
        \thead{Question} & \thead{Data} & \thead{Emotion \\ Carrier} & \thead{Valence}\\
        \midrule
        % \arrayrulecolor{lightgray}
        initial note &  \highLight[highlightgreen]{Serenità coi fiori}  &  \multicolumn{1}{c|}{$\sim$} & +1\\[1em]
        % \midrule
        cbt response a & Domenica nel mio giardino & \multicolumn{1}{c|}{$\sim$}& $\sim$ \\[1em]
        % \midrule
        cbt response b &  \highLight[highlightgreen]{Sarebbe bello poter avere un profumo simile a quello delle viole o dell' iris} &  poter avere, un profumo & +1\\[1em]
        % \midrule
        cbt response c & Occupandomi dei miei fiori \highLight[highlightgreen]{ho sentito una sensazione piacevole data dal profumo delle viole e dal sole che leggero accarezzava la pelle. Ho provato Felicità} nelle parti del corpo: Testa sensazione di inebriamento. & ho sentito, profumo, vuole, sole, accarezzava & +1 \\[1em]
        % \arrayrulecolor{black}
        \bottomrule
    \end{tabularx}
\end{table}
