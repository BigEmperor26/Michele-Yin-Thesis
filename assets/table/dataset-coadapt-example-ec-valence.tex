\begin{table}[!htbp]
\centering
\caption{An example of original data and the respective emotion carriers and valences. On the first column, the division in questions, and on the second, the respective textual answers. Emotion carriers, reported in the third column convey narrators’ emotions in the text. The last column reports the values of valence for the respective highlighted sections.}
\label{tab:dataset-coadapt-example-ec-valence}
    \centering
    \begin{tabularx}{\linewidth}{ l | X | p{2cm} | c}
        \toprule
        \multicolumn{4}{c}{ \thead{Coadapt Original Data}}\\
        \midrule
        \thead{Question} & \thead{Answer} & \thead{Emotion \\ Carrier} & \thead{Valence}\\
        \midrule
        % \arrayrulecolor{lightgray}
        initial note &  \highLight[highlightgreen]{Serenità coi fiori}  &  \multicolumn{1}{c|}{$\sim$} & +1\\[1em]
        % \midrule
        cbt response a & Domenica nel mio giardino & \multicolumn{1}{c|}{$\sim$}& 0 \\[1em]
        % \midrule
        cbt response b &  \highLight[highlightgreen]{Sarebbe bello poter avere un profumo simile a quello delle viole o dell' iris} &  poter avere, un profumo & +1\\[1em]
        % \midrule
        cbt response c & Occupandomi dei miei fiori \highLight[highlightgreen]{ho sentito una sensazione piacevole data dal profumo delle viole e dal sole che leggero accarezzava la pelle. Ho provato Felicità} nelle parti del corpo: Testa sensazione di inebriamento. & ho sentito, profumo, vuole, sole, accarezzava & +1 \\[1em]
        % \arrayrulecolor{black}
        \bottomrule
    \end{tabularx}
\end{table}
