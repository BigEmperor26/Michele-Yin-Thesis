\begin{table}[!htbp]
\centering
\caption{Table reporting the desired and undesired properties of the eliciting questions }
\label{tab:dataset-crowdsourcing-guidelines-properties}
    \centering
    % \begin{tabularx}{p{2cm}|p{3cm}|p{5cm}|p{5cm}}
    \begin{tabularx}{\linewidth}{ X | X }
        \toprule
        \thead{Desired properties} & \thead{Undesired properties} \\
        \midrule
         Suggested, but not enforced, focused on the highlighted portions of the narratives, i.e. parts of the narratives with valence. &   Questions on personal opinions. \\[2em]
         Containing feedback signals, such as \emph{``Capisco"} or \emph{``Oh, che bello"} and other signs of active listening. &  Suggestions \\[2em]
         Centered around the narrator, i.e. should not move away from the narrator the focus of the story. &   Hypothetical questions or scenarios. \\[2em]
         Containing explicit references to events that happened in the narrative. &  Questions that move the focus of the conversation away from the narrator.\\[2em]
         Showing empathy to the narrator, for instance with words as \emph{``Mi dispiace"} when sad events are mentioned. \\[2em]
         Short and on point. \\[1em]
         Correct, both grammatically and syntactically. \\[1em]
         Focused on events of the narrative. \\[1em]
        \bottomrule
    \end{tabularx}
\end{table}
% The main purpose of these guidelines was to help workers in producing a good elicitation. Our goals were for the elicitation to be questions with the following properties:
% \begin{itemize}
    
%  Focused on events of the narrative.
%  Suggested, but not enforced, focused on the highlighted portions of the narratives, i.e. parts of the narratives with valence.
%  Containing feedback signals, such as \emph{``Capisco"} or \emph{``Oh, che bello"} and other signs of active listening.
%  Centered around the narrator, i.e. should not move away from the narrator the focus of the story.
%  Containing explicit references to events that happened in the narrative.
%  Showing empathy to the narrator, for instance with words as \emph{``Mi dispiace"} when sad events are mentioned.
%  Short and on point.
%  Correct, both grammatically and syntactically.
% % \end{itemize}
% % On the other hand, we explicitly required our elicitation not to be any of the following:
% % \begin{itemize}
%  Questions on personal opinions.
%  Suggestions.
%  Hypothetical questions or scenarios.
%  Questions that move the focus of the conversation away from the narrator.
% % \end{itemize}