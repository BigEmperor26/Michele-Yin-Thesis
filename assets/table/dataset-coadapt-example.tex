\begin{table}[!htbp]
\centering
\caption{Table reporting an example of original data and revision that was applied to it. The original data is composed of answers to a set of 4 questions, which is not structured in a narrative format. A simple concatenation of all text present in the data column would not suffice to obtain a narrative. Therefore manual review was applied in order to get a coherent and fluid narrative, of which an example is reported in the rightmost column.}
\label{tab:dataset-coadapt-example}
    \centering
    % \begin{tabularx}{p{2cm}|p{3cm}|p{5cm}|p{5cm}}
    \begin{tabularx}{\linewidth}{ l | p{10cm} | X }
        \toprule
        \multicolumn{2}{c|}{ \thead{Coadapt Original Data}} & \thead{Revised Narrative}\\
        \midrule
        \thead{Question} & \thead{Data} & \multirow{5}{3.5cm}{Domenica nel mio giardino, occupandomi dei miei fiori ho sentito una sensazione piacevole data dal profumo delle viole e dal sole che leggero accarezzava la pelle. Sarebbe bello poter avere un profumo simile a quello delle viole o dell' iris. }\\
        \cmidrule{1-2}
        % \arrayrulecolor{lightgray}
        initial note & Serenità coi fiori &  \\ [1em]
        % \cmidrule{1-2}
        cbt response a & Domenica nel mio giardino \\ [1em]
        % \cmidrule{1-2}
        cbt response b & Sarebbe bello poter avere un profumo simile a quello delle viole o dell' iris \\ [2em]
        % \cmidrule{1-2}
        cbt response c & Occupandomi dei miei fiori ho sentito una sensazione piacevole data dal profumo delle viole e dal sole che leggero accarezzava la pelle. Ho provato Felicità nelle parti del corpo: Testa sensazione di inebriamento.\\[4em]
        % \arrayrulecolor{black}
        \bottomrule

    \end{tabularx}
\end{table}
