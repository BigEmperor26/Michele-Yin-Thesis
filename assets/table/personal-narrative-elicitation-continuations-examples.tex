\begin{table}[!htbp]
\centering
\caption{Examples of eliciting questions for two narratives are reported on each row. Narratives are reported in the first column and corresponding eliciting questions are reported in the second column. On the top a longer narrative with two elicitation. Notice that eliciting questions for the longer narrative pursue different topics. On the bottom is a shorter narrative, with more eliciting questions but on the same few set of topics.}
\label{tab:personal-narrative-elicitation-continuations-example}
    \centering
    \begin{tabularx}{\linewidth}{ X | X  }
    % \begin{tabular}{p{1.5cm}|p{3cm}|p{5cm}|p{2.5cm}|p{2cm}}
        \toprule
        % \thead{Mode}
        % \midrule
       \thead{Narrative} & \thead{Eliciting Questions} \\
        \midrule
        \multirow{2}{8cm}{Giornata piacevole ma stancante. Comunione di una nipotina. Oggi non è stata una giornata abbastanza calda ... mangiare al freddo non è il massimo. Non ero emozionata sapendo dove era il posto ... mi sono coperta per quanto possibile visto il periodo.} &  Congratulazioni per il bellissimo evento. La tua nipotina è stata felice? \\
 [2em]
       %        \cmidrule{2-2}
        & Se non altro hai allenato il tuo spirito di adattamento. Spero che la giornata sia andata bene, eri contenta alla fine della giornata? \\
        \arrayrulecolor{black}
        \midrule
        \multirow[t]{10}{*}{Che noia finiranno le feste?} & Mi spiace tu ti annoi, come mai?\\
 [1em]
       %        \cmidrule{2-2}
        & Ti capisco. Come mai non ti piacciono?\\
 [1em]
       %        \cmidrule{2-2}
        & Capisco, immagino che hai trascorso delle belle giornate!\\
 [1em]
       %        \cmidrule{2-2}
        & Mi dispiace non ti piacciano, perché vuoi che finiscano?\\
 [1em]
       %        \cmidrule{2-2}
        & Mi dispiace, che cosa hai fatto durante le feste che ti ha annoiato?\\
 [2em]
       %        \cmidrule{2-2}
        & Mi spiace tu stia così, che cosa ti causa noia in queste feste?\\
 [1em]
       %        \cmidrule{2-2}
        & Mi dispiace che ti annoi, perché vuoi che finiscano?\\     
 [1em]
       %        \cmidrule{2-2}
        & Concordo, quando ricominci a lavorare tu?\\      
 [1em]
       %        \cmidrule{2-2}
        & Ti capisco, per quale motivo ti senti annoiato?\\      
 [1em]
       %        \cmidrule{2-2}
        & Mi dispiace che le feste ti annoino, come mai non ti piacciono?\\
       \arrayrulecolor{black}
        \bottomrule

    \end{tabularx}
\end{table}
              % "input": "Giornata piacevole ma stancante. Comunione di una nipotina. Oggi non \u00e8 stata una giornata abbastanza calda ... mangiare al freddo non \u00e8 il massimo. Non ero emozionata sapendo dove era il posto ... mi sono coperta per quanto possibile visto il periodo. ",
              %   "output": [
              %       {
              %           "generated_text": "Congratulazioni per il bellissimo evento. La tua nipotina \u00e8 stata felice? "
              %       },
              %       {
              %           "generated_text": "Se non altro hai allenato il tuo spirito di adattamento. Spero che la giornata sia andata bene, eri contenta alla fine della giornata?"
              %       }
              %   ],
              %   "id": "264"
              %               {
              %   "input": "Che noia finiranno le feste? ",
              %   "output": [
              %       {
              %           "generated_text": "Mi spiace tu ti annoi, come mai?"
              %       },
              %       {
              %           "generated_text": "Ti capisco. Come mai non ti piacciono?"
              %       },
              %       {
              %           "generated_text": "Capisco, immagino che hai trascorso delle belle giornate!"
              %       },
              %       {
              %           "generated_text": "Mi dispiace non ti piacciano, perch\u00e9 vuoi che finiscano?"
              %       },
              %       {
              %           "generated_text": "Mi dispiace, che cosa hai fatto durante le feste che ti ha annoiato?"
              %       },
              %       {
              %           "generated_text": "Mi spiace tu stia cos\u00ec, che cosa ti causa noia in queste feste?"
              %       },
              %       {
              %           "generated_text": "Mi dispiace che ti annoi, perch\u00e9 vuoi che finiscano?"
              %       },
              %       {
              %           "generated_text": "Concordo, quando ricominci a lavorare tu?"
              %       },
              %       {
              %           "generated_text": "Ti capisco, per quale motivo ti senti annoiato?"
              %       },
              %       {
              %           "generated_text": "Mi dispiace che le feste ti annoino, come mai non ti piacciono?"
              %       }
              %   ],