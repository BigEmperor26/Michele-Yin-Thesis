\begin{table}[!htbp]
\centering
\caption{Table reporting an example that was provided to the users of the crowdsourcing data collection with one example of narrative with one example respectively of correct and incorrect elicitation with their respective motivation. The users were shown the same example, but with more examples of incorrect elicitations with their respective motivation in order to train them against easy pitfalls in the elicitation.}
\label{tab:dataset-crowdsourcing-guidelines}
    \centering
    % \begin{tabularx}{p{2cm}|p{3cm}|p{5cm}|p{5cm}}
    \begin{tabularx}{\linewidth}{ p{1.5cm} | X | p{2.5cm} | p{3cm} }
        \toprule
        \multicolumn{4}{c} { \thead{Examples of elitictations}}\\
        \midrule
        \thead{Narrative} & \multicolumn{3}{p{14.5cm}}{ \highLight[highlightgreen]{Oggi è stata una bella giornata. Mia moglie mi ha detto che sta aspettando un bambino!} Sono super felice! Mi chiedo se sarò un bravo padre. \highLight[highlightred]{Mio padre non è stato molto presente quando ero un bambino.}}\\
        \midrule
        \thead{Example} & \thead{Text} & \thead{Evaluation} & \thead{Explaination} \\
        % \arrayrulecolor{lightgray}
        \midrule
        \thead{1} & Sono felice di sentirlo. Sapete già se si tratta di un maschio o di una femmina ? &	\textbf{CORRETTO} &	Segue tutte le linee guida \\[2em]
        % \midrule
        \thead{2} &	Oh capisco. Cosa mi racconti? &	\textbf{ERRATO} &	Non esplora la narrativa, troppo generica \\
        % \arrayrulecolor{black}
        \bottomrule
    \end{tabularx}
\end{table}
