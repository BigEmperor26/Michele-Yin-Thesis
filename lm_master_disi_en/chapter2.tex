\chapter{Literature review}
\label{cha:literature_review}
This section consists of a literature review, with a focus on three main topics Narrative Elicitation, Crowdsourcing and Large Language Models.
\section{Personal Narrative}
A personal narrative is \cite{}
\section{Narrative elicitation}
Narrative elicitation is a research or interview technique used to collect qualitative data in the form of narratives or stories from individuals. It involves prompting participants to share their personal experiences, perspectives, and insights by recounting a specific event or describing a particular aspect of their lives. Researchers use this method to gain a deeper understanding of people's experiences, beliefs, values, and emotions.

% Key aspects of narrative elicitation include:

% Open-Ended Questions: Researchers typically ask open-ended questions that encourage participants to provide detailed and descriptive responses. These questions are designed to elicit narratives rather than simple yes/no answers.
% Storytelling: Participants are encouraged to tell their stories in a natural and spontaneous manner. This allows for the exploration of the richness and complexity of their experiences.
% Contextual Information: Researchers may ask participants to provide context for their narratives, such as the setting, people involved, and any relevant background information.
% Active Listening: Researchers must actively listen to the narratives and may ask follow-up questions to clarify or delve deeper into specific aspects of the story.
% Transcription and Analysis: After data collection, researchers typically transcribe the narratives and analyze them to identify common themes, patterns, and insights. This analysis can provide valuable qualitative data for research studies.
% Narrative elicitation is commonly used in various fields, including psychology, anthropology, sociology, and qualitative research in general. It can help researchers explore topics such as personal experiences, cultural practices, identity, and the impact of certain events on individuals or communities. Additionally, narrative elicitation can be a valuable tool in clinical settings, where therapists may use it to help patients express their thoughts and emotions.
\section{Crowdsourcing}
Crowdsourcing data collection refers to the practice of gathering information, data, or content from a broad and diverse group of individuals, often facilitated through online platforms. This approach involves harnessing the collective effort of a "crowd" to contribute data points, insights, or responses that are used to address specific research questions, projects, or tasks.

% Crowdsourcing data collection typically follows these steps:

% Designing the Task: Defining the specific data needed and creating a clear and concise set of instructions for contributors. This could involve labelling images, transcribing audio, answering surveys, or any other task that requires human judgment or input.
% Platform Engagement: Utilizing online platforms or websites designed for crowdsourcing, where individuals from different backgrounds and locations can access and participate in the data collection task.
% Submission and Review: Participants submit their contributions through the platform, which are then reviewed for accuracy and quality. This quality control step helps ensure reliable data.
% Aggregation: After data collection, the gathered contributions are aggregated and analyzed to draw conclusions or insights based on the collective input.
% Crowdsourcing data collection offers several benefits, including the ability to gather large volumes of data relatively quickly and at a potentially lower cost compared to traditional methods. It's especially useful for tasks that require human judgment, cultural knowledge, or subjective interpretation. However, careful attention must be paid to ensure data quality and prevent potential biases introduced by the diverse group of contributors.
\section{Large Language Models}
Large language models, such as GPT-3, are a type of artificial intelligence (AI) model designed to understand and generate human-like text based on the input they receive. These models are built upon deep learning techniques, particularly transformer architectures, and are characterized by their vast size, complexity, and ability to handle natural language processing tasks at an impressive scale. Here are some key characteristics and aspects of large language models:

Size and Scale: Large language models are incredibly massive in terms of the number of parameters they contain. for instance, GPT-3, developed by OpenAI, has 175 billion parameters, making it one of the largest language models as of my last knowledge update in September 2021. The size of these models contributes to their ability to process and generate text across various languages and topics.
Pre-trained on Diverse Text: These models are pre-trained on vast and diverse datasets containing text from the internet, books, articles, and more. This pre-training helps them learn grammar, language structure, factual information, and even some reasoning abilities from the data.
General-Purpose: Large language models are general-purpose in the sense that they can be fine-tuned for specific tasks. They can handle a wide range of natural language understanding and generation tasks, including language translation, question answering, text summarization, and chatbot functionality.
Contextual Understanding: These models exhibit contextual understanding, meaning they can analyze the context of a sentence or paragraph to generate coherent and contextually appropriate responses. This is achieved through the use of attention mechanisms and deep neural networks.
Limitations: Despite their capabilities, large language models have limitations. They may produce biased or inappropriate content based on the biases present in their training data. They can also generate plausible-sounding but factually incorrect information. Additionally, the energy consumption and environmental impact of training and running such large models have raised concerns.
Fine-Tuning: Large language models are often fine-tuned on specific datasets or tasks to make them more useful for particular applications. This fine-tuning process customizes the model's behavior for tasks like sentiment analysis or medical diagnosis.
Applications: Large language models have found applications in a wide range of domains, including natural language understanding, content generation, virtual assistants, customer support, content recommendation, and more.
Ethical and Societal Considerations: The development and deployment of large language models raise ethical and societal concerns, including issues related to privacy, bias, misinformation, and responsible AI use. Researchers and organizations are actively working on addressing these challenges.
Large language models represent a significant advancement in natural language processing and have the potential to transform various industries and applications by automating language-related tasks and enhancing human-computer interaction. However, their responsible development and usage are crucial to mitigate potential negative consequences.
\subsection{Prompting}

% Prompting refers to providing a specific stimulus or cue to elicit a response or action from someone or something. In various contexts, prompting can be used to initiate a particular behavior, guide a decision, trigger a response, or inspire creativity. The nature and purpose of prompting can vary widely depending on the context in which it is employed. Here are some common examples:

% Prompting in Education:
% In the field of education, teachers often use prompts to help students learn and recall information. for instance, a teacher might provide a writing prompt to encourage students to start a creative writing assignment.
% In special education, prompting strategies are used to assist individuals with disabilities in acquiring new skills. These prompts can be verbal, visual, or physical cues to help the person perform a specific task.
% Prompting in Human-Computer Interaction:
% In user interfaces and software applications, prompts can guide users through a process or inform them about available actions. For instance, a pop-up message might prompt a user to confirm a critical action.
% Virtual assistants like Siri or Alexa respond to voice prompts from users, initiating actions or providing information based on the user's request.
% Prompting in Behavioral Psychology:
% Behavioral therapists use prompting techniques to encourage desired behaviors in individuals. This can involve verbal cues, visual cues, or physical prompts to help a person complete a task.
% In applied behavior analysis, prompting is used to teach individuals with autism or developmental disabilities new skills, gradually fading out the prompts as the person becomes more proficient.
% Prompting in Marketing:
% In marketing and advertising, prompts are used to encourage consumers to take specific actions, such as making a purchase, signing up for a newsletter, or clicking on a link.
% Email marketing often includes call-to-action (CTA) prompts like "Buy Now" or "Subscribe Today."
% Prompting in Creativity:
% Creative professionals may use prompts to overcome creative blocks or generate new ideas. These prompts can be in the form of words, images, or themes that spark inspiration.
% Writing prompts are commonly used by authors and bloggers to generate content ideas or start writing projects.
% Prompting in Research and Surveys:
% In research, survey questions are prompts that elicit specific information from participants. Researchers carefully design prompts to gather relevant data.
% Qualitative research may use open-ended prompts to encourage participants to share their thoughts and experiences in their own words.
% Prompting is a versatile concept used in various fields and settings to guide, encourage, or facilitate actions, responses, or creative thinking. Effective prompting often requires careful consideration of the audience, context, and desired outcome.
