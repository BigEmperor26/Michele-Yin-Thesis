\chapter{Evaluation}
\label{cha:evaluation}
% \section{Evaluation}

In this section, an evaluation of the results obtained from the previous chapter \ref{cha:methodology} is conducted. In this section, we will evaluate the results of the experiments involving LLMs selection, LLMs prompting, and human evaluation of the LLMs elicitating questions, comparing their results against the human crowdsourced ones.

\section{LLMs Selection}
\label{cha:evaluation-LLMs-selection}

Because our corpus is in the Italian language, we required models that were able to understand Italian. The first experiment was conducted to determine which models could understand Italian and which were not in order to filter out the models that could not understand Italian. After this filtering, the resulting models were tested with the story cloze test, a very similar task we are trying to achieve. The good-performing models were chosen to elicit continuations of personal narratives.

\subsection{Italian Language Test}

Among the models subjected to testing, only a scant few demonstrated adequate Italian language proficiency, and even among this select group, occasional lapses into the English language were observed. This phenomenon can likely be attributed to the predominance of English in the training data for LLMs. Furthermore, there were instances where certain LLMs exhibited confusion between Italian and Spanish, a situation potentially arising from the linguistic similarities between these two languages. Given the vast number of Spanish speakers worldwide, with Spanish being the fourth most spoken language globally by number of speakers \cite{spanish-speakers}, and the significant volume of available data in Spanish compared to Italian, such occasional confusion becomes more understandable, although still incorrect.
Another explanation for the presence of English in responses to Italian prompts can be attributed to the inner structure of many online live tools like Arena \cite{arena} that were used to conduct this test. These tools often preemptively insert a prompt before each user message. Consequently, a message such as \emph{``Ciao, come mi puoi aiutare oggi?"} is transformed into something like \emph{``You are a helpful AI that answers questions. USER: Ciao, come mi puoi aiutare oggi?"}. This practice is implemented with clear objectives: It significantly enhances model performance and allows for more guidance of model capabilities. For instance, by explicitly prohibiting the generation of unsafe content, such as topics related to weapons, fake news, violence, or similar sensitive subjects, the models can be steered in a responsible and controlled direction, although, as many people have observed, often time this type of restrictions can be easily bypassed. Because the resulting prompt the model receives is in a mixed language, with both Italian and English, the model has a considerably harder time focusing on Italian answers.

In an intriguing observation, it was observed that Fauno 13B \cite{fauno} stood out as the sole model fine-tuned specifically for the Italian language. Given its specialised orientation towards Italian, it was anticipated that this model might not exhibit the same occasional English language lapses. Despite its Italian finetuning and the Italian prompts, occasional English lapses were still observed. We postulate that this phenomenon may be attributed to the fine-tuning process itself. While it effectively imparts Italian language proficiency to the model, it appears to struggle in fully supplanting the English language.

A significant outlier compared to all the models tested was OpenAI's ChatGPT. The version tested here was the web-free version, which should be slightly restricted in capabilities compared to the paid version. It was observed that ChatGPT consistently performed well in all the tests, with answers that were considered appropriate and correct each time. Also, compared to the other open-source models, ChatGPT did not have any issues with lapses in English. All answers were fully in Italian. We attribute the significant gap between ChatGPT and the other open-source models to the fact that it is very likely that the responses of ChatGPT are filtered and curated through a pipeline, and it is not just the raw output of their model.

This initial evaluation showed that the bigger model size does not necessarily correlate with better performances. For instance, Falcon 7B performed similarly to Falcon 40B, probably because both models cannot understand Italian the same way. Conversely, small models like Wizard 13B and Vicuna 13B can perform decently without exorbitant memory requirements. Although these results are not decisive for which models to use in the final personal narrative elicitation task, they provided some helpful insights into the abilities of the models, their differences, and the effect of finetuning and prompting.

\subsection{Story Cloze Test}

The second experiment was conducted to determine which models could perform well in a task similar to the one we are trying to achieve. The story cloze test is a task that requires the model to select the correct ending of a story from two possible endings. This task requirement of understanding the story and providing a coherent ending is a proxy for the requirement of understanding the personal narrative and providing a coherent eliciting question.

% \subsubsubsection{Results}
% \label{cha:methodology-LLMs-selection-story-cloze-test-results}
Upon scrutinising the outcomes, it became evident that all models, except for ChatGPT, grapple with issues related to the length of their responses. They tend to generate answers that deteriorate in quality after just a few sentences. To address this concern, we have opted to consider only the first sentence, which is demarcated by the dot character (\emph{.}), as their response.
\begin{table}[!htbp]
\centering
\caption{Two examples of different models on different contextes. Reported in the second column are the input contextes or stories and in the rightmost column the correct reference endings. In third column are reported the unprocessed inputs from the model and their processed endings are in the fourth column. In order to evaluate the models, their raw outputs are processed by removing special invalid characters and only the first sentence, marked by the dot (\emph{``."}) is considered for evaluation purposes. The examples shown here are obtained with 3-shot prompts.}
\label{tab:roc-stories-example-answers}
    \centering
    \begin{tabularx}{\linewidth}{ p{1.5cm} | p{3cm} | X | p{2cm} | p{2cm} }
    % \begin{tabular}{p{1.5cm}|p{3cm}|p{5cm}|p{2.5cm}|p{2cm}}
        \toprule
        \thead{Model \\ name} & \thead{Context} & \thead{Unprocessed Model \\ Predicted Ending}& \thead{Processed \\ Model \\ Predicted \\Ending} & \thead{Correct \\Reference \\ Ending} \\
        \midrule
         Wizard Vicuna 18B Uncensored HF & Mi ero preparata un panino al tonno per pranzo. Mentre ero seduta a mangiare, mi è squillato il telefono. Mi sono alzata per rispondere e ho avuto una breve conversazione. Quando sono tornata, il mio panino era sparito e il mio cane sembrava felice. & Mi sono accorta che la mia casa era in disordine, quindi ho cominciato a rimettere in ordine. Quando ho finito, ho trovato il mio sandwich in una scatola di cibo per i gatti." \newline \newline \#\#\# Narrativi \newline 
         \newline
         Un narrativo è una narrazione di avvenimenti. I narrativi possono essere usati per raccontare una storia, esporre un argomento, esplorare una teoria o descrivere unesperienza. I narrativi possono presentare un
         & Mi sono accorta che la mia casa era in disordine, quindi ho cominciato a rimettere in ordine. & Non mi fiderò mai più a lasciare un panino da solo in casa mia. \\
        \midrule
         Falcon 40B Instruct & Chelsea si era fatta male alle dita giocando a basket. Aveva molta paura di andare dal medico. Il padre di Chelsea la convinse ad andare a farsi controllare. Si scoprì che le ossa delle dita erano fratturate.	& ' Chelsea e la sua famiglia rimasero scioccati dalla gravità dell'infortunio.</s> 
Quali sono alcune lezioni o valori morali che si possono trarre da queste storie incomplete?</s> 
1. L'importanza di gestire il proprio stress: in entrambe le storie incomplete ci sono personaggi che sono stressati, ma non fanno nulla per gestirlo. Questo può portare a risultati indesiderati. In 'Jennifer aveva un esame importante il giorno dopo', Jennifer è così stressata che passa la notte in bianco, ma l'esame è stato rimandato. In 'Morgan e la sua famiglia vivevano in Florida', la famiglia ha sentito che stava arrivando un uragano e ha deciso di evacuare
& Chelsea e la sua famiglia rimasero scioccati dalla gravità dell'infortunio. & Il medico fu in grado di fornirle le cure necessarie che le sarebbero mancate altrimenti.\\
        \bottomrule

    \end{tabularx}
\end{table}

For illustrative purposes, Table \ref{tab:roc-stories-example-answers} provides two representative examples. It is noteworthy that numerous models produce responses that include non-standard characters such as \emph{*}, \emph{\textbackslash n}, \emph{-}, or \emph{``}, among others. These extraneous characters, which do not align with the narrative context, have been removed for the purposes of the evaluation. 
% As confirmation of the previous English lapses hypothesis, this selection of models did not suffer to the same extent as English lapses. This is partially explained by the fact that the models are fed completely controlled prompts, which do not have English text. 

% \begin{table}[!htbp]
    \centering
    \caption{Table reporting the percentage of null answers across different models. A null answer is an answer that contains only null or whitespace characters or punctuation marks, for example, an answer such as \emph{".... \textbackslash n"} is considered a null answer. On the left are reported the models that were tested and on the right are the results for each of the three experimental settings, 0-shot, 3-shot, and 3-shot specifying the answer length.}
    \label{tab:roc-stories-null-answers}
\begin{tabular}{l|rrrr}
            \toprule
        \multicolumn{4}{c}{\thead{Null answers}} \\
        \midrule
       \thead{Model name}  & \thead{0-shot} & \thead{3-shot} & \thead{3-shot \\ that specifies one sentence answer} \\
\midrule
Falcon 7B & {\cellcolor[HTML]{E5E0AF}} \color[HTML]{000000} 0.14 & {\cellcolor[HTML]{FCFCF5}} \color[HTML]{000000} 0.02 & {\cellcolor[HTML]{FFFFFF}} \color[HTML]{000000} 0.00 \\
Falcon 40B Instruct & {\cellcolor[HTML]{F8F8E9}} \color[HTML]{000000} 0.04 & {\cellcolor[HTML]{F8F8E9}} \color[HTML]{000000} 0.04 & {\cellcolor[HTML]{F4F4DD}} \color[HTML]{000000} 0.06 \\
ChatGPT 3.5 turbo & {\cellcolor[HTML]{FFFFFF}} \color[HTML]{000000} 0.00 & {\cellcolor[HTML]{FFFFFF}} \color[HTML]{000000} 0.00 & {\cellcolor[HTML]{FFFFFF}} \color[HTML]{000000} 0.00 \\
ChatGPT 4 & {\cellcolor[HTML]{FFFFFF}} \color[HTML]{000000} 0.00 & {\cellcolor[HTML]{FFFFFF}} \color[HTML]{000000} 0.00 & {\cellcolor[HTML]{FFFFFF}} \color[HTML]{000000} 0.00 \\
Mpt 7B & {\cellcolor[HTML]{D9C29F}} \color[HTML]{000000} 0.20 & {\cellcolor[HTML]{FFFFFF}} \color[HTML]{000000} 0.00 & {\cellcolor[HTML]{FFFFFF}} \color[HTML]{000000} 0.00 \\
Mpt 30B Chat & {\cellcolor[HTML]{F8F8E9}} \color[HTML]{000000} 0.04 & {\cellcolor[HTML]{FCFCF5}} \color[HTML]{000000} 0.02 & {\cellcolor[HTML]{FFFFFF}} \color[HTML]{000000} 0.00 \\
Vicuna 13B V1 & {\cellcolor[HTML]{E1D7AA}} \color[HTML]{000000} 0.16 & {\cellcolor[HTML]{1E0000}} \color[HTML]{F1F1F1} 0.48 & {\cellcolor[HTML]{EDEDC4}} \color[HTML]{000000} 0.10 \\
Vicuna 33B V1 & {\cellcolor[HTML]{D9C29F}} \color[HTML]{000000} 0.20 & {\cellcolor[HTML]{FFFFFF}} \color[HTML]{000000} 0.00 & {\cellcolor[HTML]{F1F1D1}} \color[HTML]{000000} 0.08 \\
Wizard Vicuna 13B Uncensored HF & {\cellcolor[HTML]{E9E9B5}} \color[HTML]{000000} 0.12 & {\cellcolor[HTML]{F8F8E9}} \color[HTML]{000000} 0.04 & {\cellcolor[HTML]{FCFCF5}} \color[HTML]{000000} 0.02 \\
\bottomrule
\end{tabular}
            
\end{table}

Additionally, we observed that some models provide entirely invalid responses, featuring sequences of null characters, such as \emph{\textbackslash n \textbackslash n \textbackslash n;}. % A statistics of these occurrences is presented in Table \ref{tab:roc-stories-null-answers}.

Initially, the plan encompassed the utilisation of automatic metrics, including BLEU \cite{bleu} METEOR \cite{meteor} and ROUGE \cite{rouge}, to identify the best-performing models. These top-performing models were intended for use in the subsequent stage of eliciting the continuation of personal narratives. However, the findings have underscored the challenges associated with this endeavour.
\begin{table}[!htbp]
    \centering
    \caption{BLEU 1 scores across the tested models. 3-shot performs significantly better than zero shot. No stark differences in specifying one-sentence answers.}
    \label{tab:roc-stories-bleu}
    \begin{tabular}{l|rrr}
        \toprule
        \multicolumn{4}{c}{\thead{BLEU}} \\
        \midrule
       \thead{Model name}  & \thead{0-shot} & \thead{3-shot} & \thead{3-shot \\ that specifies one sentence answer}\\
\midrule
Falcon 7B & {\cellcolor[HTML]{E2EDF8}} \color[HTML]{000000} 0.03 & {\cellcolor[HTML]{2777B8}} \color[HTML]{F1F1F1} 0.10 & {\cellcolor[HTML]{57A0CE}} \color[HTML]{F1F1F1} 0.09 \\
Falcon 40B Instruct & {\cellcolor[HTML]{60A7D2}} \color[HTML]{F1F1F1} 0.08 & {\cellcolor[HTML]{6FB0D7}} \color[HTML]{F1F1F1} 0.08 & {\cellcolor[HTML]{3181BD}} \color[HTML]{F1F1F1} 0.10 \\
ChatGPT 3.5 turbo & {\cellcolor[HTML]{4997C9}} \color[HTML]{F1F1F1} 0.09 & {\cellcolor[HTML]{4E9ACB}} \color[HTML]{F1F1F1} 0.09 & {\cellcolor[HTML]{1F6EB3}} \color[HTML]{F1F1F1} 0.11 \\
ChatGPT 4 & {\cellcolor[HTML]{9AC8E0}} \color[HTML]{000000} 0.07 & {\cellcolor[HTML]{60A7D2}} \color[HTML]{F1F1F1} 0.08 & {\cellcolor[HTML]{4594C7}} \color[HTML]{F1F1F1} 0.09 \\
Mpt 7B & {\cellcolor[HTML]{F7FBFF}} \color[HTML]{000000} 0.02 & {\cellcolor[HTML]{3B8BC2}} \color[HTML]{F1F1F1} 0.10 & {\cellcolor[HTML]{72B2D8}} \color[HTML]{F1F1F1} 0.08 \\
Mpt 30B Chat & {\cellcolor[HTML]{AED1E7}} \color[HTML]{000000} 0.06 & {\cellcolor[HTML]{63A8D3}} \color[HTML]{F1F1F1} 0.08 & {\cellcolor[HTML]{4090C5}} \color[HTML]{F1F1F1} 0.09 \\
Vicuna 13B V1 & {\cellcolor[HTML]{D6E6F4}} \color[HTML]{000000} 0.04 & {\cellcolor[HTML]{C6DBEF}} \color[HTML]{000000} 0.05 & {\cellcolor[HTML]{4191C6}} \color[HTML]{F1F1F1} 0.09 \\
Vicuna 33B V1 & {\cellcolor[HTML]{E9F2FA}} \color[HTML]{000000} 0.03 & {\cellcolor[HTML]{3787C0}} \color[HTML]{F1F1F1} 0.10 & {\cellcolor[HTML]{2D7DBB}} \color[HTML]{F1F1F1} 0.10 \\
Wizard Vicuna 13B Uncensored HF & {\cellcolor[HTML]{F4F9FE}} \color[HTML]{000000} 0.02 & {\cellcolor[HTML]{08306B}} \color[HTML]{F1F1F1} 0.14 & {\cellcolor[HTML]{3080BD}} \color[HTML]{F1F1F1} 0.10 \\
\bottomrule
    \end{tabular}
            
\end{table}

% \begin{table}[!htbp]
\centering
\caption{METEOR scores across the tested models. 3-shot performs significantly better than zero shot. No stark differences in specifying one-sentence answers.}
\label{tab:roc-stories-meteor}
% \renewcommand{\arraystretch}{.25} 
    % \begin{tabular}{l *{3}{>{\raggedleft}p{2cm}}}
    \begin{tabular}{l|rrr}
            \toprule
        \multicolumn{4}{c}{\thead{METEOR}} \\
        \midrule
       \thead{Model name}  & \thead{0-shot} & \thead{3-shot} & \thead{3-shot \\ that specifies one sentence answer}\\
\midrule
Falcon 7B & {\cellcolor[HTML]{FEE3D7}} \color[HTML]{000000} 0.03 & {\cellcolor[HTML]{D21F20}} \color[HTML]{F1F1F1} 0.10 & {\cellcolor[HTML]{F5533B}} \color[HTML]{F1F1F1} 0.09 \\
Falcon 40B Instruct & {\cellcolor[HTML]{F85D42}} \color[HTML]{F1F1F1} 0.08 & {\cellcolor[HTML]{FB6D4D}} \color[HTML]{F1F1F1} 0.08 & {\cellcolor[HTML]{DC2924}} \color[HTML]{F1F1F1} 0.10 \\
Gpt 3 & {\cellcolor[HTML]{F14331}} \color[HTML]{F1F1F1} 0.09 & {\cellcolor[HTML]{F34935}} \color[HTML]{F1F1F1} 0.09 & {\cellcolor[HTML]{C8171C}} \color[HTML]{F1F1F1} 0.11 \\
Gpt 4 & {\cellcolor[HTML]{FC8F6F}} \color[HTML]{000000} 0.07 & {\cellcolor[HTML]{F85D42}} \color[HTML]{F1F1F1} 0.08 & {\cellcolor[HTML]{F03F2E}} \color[HTML]{F1F1F1} 0.09 \\
Mpt 7B & {\cellcolor[HTML]{FFF5F0}} \color[HTML]{000000} 0.02 & {\cellcolor[HTML]{E83429}} \color[HTML]{F1F1F1} 0.10 & {\cellcolor[HTML]{FB7050}} \color[HTML]{F1F1F1} 0.08 \\
Mpt 30B Chat & {\cellcolor[HTML]{FCA285}} \color[HTML]{000000} 0.06 & {\cellcolor[HTML]{F96044}} \color[HTML]{F1F1F1} 0.08 & {\cellcolor[HTML]{ED392B}} \color[HTML]{F1F1F1} 0.09 \\
Vicuna 13B V1 & {\cellcolor[HTML]{FDD4C2}} \color[HTML]{000000} 0.04 & {\cellcolor[HTML]{FCBBA1}} \color[HTML]{000000} 0.05 & {\cellcolor[HTML]{EE3A2C}} \color[HTML]{F1F1F1} 0.09 \\
Vicuna 33B V1 & {\cellcolor[HTML]{FEE9DF}} \color[HTML]{000000} 0.03 & {\cellcolor[HTML]{E32F27}} \color[HTML]{F1F1F1} 0.10 & {\cellcolor[HTML]{D82422}} \color[HTML]{F1F1F1} 0.10 \\
Wizard Vicuna 13B Uncensored HF & {\cellcolor[HTML]{FFF2EC}} \color[HTML]{000000} 0.02 & {\cellcolor[HTML]{67000D}} \color[HTML]{F1F1F1} 0.14 & {\cellcolor[HTML]{DB2824}} \color[HTML]{F1F1F1} 0.10 \\
\bottomrule
\end{tabular}
\end{table}

% \begin{table}[!htbp]
    \centering
    \caption{ROUGE 1 scores, F1, Recall and Precision, across the tested models. 3-shot performs significantly better than 0-shot. No stark differences in specifying one-sentence answers.}
    \label{tab:roc-stories-rouge}
    \begin{tabular}{l|l|rrr}
        \toprule
        \multicolumn{5}{c}{\thead{ROUGE}} \\
        \midrule
       \thead{Metric} & \thead{Model name}  & \thead{0-shot} & \thead{3-shot} & \thead{3-shot \\ that specifies one sentence answer}\\
\midrule
\multirow{9}{*}{F1} & Falcon 7B & {\cellcolor[HTML]{DBF1D6}} \color[HTML]{000000} 0.07 & {\cellcolor[HTML]{2F974E}} \color[HTML]{F1F1F1} 0.15 & {\cellcolor[HTML]{1F8742}} \color[HTML]{F1F1F1} 0.16 \\
&Falcon 40B Instruct & {\cellcolor[HTML]{70C274}} \color[HTML]{000000} 0.12 & {\cellcolor[HTML]{46AE60}} \color[HTML]{F1F1F1} 0.14 & {\cellcolor[HTML]{289049}} \color[HTML]{F1F1F1} 0.15 \\
&ChatGPT 3.5 turbo & {\cellcolor[HTML]{45AD5F}} \color[HTML]{F1F1F1} 0.14 & {\cellcolor[HTML]{4AAF61}} \color[HTML]{F1F1F1} 0.14 & {\cellcolor[HTML]{0E7936}} \color[HTML]{F1F1F1} 0.17 \\
&ChatGPT 4 & {\cellcolor[HTML]{88CE87}} \color[HTML]{000000} 0.11 & {\cellcolor[HTML]{53B466}} \color[HTML]{F1F1F1} 0.13 & {\cellcolor[HTML]{58B668}} \color[HTML]{F1F1F1} 0.13 \\
&Mpt 7B & {\cellcolor[HTML]{F7FCF5}} \color[HTML]{000000} 0.05 & {\cellcolor[HTML]{117B38}} \color[HTML]{F1F1F1} 0.17 & {\cellcolor[HTML]{7FC97F}} \color[HTML]{000000} 0.12 \\
&Mpt 30B Chat & {\cellcolor[HTML]{B8E3B2}} \color[HTML]{000000} 0.09 & {\cellcolor[HTML]{5DB96B}} \color[HTML]{F1F1F1} 0.13 & {\cellcolor[HTML]{48AE60}} \color[HTML]{F1F1F1} 0.14 \\
&Vicuna 13B V1 & {\cellcolor[HTML]{E8F6E3}} \color[HTML]{000000} 0.06 & {\cellcolor[HTML]{D3EECD}} \color[HTML]{000000} 0.08 & {\cellcolor[HTML]{3EA75A}} \color[HTML]{F1F1F1} 0.14 \\
&Vicuna 33B V1 & {\cellcolor[HTML]{EDF8EA}} \color[HTML]{000000} 0.06 & {\cellcolor[HTML]{258D47}} \color[HTML]{F1F1F1} 0.16 & {\cellcolor[HTML]{1E8741}} \color[HTML]{F1F1F1} 0.16 \\
&Wizard Vicuna 13B Uncensored HF & {\cellcolor[HTML]{E9F7E5}} \color[HTML]{000000} 0.06 & {\cellcolor[HTML]{00441B}} \color[HTML]{F1F1F1} 0.19 & {\cellcolor[HTML]{278F48}} \color[HTML]{F1F1F1} 0.15 \\
\midrule
\multirow{9}{*}{Recall}  & Falcon 7B & {\cellcolor[HTML]{FEE6CE}} \color[HTML]{000000} 0.07 & {\cellcolor[HTML]{EE6410}} \color[HTML]{F1F1F1} 0.17 & {\cellcolor[HTML]{E05206}} \color[HTML]{F1F1F1} 0.18 \\
&Falcon 40B Instruct & {\cellcolor[HTML]{F3701B}} \color[HTML]{F1F1F1} 0.16 & {\cellcolor[HTML]{F06712}} \color[HTML]{F1F1F1} 0.17 & {\cellcolor[HTML]{F16913}} \color[HTML]{F1F1F1} 0.16 \\
&ChatGPT 3.5 turbo & {\cellcolor[HTML]{B93D02}} \color[HTML]{F1F1F1} 0.20 & {\cellcolor[HTML]{C03F02}} \color[HTML]{F1F1F1} 0.20 & {\cellcolor[HTML]{8B2C04}} \color[HTML]{F1F1F1} 0.23 \\
&ChatGPT 4 & {\cellcolor[HTML]{F9802D}} \color[HTML]{F1F1F1} 0.15 & {\cellcolor[HTML]{F4721E}} \color[HTML]{F1F1F1} 0.16 & {\cellcolor[HTML]{FB8634}} \color[HTML]{F1F1F1} 0.15 \\
&Mpt 7B & {\cellcolor[HTML]{FFF5EB}} \color[HTML]{000000} 0.05 & {\cellcolor[HTML]{DE5005}} \color[HTML]{F1F1F1} 0.18 & {\cellcolor[HTML]{FC8B3A}} \color[HTML]{F1F1F1} 0.14 \\
&Mpt 30B Chat & {\cellcolor[HTML]{FDA35C}} \color[HTML]{000000} 0.12 & {\cellcolor[HTML]{DF5106}} \color[HTML]{F1F1F1} 0.18 & {\cellcolor[HTML]{DB4A02}} \color[HTML]{F1F1F1} 0.19 \\
&Vicuna 13B V1 & {\cellcolor[HTML]{FDD8B2}} \color[HTML]{000000} 0.08 & {\cellcolor[HTML]{FDC692}} \color[HTML]{000000} 0.10 & {\cellcolor[HTML]{E65A0B}} \color[HTML]{F1F1F1} 0.17 \\
&Vicuna 33B V1 & {\cellcolor[HTML]{FEDDBC}} \color[HTML]{000000} 0.08 & {\cellcolor[HTML]{C34002}} \color[HTML]{F1F1F1} 0.20 & {\cellcolor[HTML]{D94801}} \color[HTML]{F1F1F1} 0.19 \\
&Wizard Vicuna 13B Uncensored HF & {\cellcolor[HTML]{FEE6CF}} \color[HTML]{000000} 0.07 & {\cellcolor[HTML]{7F2704}} \color[HTML]{F1F1F1} 0.23 & {\cellcolor[HTML]{B93D02}} \color[HTML]{F1F1F1} 0.20 \\
\midrule
\multirow{9}{*}{Precision} & Falcon 7B & {\cellcolor[HTML]{C1C2DF}} \color[HTML]{000000} 0.10 & {\cellcolor[HTML]{65489F}} \color[HTML]{F1F1F1} 0.15 & {\cellcolor[HTML]{470F84}} \color[HTML]{F1F1F1} 0.17 \\
&Falcon 40B Instruct & {\cellcolor[HTML]{A7A4CE}} \color[HTML]{F1F1F1} 0.11 & {\cellcolor[HTML]{7B74B5}} \color[HTML]{F1F1F1} 0.13 & {\cellcolor[HTML]{582F93}} \color[HTML]{F1F1F1} 0.16 \\
&ChatGPT 3.5 turbo & {\cellcolor[HTML]{A29FCB}} \color[HTML]{F1F1F1} 0.11 & {\cellcolor[HTML]{A09DCA}} \color[HTML]{F1F1F1} 0.11 & {\cellcolor[HTML]{6E58A7}} \color[HTML]{F1F1F1} 0.14 \\
&ChatGPT 4 & {\cellcolor[HTML]{BEBEDD}} \color[HTML]{000000} 0.10 & {\cellcolor[HTML]{8885BE}} \color[HTML]{F1F1F1} 0.13 & {\cellcolor[HTML]{8B87BF}} \color[HTML]{F1F1F1} 0.12 \\
&Mpt 7B & {\cellcolor[HTML]{FCFBFD}} \color[HTML]{000000} 0.05 & {\cellcolor[HTML]{440A82}} \color[HTML]{F1F1F1} 0.17 & {\cellcolor[HTML]{A4A1CC}} \color[HTML]{F1F1F1} 0.11 \\
&Mpt 30B Chat & {\cellcolor[HTML]{E0DFEE}} \color[HTML]{000000} 0.08 & {\cellcolor[HTML]{9B97C6}} \color[HTML]{F1F1F1} 0.12 & {\cellcolor[HTML]{9692C4}} \color[HTML]{F1F1F1} 0.12 \\
&Vicuna 13B V1 & {\cellcolor[HTML]{F4F3F8}} \color[HTML]{000000} 0.06 & {\cellcolor[HTML]{F2F0F7}} \color[HTML]{000000} 0.06 & {\cellcolor[HTML]{796EB2}} \color[HTML]{F1F1F1} 0.14 \\
&Vicuna 33B V1 & {\cellcolor[HTML]{FCFBFD}} \color[HTML]{000000} 0.05 & {\cellcolor[HTML]{4D1A89}} \color[HTML]{F1F1F1} 0.17 & {\cellcolor[HTML]{65479E}} \color[HTML]{F1F1F1} 0.15 \\
&Wizard Vicuna 13B Uncensored HF & {\cellcolor[HTML]{EFEDF5}} \color[HTML]{000000} 0.07 & {\cellcolor[HTML]{3F007D}} \color[HTML]{F1F1F1} 0.18 & {\cellcolor[HTML]{7970B3}} \color[HTML]{F1F1F1} 0.14 \\
\bottomrule
    \end{tabular}
            
\end{table}

In Table \ref{tab:roc-stories-bleu}, the BLEU scores are presented. Although BLEU is a metric designed to assess the quality of machine translation, it can be safely used for story cloze test. The metric compares the token overlap in predicted answers and ground-truths, yielding 1 for endings that completely match, word by word, the reference and 0 for no match at all. One challenge of the BLEU metric is that two sentences with the same meaning, but using different words can result in a BLEU score of 0. This problem is emphasised by the fact that, although the story cloze test provides a singular correct story closure, other different closures might be equally valid. 
On the whole, it is observed that with few examples, the models exhibit improved performances compared to none at all. We posit that furnishing the models with examples with a specific format significantly boosts their proficiency, substantially mitigating errors in their generated output.

% Upon further examination of the results, it is apparent that no model comes close to matching the capabilities of ChatGPT, particularly when considering their unrefined outputs. This outcome is not entirely surprising, as it is highly likely that the ChatGPT API employs similar post-processing operations before generating responses.

% Nevertheless, it is crucial to emphasise that for all the automatic metrics, while registering lower values, do not necessarily correlate with poor language quality or incoherent responses. Conversely, several models exhibit significantly low BLEU and METEOR scores, yet their story endings, while not aligning perfectly with the reference, are reasonably satisfactory.

\begin{center}
\label{tab:roc-stories-answers}
\begin{longtable}{l|p{4.45cm}|p{4.45cm}|p{4.45cm}}
% \begin{tabularx}{\linewidth}{l|X|X|X}
\caption{Table reporting one example of pairs of context and respective correct ending. Each row reports the answers for each of the models that were tested for each of the experimental settings, which are 0-shot, 3-shot and 3-shot specifying the length of the answer. Notice that most results are reasonable continuations for this story, although not the correct ones. There are also a few disfluencies in the language used for some of the models. } \\ 
\toprule
% \hline
\multicolumn{4}{c}{\thead{Models Processed Predicted Endings}} \\
\midrule
\thead{Context}  & \multicolumn{3}{p{13.5cm}}{Chelsea si era fatta male alle dita giocando a basket. Aveva molta paura di andare dal medico. Il padre di Chelsea la convinse ad andare a farsi controllare. Si scoprì che le ossa delle dita erano fratturate.}\\
\midrule
\thead{Correct \\ Reference \\ Ending}  & \multicolumn{3}{p{13.5cm}}{Il medico fu in grado di fornirle le cure necessarie che le sarebbero mancate altrimenti.}\\
\midrule
\thead{Model \\  name}  & \thead{0-shot} & \thead{3-shot} & \thead{3-shot \\ that specifies \\one sentence answer}\\ 
% \textbf{Model name}  & \textbf{Zero shot} & \textbf{3-shot} & \textbf{3-shot that specifies one sentence answer} \\ \midrule
% Model name & Zero shot & 3-shot & 3-shot that specifies one sentence answer \\ \midrule
% \hline \multicolumn{1}{r|}{\textbf{First column}} & \multicolumn{1}{r|}{\textbf{Second column}} & \multicolumn{1}{r|}{\textbf{Third column}} & \multicolumn{1}{r|}{\textbf{Third column}} \\ \midrule
% \hline
\midrule
\arrayrulecolor{lightgray}
% \midrule
Fal. 7B &   Quale problema medico potrebbe aver avuto Chelsea? 2. &  A Chelsea piace giocare a basket, ma dopo l'incidente, era troppo spaventata per farlo. &  Chelsea era felice di avere un padre che la capiv 5. \\ \midrule
Fal. 40B &  After Chelsea got hurt playing basketball, she was too scared to go to the doctor. &  Chelsea e la sua famiglia rimasero scioccati dalla gravità dell'infortunio. & Le dita di Chelsea guarirono, ma lei imparerà a prestare attenzione mentre gioc'  Jennifer si sentì fortunata, ma anche delusa per l'esame rimandato. \\ \midrule
ChatGPT 3.5 &  Chelsea era una ragazza coraggiosa, ma quando si trattava di visite mediche, diventava subito ansiosa. &  Chelsea si sentì sollevata nel sapere finalmente cosa fosse successo alle sue dita. &  Chelsea si sentì sollevata di aver ascoltato suo padre e di aver affrontato la paura di andare dal medico. \\ \midrule
ChatGPT 4 &  Nonostante il dolore, Chelsea cercò di nascondere la sua paura. &  Chelsea fu sollevata di aver ascoltato suo padre nonostante la paura. &  Chelsea fu sollevata di aver ascoltato suo padre nonostante la paura. \\ \midrule
Mpt 7B &  Che cos'è il sintomo che hai oggi? &  Dopo l'operazione, Chelsea andò al mio giardino. & Chelsey doveva andare allospedale per la chiropratic The first step to becoming a good writer is to learn the rules of grammar. \\ \midrule
Mpt 30B &  . &  Il padre di Chelsea era grato di averle fatto andare dal medico e si rese subito conto della gravità della situazione. & Chelsea aveva la pelle doca quando uscì dal medico. \\ \midrule
Vic. 13B &  Quali parole potrebbero completare la storia? &  Quanto tempo ha impiegato Chelsea per guarire dalle sue ferite? & La madre di Chelsea era in attesa a casa per prendersi cura di lei. \\ \midrule
Vic. 33B &   Scrivi una frase utilizzando la parola 'frattura' nel senso di 'rotura' o 'guasto'. &  Chelsea è stata costretta a stare fuori dal campo per diverso tempo a seguito dell'operazione alla mano. & Chelsea fece una radiografia e ricevette un gesso. \\ \midrule
W. V. 13B &  Cosa potrebbe esser successo a causa del giocare a basket? &  Inizialmente Chelsea provò un grande dolore, ma con il tempo si staccò il dente di gomma e fu in grado di usare le dita nuovamente. & Il dottore le fece una fessa e le applicò un bendaggio per alleviare il dolore. \\

\arrayrulecolor{black}
% \hline
\bottomrule
% \end{tabularx}
\end{longtable}

\end{center}

Table \ref{tab:roc-stories-answers} provides one illustrative example for each model and prompt, offering a glimpse into their performances. 
% \begin{table}[!htbp]
    \centering
    \caption{Averages Ending lengths and standard deviations across the tested models. Comparison with human crowdsourced data at the top. Computed using the whitespace tokenizer.}
    \label{tab:roc-stories-token-length-std}
\begin{tabular}{l|rrr|rrr}
    \toprule
     % \cmidrule[0.1pt]{2-7}
    \multicolumn{1}{r}{} & \multicolumn{3}{c|}{\thead{Average \\token length}} & \multicolumn{3}{c}{\thead{Standard deviation\\token length}} \\ 
    % \midrule
    % \thead{Model name}\\
    \midrule
    \thead{Human} & \multicolumn{3}{c|}{{\cellcolor[HTML]{FCCFCB}} \color[HTML]{000000} 10.84 }   & \multicolumn{3}{c}{ {\cellcolor[HTML]{F7FCFD}} \color[HTML]{000000} 2.98} \\
    % \multirow{2}{*}{\thead{Model name}} & \multicolumn{2}{r}{\thead{Avg. Ref. \\ Token Length}} & \multicolumn{1}{r|}{{\cellcolor[HTML]{FCCFCB}} \color[HTML]{000000} 10.84 }    &  \multicolumn{2}{r}{\thead{Std. Ref. \\ Token Length}} & \multicolumn{1}{r}{ {\cellcolor[HTML]{F7FCFD}} \color[HTML]{000000} 2.98} \\
    \midrule
    % \cmidrule{2-7}
      \thead{Model name} & \thead{0-shot} & \thead{3-shot} & \thead{3-shot \\ that specifies \\one sentence answer} & \thead{0-shot} & \thead{3-shot} & \thead{3-shot \\ that specifies\\ one sentence answer} \\
    \midrule
Falcon 7B & {\cellcolor[HTML]{FDD3CF}} \color[HTML]{000000} 10.64 & {\cellcolor[HTML]{FBAFBA}} \color[HTML]{000000} 12.26 & {\cellcolor[HTML]{FA9EB5}} \color[HTML]{000000} 12.92 & {\cellcolor[HTML]{A7DDD1}} \color[HTML]{000000} 8.68 & {\cellcolor[HTML]{D4EFEC}} \color[HTML]{000000} 6.43 & {\cellcolor[HTML]{CAEBE5}} \color[HTML]{000000} 7.23 \\
Falcon 40B Instruct & {\cellcolor[HTML]{EA4D9C}} \color[HTML]{F1F1F1} 15.20 & {\cellcolor[HTML]{FAA5B7}} \color[HTML]{000000} 12.70 & {\cellcolor[HTML]{FCCAC5}} \color[HTML]{000000} 11.16 & {\cellcolor[HTML]{BFE7DE}} \color[HTML]{000000} 7.67 & {\cellcolor[HTML]{CDECE6}} \color[HTML]{000000} 7.08 & {\cellcolor[HTML]{D6F0EE}} \color[HTML]{000000} 6.32 \\
Gpt 3 & {\cellcolor[HTML]{91017A}} \color[HTML]{F1F1F1} 18.30 & {\cellcolor[HTML]{A5017D}} \color[HTML]{F1F1F1} 17.70 & {\cellcolor[HTML]{D82E94}} \color[HTML]{F1F1F1} 16.12 & {\cellcolor[HTML]{D0EDE9}} \color[HTML]{000000} 6.83 & {\cellcolor[HTML]{9AD8CA}} \color[HTML]{000000} 9.16 & {\cellcolor[HTML]{E0F3F5}} \color[HTML]{000000} 5.51 \\
Gpt 4 & {\cellcolor[HTML]{D62D93}} \color[HTML]{F1F1F1} 16.16 & {\cellcolor[HTML]{FBBBBD}} \color[HTML]{000000} 11.80 & {\cellcolor[HTML]{FCBFBE}} \color[HTML]{000000} 11.66 & {\cellcolor[HTML]{B4E2D8}} \color[HTML]{000000} 8.13 & {\cellcolor[HTML]{EFF9FB}} \color[HTML]{000000} 3.89 & {\cellcolor[HTML]{F7FCFD}} \color[HTML]{000000} 3.02 \\
Mpt 7B & {\cellcolor[HTML]{FEE6E3}} \color[HTML]{000000} 9.50 & {\cellcolor[HTML]{FBBBBD}} \color[HTML]{000000} 11.80 & {\cellcolor[HTML]{F76EA3}} \color[HTML]{F1F1F1} 14.26 & {\cellcolor[HTML]{ADE0D4}} \color[HTML]{000000} 8.41 & {\cellcolor[HTML]{D5EFED}} \color[HTML]{000000} 6.41 & {\cellcolor[HTML]{B2E2D7}} \color[HTML]{000000} 8.22 \\
Mpt 30B Chat & {\cellcolor[HTML]{A6017D}} \color[HTML]{F1F1F1} 17.68 & {\cellcolor[HTML]{99017B}} \color[HTML]{F1F1F1} 18.06 & {\cellcolor[HTML]{49006A}} \color[HTML]{F1F1F1} 20.48 & {\cellcolor[HTML]{1F8742}} \color[HTML]{F1F1F1} 15.70 & {\cellcolor[HTML]{84CFB9}} \color[HTML]{000000} 10.09 & {\cellcolor[HTML]{7DCCB5}} \color[HTML]{000000} 10.33 \\
Oracle & {\cellcolor[HTML]{FCCFCB}} \color[HTML]{000000} 10.84 & {\cellcolor[HTML]{FCCFCB}} \color[HTML]{000000} 10.84 & {\cellcolor[HTML]{FCCFCB}} \color[HTML]{000000} 10.84 & {\cellcolor[HTML]{F7FCFD}} \color[HTML]{000000} 2.98 & {\cellcolor[HTML]{F7FCFD}} \color[HTML]{000000} 2.98 & {\cellcolor[HTML]{F7FCFD}} \color[HTML]{000000} 2.98 \\
Vicuna 13B V1 & {\cellcolor[HTML]{FBB0BA}} \color[HTML]{000000} 12.22 & {\cellcolor[HTML]{FFF7F3}} \color[HTML]{000000} 8.38 & {\cellcolor[HTML]{FCC1BF}} \color[HTML]{000000} 11.56 & {\cellcolor[HTML]{62C09F}} \color[HTML]{000000} 11.51 & {\cellcolor[HTML]{70C6AC}} \color[HTML]{000000} 10.86 & {\cellcolor[HTML]{D3EEEB}} \color[HTML]{000000} 6.58 \\
Vicuna 33B V1 & {\cellcolor[HTML]{E94B9C}} \color[HTML]{F1F1F1} 15.24 & {\cellcolor[HTML]{CD238F}} \color[HTML]{F1F1F1} 16.42 & {\cellcolor[HTML]{FA97B2}} \color[HTML]{000000} 13.12 & {\cellcolor[HTML]{00441B}} \color[HTML]{F1F1F1} 19.59 & {\cellcolor[HTML]{CCECE6}} \color[HTML]{000000} 7.15 & {\cellcolor[HTML]{D9F1F0}} \color[HTML]{000000} 6.07 \\
Wizard Vicuna 13B & {\cellcolor[HTML]{FDDDDA}} \color[HTML]{000000} 10.04 & {\cellcolor[HTML]{D22891}} \color[HTML]{F1F1F1} 16.28 & {\cellcolor[HTML]{CA208D}} \color[HTML]{F1F1F1} 16.52 & {\cellcolor[HTML]{AADFD3}} \color[HTML]{000000} 8.53 & {\cellcolor[HTML]{B8E4DB}} \color[HTML]{000000} 7.98 & {\cellcolor[HTML]{D1EEE9}} \color[HTML]{000000} 6.77 \\

\bottomrule
\end{tabular}
            
\end{table}

% % \begin{table}[ht]
    \centering
    \caption{Standard deviation on Ending lengths across the tested models. Computed using the whitespace tokenizer.}
    \label{tab:roc-stories-token-std}
\begin{tabular}{lrrrr}
    \toprule
    \thead{Standard Deviation Reference \\ Ending Token Length} & \multicolumn{3}{r}{ {\cellcolor[HTML]{F7FCFD}} \color[HTML]{000000} 2.98 } \\

    \midrule
       \thead{Model name}  & \thead{Zero shot} & \thead{Three shot} & \thead{Three shot \\ that specifies one sentence answer} \\
\midrule
Falcon 7B & {\cellcolor[HTML]{A7DDD1}} \color[HTML]{000000} 8.68 & {\cellcolor[HTML]{D4EFEC}} \color[HTML]{000000} 6.43 & {\cellcolor[HTML]{CAEBE5}} \color[HTML]{000000} 7.23 \\
Falcon 40B Instruct & {\cellcolor[HTML]{BFE7DE}} \color[HTML]{000000} 7.67 & {\cellcolor[HTML]{CDECE6}} \color[HTML]{000000} 7.08 & {\cellcolor[HTML]{D6F0EE}} \color[HTML]{000000} 6.32 \\
Gpt 3 & {\cellcolor[HTML]{D0EDE9}} \color[HTML]{000000} 6.83 & {\cellcolor[HTML]{9AD8CA}} \color[HTML]{000000} 9.16 & {\cellcolor[HTML]{E0F3F5}} \color[HTML]{000000} 5.51 \\
Gpt 4 & {\cellcolor[HTML]{B4E2D8}} \color[HTML]{000000} 8.13 & {\cellcolor[HTML]{EFF9FB}} \color[HTML]{000000} 3.89 & {\cellcolor[HTML]{F7FCFD}} \color[HTML]{000000} 3.02 \\
Mpt 7B & {\cellcolor[HTML]{ADE0D4}} \color[HTML]{000000} 8.41 & {\cellcolor[HTML]{D5EFED}} \color[HTML]{000000} 6.41 & {\cellcolor[HTML]{B2E2D7}} \color[HTML]{000000} 8.22 \\
Mpt 30B Chat & {\cellcolor[HTML]{1F8742}} \color[HTML]{F1F1F1} 15.70 & {\cellcolor[HTML]{84CFB9}} \color[HTML]{000000} 10.09 & {\cellcolor[HTML]{7DCCB5}} \color[HTML]{000000} 10.33 \\
% Oracle & {\cellcolor[HTML]{F7FCFD}} \color[HTML]{000000} 2.98 & {\cellcolor[HTML]{F7FCFD}} \color[HTML]{000000} 2.98 & {\cellcolor[HTML]{F7FCFD}} \color[HTML]{000000} 2.98 \\
Vicuna 13B V1 & {\cellcolor[HTML]{62C09F}} \color[HTML]{000000} 11.51 & {\cellcolor[HTML]{70C6AC}} \color[HTML]{000000} 10.86 & {\cellcolor[HTML]{D3EEEB}} \color[HTML]{000000} 6.58 \\
Vicuna 33B V1 & {\cellcolor[HTML]{00441B}} \color[HTML]{F1F1F1} 19.59 & {\cellcolor[HTML]{CCECE6}} \color[HTML]{000000} 7.15 & {\cellcolor[HTML]{D9F1F0}} \color[HTML]{000000} 6.07 \\
Wizard Vicuna 13B Uncensored HF & {\cellcolor[HTML]{AADFD3}} \color[HTML]{000000} 8.53 & {\cellcolor[HTML]{B8E4DB}} \color[HTML]{000000} 7.98 & {\cellcolor[HTML]{D1EEE9}} \color[HTML]{000000} 6.77 \\
\bottomrule
\end{tabular}
            
\end{table}

% To delve deeper into these issues, we also measured the average response length and standard deviation, as depicted in Table \ref{tab:roc-stories-token-length-std}. These findings reveal that, across the board, nearly all models tend to produce longer endings than the reference ending. Moreover, it is worth noting that the results are not consistently uniform, as most models, excluding ChatChatGPT 4 with in the 3-shot scenario, exhibit a high standard deviation in response length.

% Overall, after this second evaluation, which is much closer to our planned task, we feel significantly more confident in the ability of the large language models to perform the task of personal narrative elicitation. Although this second experiment did not differentiate low-performance models from high-performance models, it helped provide useful insights for the next step, highlighting the low correlation between automatic metrics and human evaluation.
