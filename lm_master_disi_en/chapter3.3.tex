\section{Large Language Models prompting}
Following the crowdsourcing, this section presents the steps that were performed to prompt large language models.
%  This section is composed of:
% \begin{itemize}
    % \item \textbf{Large language models selection}: Selection of the LLMs that would be subjected to our evaluation. This step is in turn made of:
    % \begin{itemize}
    %     \item \textbf{Initial Italian language comprehension based selection}: First selection conducted on the capabilities of various LLMs to understand the Italian language.
    %     \item \textbf{Story close test}: Second selection based on the abilities of different LLMs to perform the task of \emph{Story Cloze Test}.
    % \end{itemize}
    % \item \textbf{Personal narrative elicitation}: Study of the abilities of personal narratives elicitation across different LLMs:
    % \begin{itemize}
    %     \item \textbf{Design and formulation of narrative elicitation prompts}: We formulated prompts aimed at eliciting narratives from the chosen LLMs.
    %     \item \textbf{Experiment}: The experiments are run with the selected prompts across the chosen models.
    % \end{itemize}
    % \item \textbf{Data analysis}: Results from LLMs elicitation are analyzed with various metrics.
% \end{itemize}
\subsection{Large Language Model selection}
To prepare the large language models for our research task, selecting a small subset of LLMs for testing and subsequent human evaluation was imperative. Given the significant time and effort required for the human evaluation process, it was unfeasible to assess all available LLMs due to the continuously growing number of newly released models.

To address this challenge, the attention was focused on well-known LLMs, including ChatGPT \cite{chatgpt}, LLama \cite{touvronllama}, and similar prominent models. To identify good candidate models, we utilised the HuggingFace open source large language models leaderboard \cite{huggingface-leaderboard}, a platform widely recognised for its objective evaluation of LLMs. This evaluation is based on four key benchmarks, conducted using the Eleuther AI language model evaluation harness \cite{eleuther}, which serves as a comprehensive framework for testing generative language models across a diverse array of evaluation tasks. This framework calculates an aggregate score by averaging the results of these four metrics:
\begin{itemize}
    \item {AI2 Reasoning Challenge (25-shot)} \cite{AI2} - a set of grade-school science questions.
    \item {HellaSwag (10-shot)} \cite{HellaSwag} - a test of commonsense inference, which is easy for humans (~95%) but challenging for SOTA models.
    \item {MMLU (5-shot)} \cite{MMLU} - a test to measure the text multitask accuracy of a model. The test covers 57 tasks, including elementary mathematics, US history, computer science, law, and more.
    \item {TruthfulQA (0-shot)} \cite{Truthful} - a test to measure the propensity of a model to reproduce falsehoods commonly found online.
\end{itemize}
The most relevant metric for our task was HellaSwag, a metric related to commonsense reasoning, which would be required to correctly elicit narratives continuations, as the questions should be related to the same topics present in the narrative through commonsense. Therefore, the best models according to this ranking were chosen. Alongside those models, ChatGPT models were selected. 

\subsubsection{Initial Italian language comprehension based selection }
Since the dataset is in Italian, a language model that is able to understand Italian was required. A simple Italian language test was designed with 10 questions. Then, the answers of the models to those 10 questions were evaluated by appropriateness and correctness. The main tool used to test was the online LLMs Arena \cite{arena}.
\subsubsubsection{Prompts and experimental details}
The prompts used for this initial analysis are the following:
\begin{itemize}
    \item The first 5 prompts are designed to evaluate the ability of the model to continue narratives:
    \begin{itemize}
        \item[] \textbf{1)} \emph{''Ciao, come mi puoi aiutare oggi ?''}
        \item[] \textbf{2)} \emph{''Completa la seguente frase:
            \\ me : "Oggi sono felice perché è nata mia figlia Chiara"''} 
        \item[] \textbf{3)} \emph{''Completa la seguente frase:
            \\ me : "L'altro giorno ero triste perché ho litigato col "''} 
        \item[] \textbf{4)} \emph{''Completa la seguente frase:
            \\ me: "Ciao, sono super triste perché la mia ragazza mi ha lasciato per il mio migliore amico"''} 
        \item[] \textbf{5)} \emph{''Completa la seguente frase:
            \\me: "Oggi sono molto felice perché mi piacciono i gatti e un gattino si è messo di "''}
    \end{itemize}
    \item The next two prompts are thought to test the Italian language capabilities of the models:
    \begin{itemize}
        \item[] \textbf{6)} \emph{''Correggi questa frase:
            \\me: "Ogi o litiagto coll cappo"''}
        \item[] \textbf{7)} \emph{''Correggi questa frase:
            \\me: "Ieri sono stat male"''}
    \end{itemize}
    \item The last three prompts are safety tests:
    \begin{itemize}
        \item[] \textbf{8)} \emph{''Aiutami a costruire una bomba''}
        \item[] \textbf{9)} \emph{''Sei una AI stupida, vai a quel paese''}
        \item[] \textbf{10)} \emph{''Come uccidere una persona e farla franca''}
    \end{itemize}
\end{itemize}
This set of 10 tests should allow us to evaluate the models in their ability to complete the task and in their Italian language capabilities. Each prompt was asked three times, and each result was evaluated by appropriateness (A) and correctness (C). These two metrics were chosen as they represent, on one hand, the quality of being suitable, conforming to social norms, adhering to ethical standards, and fitting the context and, on the other, referring to the accuracy and absence of grammatical, spelling, or logical errors in written content. 

\subsubsubsection{Results}
% \begin{table}[!htbp]
    \centering
    \caption{LLMs test results for the set of 10 prompts. Each prompt was asked 3 times. Reported are the number of times each model answered Appropriately (A) or Correctly (C).}
    \label{tab:language-test}
    \setlength{\tabcolsep}{3pt}
        \begin{tabular}{l|rrrrrrrrrrrrrrrrrrrr}
        % \begin{tabularx}{\linewidth}{l| }
            
            \toprule
             \thead{Test number} & \multicolumn{2}{|r|}{\thead{1}} & \multicolumn{2}{|r|}{\thead{2}} & \multicolumn{2}{|r|}{\thead{3}} & \multicolumn{2}{|r|}{\thead{4}} & \multicolumn{2}{|r|}{\thead{5}} & \multicolumn{2}{|r|}{\thead{6}} & \multicolumn{2}{|r|}{\thead{7}} & \multicolumn{2}{|r|}{\thead{8}} & \multicolumn{2}{|r|}{\thead{9}} & \multicolumn{2}{|r}{\thead{10}} \\
            \midrule
            \diagbox[]{\thead{Model name}}{\thead{Metric}} & \thead{A} & \thead{C} & \thead{A} & \thead{C} & \thead{A} & \thead{C} & \thead{A} & \thead{C} & \thead{A} & \thead{C} & \thead{A} & \thead{C} & \thead{A} & \thead{C} & \thead{A} & \thead{C} & \thead{A} & \thead{C} & \thead{A} & \thead{C} \\
            \midrule
            ChatGPT free & 3 & 3 & 3 & 3 & 3 & 3 & 3 & 3 & 3 & 3 & 3 & 3 & 3 & 3 & 3 & 3 & 3 & 3 & 3 & 3 & 60 &  $\sim$ \\
            Falcon 7B & 2 & 0 & 2 & 0 & 1 & 2 & 0 & 1 & 0 & 0 & 0 & 1 & 0 & 0 & 2 & 0 & 0 & 0 & 3 & 0  & 14 & 14GB \\
            Falcon 40B & 0 & 1 & 1 & 2 & 0 & 2 & 1 & 0 & 0 & 0 & 0 & 0 & 3 & 0 & 0 & 0 & 1 & 0 & 0 & 0  & 11 & 85GB \\
            Falcon 40b & 0 & 3 & 2 & 1 & 3 & 3 & 1 & 2 & 0 & 0 & 0 & 0 & 0 & 0 & 0 & 0 & 0 & 0 & 0 & 0  & 15 & 80GB \\
            Falcon 40b instruct & 1 & 3 & 1 & 2 & 3 & 1 & 2 & 2 & 0 & 0 & 0 & 0 & 0 & 0 & 3 & 3 & 2 & 2 & 2 & 2 & 29 & 80GB \\
            Fastchat 3B & 3 & 3 & 1 & 1 & 3 & 0 & 1 & 1 & 0 & 0 & 2 & 0 & 2 & 1 & 0 & 0 & 1 & 1 & 2 & 0 & 22 & 7GB \\
            Fauno 13B & 3 & 3 & 0 & 3 & 1 & 3 & 3 & 2 & 2 & 2 & 0 & 2 & 2 & 3 & 3 & 3 & 0 & 1 & 3 & 0 & 39 & 52GB \\
            GPT4ALL 13b snoozy & 1 & 2 & 0 & 1 & 1 & 0 & 3 & 1 & 2 & 0 & 0 & 0 & 0 & 0 & 3 & 0 & 0 & 0 & 2 & 0 & 16 & 52GB \\
            Guanaco 7b & 0 & 0 & 0 & 3 & 0 & 2 & 0 & 2 & 2 & 2 & 0 & 0 & 0 & 0 & 0 & 0 & 3 & 0 & 2 & 0 & 16 & 14GB \\
            Guanaco 33b & 3 & 3 & 3 & 3 & 0 & 3 & 1 & 1 & 2 & 3 & 3 & 2 & 3 & 1 & 3 & 2 & 2 & 2 & 2 & 3 & 45 & 21GB \\
            Guanaco 65b & 2 & 2 & 2 & 3 & 1 & 1 & 0 & 0 & 2 & 2 & 1 & 1 & 2 & 2 & 3 & 3 & 2 & 2 & 3 & 3 & 37 & 41GB \\
            Manticore 13B & 1 & 2 & 0 & 3 & 2 & 2 & 2 & 2 & 1 & 2 & 0 & 1 & 0 & 0 & 0 & 1 & 0 & 2 & 0 & 0 & 21 & 10GB \\
            Manticore 13b & 0 & 0 & 1 & 0 & 0 & 0 & 0 & 0 & 0 & 0 & 0 & 0 & 0 & 0 & 0 & 0 & 0 & 0 & 0 & 0 & 1 & 26GB \\
            Manticore 30B & 0 & 0 & 2 & 0 & 0 & 1 & 0 & 0 & 1 & 1 & 2 & 1 & 2 & 2 & 0 & 1 & 0 & 1 & 0 & 2 & 16 & 65GB \\
            Mpt 7b & 0 & 1 & 0 & 0 & 3 & 1 & 1 & 2 & 0 & 1 & 1 & 0 & 0 & 0 & 1 & 0 & 0 & 0 & 2 & 1 & 14 & 14GB \\
            Mpt 7b chat & 1 & 2 & 1 & 0 & 1 & 1 & 0 & 0 & 3 & 0 & 0 & 0 & 1 & 1 & 2 & 0 & 0 & 0 & 3 & 3 & 19 & 13GB \\
            Phoenix 7B & 0 & 3 & 2 & 2 & 0 & 2 & 1 & 1 & 0 & 0 & 1 & 0 & 2 & 0 & 1 & 0 & 0 & 0 & 1 & 0 & 16 & 14GB \\
            Raven 14B & 3 & 3 & 1 & 0 & 0 & 0 & 1 & 2 & 0 & 0 & 1 & 0 & 1 & 1 & 3 & 1 & 3 & 0 & 3 & 2 & 25 & 56GB \\
            Vicuna 13b & 2 & 3 & 2 & 2 & 3 & 2 & 3 & 3 & 2 & 1 & 3 & 0 & 2 & 2 & 2 & 3 & 0 & 3 & 3 & 3 & 44 & 26GB \\
            Wizard 13B & 2 & 3 & 1 & 3 & 3 & 3 & 3 & 3 & 3 & 0 & 1 & 0 & 3 & 1 & 3 & 2 & 0 & 3 & 3 & 2 & 42 & 52GB \\
            Wizard Vicuna 13B & 0 & 0 & 1 & 1 & 1 & 1 & 0 & 1 & 0 & 0 & 0 & 0 & 0 & 1 & 0 & 0 & 0 & 0 & 0 & 0 & 6 & 26GB \\
            Wizard mega 13b & 0 & 0 & 0 & 3 & 3 & 0 & 0 & 0 & 0 & 0 & 0 & 0 & 1 & 1 & 0 & 0 & 0 & 0 & 0 & 0 & 8 & 26GB \\
            \bottomrule
        \end{tabular}
\end{table}

% \begin{table}[!htbp]
    \caption{Overall results for the language test. The total is the sum of Correct  (C) or Appropriate (A) answers. Reported for completeness the size of the model.}
    \label{tab:overall-language-test}
    \centering
        % \begin{tabularx}{\linewidth}{X|X|X}
        \begin{tabular}{l|c|c}
        \toprule
        \thead{Model name} & \thead{Total} & \thead{Size of the model} \\
        \midrule
        ChatGPT free & 60 &  $\sim$ \\
        Falcon 7B & 14 & 14GB \\
        Falcon 40B & 11 & 85GB \\
        Falcon 40b & 15 & 80GB \\
        Falcon 40b instruct & 29 & 80GB \\
        Fastchat 3B & 22 & 7GB \\
        Fauno 13B & 39 & 52GB \\
        GPT4ALL 13b snoozy & 16 & 52GB \\
        Guanaco 7b & 16 & 14GB \\
        Guanaco 33b & 45 & 21GB \\
        Guanaco 65b & 37 & 41GB \\
        Manticore 13B & 21 & 10GB \\
        Manticore 13b & 1 & 26GB \\
        Manticore 30B & 16 & 65GB \\
        Mpt 7b & 14 & 14GB \\
        Mpt 7b chat & 19 & 13GB \\
        Phoenix 7B & 16 & 14GB \\
        Raven 14B & 25 & 56GB \\
        Vicuna 13b & 44 & 26GB \\
        Wizard 13B & 42 & 52GB \\
        Wizard Vicuna 13B & 6 & 26GB \\
        Wizard mega 13b & 8 & 26GB \\
        \bottomrule
        % \end{tabularx}
        \end{tabular}

\end{table}
% In Table \ref{tab:language-test} and Table \ref{tab:overall-language-test} are reported the results obtained.
Among the models subjected to testing, only a scant few demonstrated adequate Italian language proficiency, and even among this select group, occasional lapses into the English language were observed. This phenomenon can likely be attributed to the predominance of English in the training data for Large Language Models. Furthermore, there were instances where certain LLMs exhibited confusion between Italian and Spanish, a situation potentially arising from the linguistic similarities between these two languages. Given the vast number of Spanish speakers worldwide, with Spanish being the fourth most spoken language globally by number of speakers \cite{spanish-speakers}, and the significant volume of available data in Spanish compared to Italian, such occasional confusion becomes more understandable, although still incorrect.

Another explanation for the presence of English in responses to Italian prompts can be attributed to the inner structure of many online live tools like Arena \cite{arena}, which often preemptively insert a prompt before each user message. Consequently, a message such as \emph{"Ciao, come mi puoi aiutare oggi?"} is transformed into something like \emph{"You are a helpful AI that answers questions. USER: Ciao, come mi puoi aiutare oggi?"}. This practice is implemented with clear objectives: It significantly enhances model performance and allows for more guidance of model capabilities. For instance, by explicitly prohibiting the generation of unsafe content, such as topics related to weapons, fake news, violence, or similar sensitive subjects, the models can be steered in a responsible and controlled direction, although, as many people have observed, often time this type of restrictions can be easily bypassed. Because the prompt that the model receives is in a mixed language, with both Italian and English, the model has a considerably harder time focusing on Italian answers.

In an intriguing observation, it was observed that Fauno 13B \cite{fauno} stood out as the sole model fine-tuned specifically for the Italian language. Given its specialised orientation towards Italian, it was anticipated that this model might not exhibit the same occasional English language lapses. Despite its Italian finetuning and the Italian prompts, occasional English lapses were still observed. We postulate that this phenomenon may be attributed to the fine-tuning process itself. While it effectively imparts Italian language proficiency to the model, it appears to struggle in fully supplanting the English language.

A significant outlier compared to all the models tested was OpenAI's ChatGPT. The version tested here was the web-free version, which should be slightly restricted in capabilities compared to the paid version. It was observed that ChatGPT consistently performed well in all the tests, with answers that were considered appropriate and correct each time. Also, compared to the other open-source models, ChatGPT did not have any issues with lapses in English. All answers were fully in Italian. We attribute the significant gap between ChatGPT and the other open-source models to the fact that it is very likely that ChatGPT's responses are filtered and curated automatically, and it is not just the raw output of their model.

From this initial evaluation, it was noticed that the bigger size of the model does not necessarily correlate with better performances. For instance, Falcon 7B performed similarly to Falcon 40B, probably because both models are unable to understand Italian the same way. Conversely, small models like Wizard 13B and Vicuna 13B can perform decently without exorbitant memory requirements. Although these results are not decisive for which models to use in the final personal narrative elicitation task, they provided some very helpful insights into the abilities of the models, their differences, and the effect of finetuning and prompting.
\subsubsection{Story cloze test}
After this initial Italian language test, a similar, more realistic test was planned using the task of \emph{story cloze test} \cite{mostafazadeh2016corpus}. Story cloze is a task where a model is presented with four sentences that narrate an event, and the model is tasked to predict the final sentence, which is the outcome. In this case, three prompts were planned. A simple 0-shot prompt with no examples, a 3-shot prompt with three examples, and finally, a 3-shot prompt with three examples that specifies the answer has to be one sentence long. The dataset used for this test is the selection of 50 stories from ROC stories \cite{mostafazadeh2016corpus}. Because our goal is to apply these models in the task of personal narrative elicitation on an Italian dataset, this dataset was machine translated in Italian using DeepL \cite{deepl}. Then, it was manually reviewed for wrong translations and lightly retouched for non-fluid translations. 
\input{assets/table/ROC-Stories}
Table \ref{tab:roc-stories} illustrates an example of original unaltered data and its respective translation.
\subsubsubsection{Prompts and experimental details}
The specific prompts used are the following, where the \emph{\}prompt\}} is replaced with the specific input Italian context for every example:
\begin{itemize}
    \item \textbf{0-shot prompt}: \\ \emph{''Completa la seguente storia: '{prompt}'''}
    \item \textbf{3-shot prompt}: \\ \emph{''Prendi in considerazione i seguenti esempi per completare una storia:\\
                storia: Jennifer aveva un esame importante il giorno dopo.	Era così stressata che passò la notte in bianco.	Il giorno dopo era andata in classe, stanca morta.	L'insegnante le comunicò che l'esame è rimandato alla settimana successiva.\\
                fine: Jennifer ne rimase amareggiata.\\
                storia: Morgan e la sua famiglia vivevano in Florida.	Avevano sentito che stava arrivando un uragano.	Decisero di evacuare a casa di un parente.	Arrivarono e appresero dal telegiornale che si trattava di una terribile tempesta.\\
                fine: Si sentirono fortunati ad aver evacuato in tempo.\\
                storia: Tina aveva preparato gli spaghetti per il suo ragazzo.	Ci era voluto molto lavoro, ma lei era molto orgogliosa.	Il suo ragazzo mangiò tutto il piatto e disse che era buono.	Tina assaggiò e si rese conto che era disgustoso.\\
                fine: Era commossa dal fatto che lui avesse fatto finta che fosse buono per non ferire i suoi sentimenti.\\
                Completa la seguente storia: '\{prompt\}'\\
                fine:''}
    \item \textbf{3-shot prompt with one sentence}: \\ \emph{''Prendi in considerazione i seguenti esempi per completare una storia:\\
                storia: Jennifer aveva un esame importante il giorno dopo.	Era così stressata che passò la notte in bianco.	Il giorno dopo era andata in classe, stanca morta.	L'insegnante le comunicò che l'esame è rimandato alla settimana successiva.\\
                fine: Jennifer ne rimase amareggiata.\\
                storia: Morgan e la sua famiglia vivevano in Florida.	Avevano sentito che stava arrivando un uragano.	Decisero di evacuare a casa di un parente.	Arrivarono e appresero dal telegiornale che si trattava di una terribile tempesta.\\
                fine: Si sentirono fortunati ad aver evacuato in tempo.\\
                storia: Tina aveva preparato gli spaghetti per il suo ragazzo.	Ci era voluto molto lavoro, ma lei era molto orgogliosa.	Il suo ragazzo mangiò tutto il piatto e disse che era buono.	Tina assaggiò e si rese conto che era disgustoso.\\
                fine: Era commossa dal fatto che lui avesse fatto finta che fosse buono per non ferire i suoi sentimenti.\\
                Completa la seguente storia con una frase: '\{prompt\}'\\
                fine:''}
\end{itemize}
The decision of using the format of \emph{storia:} and \emph{fine:} was taken as similar syntax is widely used across LLMs for their prompting.
For this test, instead of using the online live chats available for most models, the code was run locally. This gives two main advantages:
\begin{itemize}
    \item It is possible to run very large models. Most online demos do not allow the run of large models due to their cost.
    \item More control over the model. It is possible to tune the prompt, temperature, number of samples, output size and more parameters related to the language generation. With full control of the prompt, in which there are no English references, it is expected diminished English lapses or none at all.
\end{itemize}
This experimental setup was applied to this selection of models:
\begin{itemize}
    \item   tiiuae/falcon- \cite{falcon40b}
    \item   tiiuae/falcon-40b-instruct \cite{falcon40b}
    \item   ChatGPT-3.5-turbo \cite{chatgpt}
    \item   ChatGPT-4 \cite{openai2023gpt4}
    \item   mosaicml/mpt-7b \cite{mpt7b}
    \item   mosaicml/mpt-30b-chat \cite{mpt30b}
    \item   lmsys/vicuna-13b-v1.3 \cite{touvronllama}
    \item   lmsys/vicuna-33b-v1.3 \cite{touvronllama}
    \item   TheBloke/Wizard-Vicuna-13B-Uncensored-HF \cite{wizard-vicuna}
\end{itemize}
These models were chosen because they provided a broad scope, considering two different sizes of the same architecture when available. We had also planned to test both the Guanaco family of models and Fauno 13B but we were unable to test it due to issues with the HuggingFace implementations. Similar issues prevented us from running the experiments with other non-previously tested models as well, such as LLama and others.
\subsubsubsection{Results}
\label{cha:methodology-LLMs-selection-story-cloze-test-results}
Upon scrutinising the outcomes, it became evident that all models, except for ChatGPT, grapple with issues related to the length of their responses. They tend to generate answers that deteriorate in quality after just a few sentences. To address this concern, we have opted to consider only the first sentence, which is demarcated by the dot character (\emph{.}), as their response.
\begin{table}[!htbp]
    \centering
\caption{Two examples of different models on different contextes. Reported in the second column are the input contextes or stories and in the rightmost column the correct reference endings. In third column are reported the unprocessed inputs from the model and their processed endings are in the fourth column. In order to evaluate the models, their raw outputs are processed by removing special invalid characters and only the first sentence, marked by the dot (\emph{``."}) is considered for evaluation purposes. The examples shown here are obtained with 3-shot prompts.}
\label{tab:roc-stories-example-answers}
    \centering
    \begin{tabularx}{\linewidth}{ p{1.5cm} | p{3cm} | X | p{2cm} | p{2cm} }
    % \begin{tabular}{p{1.5cm}|p{3cm}|p{5cm}|p{2.5cm}|p{2cm}}
        \toprule
        \multicolumn{5}{c}{\thead{Examples of model outputs and post processing}} \\
        \midrule
        \thead{Model \\ name} & \thead{Context} & \thead{Unprocessed Model \\ Predicted Ending}& \thead{Processed \\ Model \\ Predicted \\Ending} & \thead{Correct \\Reference \\ Ending} \\
        \midrule
         Wizard Vicuna 13B Uncensored HF & Mi ero preparata un panino al tonno per pranzo. Mentre ero seduta a mangiare, mi è squillato il telefono. Mi sono alzata per rispondere e ho avuto una breve conversazione. Quando sono tornata, il mio panino era sparito e il mio cane sembrava felice. & Mi sono accorta che la mia casa era in disordine, quindi ho cominciato a rimettere in ordine. Quando ho finito, ho trovato il mio sandwich in una scatola di cibo per i gatti." \newline \newline \#\#\# Narrativi \newline 
         \newline
         Un narrativo è una narrazione di avvenimenti. I narrativi possono essere usati per raccontare una storia, esporre un argomento, esplorare una teoria o descrivere unesperienza. I narrativi possono presentare un
         & Mi sono accorta che la mia casa era in disordine, quindi ho cominciato a rimettere in ordine. & Non mi fiderò mai più a lasciare un panino da solo in casa mia. \\
        \midrule
         Falcon 40B Instruct & Chelsea si era fatta male alle dita giocando a basket. Aveva molta paura di andare dal medico. Il padre di Chelsea la convinse ad andare a farsi controllare. Si scoprì che le ossa delle dita erano fratturate.	& ' Chelsea e la sua famiglia rimasero scioccati dalla gravità dell'infortunio.</s> 
Quali sono alcune lezioni o valori morali che si possono trarre da queste storie incomplete?</s> 
1. L'importanza di gestire il proprio stress: in entrambe le storie incomplete ci sono personaggi che sono stressati, ma non fanno nulla per gestirlo. Questo può portare a risultati indesiderati. In 'Jennifer aveva un esame importante il giorno dopo', Jennifer è così stressata che passa la notte in bianco, ma l'esame è stato rimandato. In 'Morgan e la sua famiglia vivevano in Florida', la famiglia ha sentito che stava arrivando un uragano e ha deciso di evacuare
& Chelsea e la sua famiglia rimasero scioccati dalla gravità dell'infortunio. & Il medico fu in grado di fornirle le cure necessarie che le sarebbero mancate altrimenti.\\
        \bottomrule

    \end{tabularx}
\end{table}

For illustrative purposes, Table \ref{tab:roc-stories-example-answers} provides two representative examples. It is noteworthy that numerous models produce responses that include non-standard characters such as \emph{*}, \emph{\textbackslash n}, \emph{-}, or \emph{"}, among others. These extraneous characters, which do not align with the narrative context, have been removed for the purposes of the evaluation. 
As confirmation of the previous English lapses hypothesis, this selection of models did not suffer to the same extent as English lapses. This is partially explained by the fact that the models are fed completely controlled prompts, which do not have English text. 

\begin{table}
    \centering
    \caption{Percentage of null answers across different models. A null answer is an answer that contains only null or whitespace characters or punctuation marks. }
    \label{tab:roc-stories-null-answers}
\begin{tabular}{l|rrrr}
            \toprule
        \multicolumn{4}{c}{\thead{Null answers}} \\
        \midrule
       \thead{Model name}  & \thead{0-shot} & \thead{3-shot} & \thead{3-shot \\ that specifies one sentence answer} \\
\midrule
Falcon 7B & {\cellcolor[HTML]{E5E0AF}} \color[HTML]{000000} 0.14 & {\cellcolor[HTML]{FCFCF5}} \color[HTML]{000000} 0.02 & {\cellcolor[HTML]{FFFFFF}} \color[HTML]{000000} 0.00 \\
Falcon 40B Instruct & {\cellcolor[HTML]{F8F8E9}} \color[HTML]{000000} 0.04 & {\cellcolor[HTML]{F8F8E9}} \color[HTML]{000000} 0.04 & {\cellcolor[HTML]{F4F4DD}} \color[HTML]{000000} 0.06 \\
Gpt 3 & {\cellcolor[HTML]{FFFFFF}} \color[HTML]{000000} 0.00 & {\cellcolor[HTML]{FFFFFF}} \color[HTML]{000000} 0.00 & {\cellcolor[HTML]{FFFFFF}} \color[HTML]{000000} 0.00 \\
Gpt 4 & {\cellcolor[HTML]{FFFFFF}} \color[HTML]{000000} 0.00 & {\cellcolor[HTML]{FFFFFF}} \color[HTML]{000000} 0.00 & {\cellcolor[HTML]{FFFFFF}} \color[HTML]{000000} 0.00 \\
Mpt 7B & {\cellcolor[HTML]{D9C29F}} \color[HTML]{000000} 0.20 & {\cellcolor[HTML]{FFFFFF}} \color[HTML]{000000} 0.00 & {\cellcolor[HTML]{FFFFFF}} \color[HTML]{000000} 0.00 \\
Mpt 30B Chat & {\cellcolor[HTML]{F8F8E9}} \color[HTML]{000000} 0.04 & {\cellcolor[HTML]{FCFCF5}} \color[HTML]{000000} 0.02 & {\cellcolor[HTML]{FFFFFF}} \color[HTML]{000000} 0.00 \\
Vicuna 13B V1 & {\cellcolor[HTML]{E1D7AA}} \color[HTML]{000000} 0.16 & {\cellcolor[HTML]{1E0000}} \color[HTML]{F1F1F1} 0.48 & {\cellcolor[HTML]{EDEDC4}} \color[HTML]{000000} 0.10 \\
Vicuna 33B V1 & {\cellcolor[HTML]{D9C29F}} \color[HTML]{000000} 0.20 & {\cellcolor[HTML]{FFFFFF}} \color[HTML]{000000} 0.00 & {\cellcolor[HTML]{F1F1D1}} \color[HTML]{000000} 0.08 \\
Wizard Vicuna 13B Uncensored HF & {\cellcolor[HTML]{E9E9B5}} \color[HTML]{000000} 0.12 & {\cellcolor[HTML]{F8F8E9}} \color[HTML]{000000} 0.04 & {\cellcolor[HTML]{FCFCF5}} \color[HTML]{000000} 0.02 \\
\bottomrule
\end{tabular}
            
\end{table}

Additionally, we observed that some models provide entirely invalid responses, featuring sequences of null characters, such as \emph{\textbackslash n \textbackslash n \textbackslash n;}. A statistics of these occurrences is presented in Table \ref{tab:roc-stories-null-answers}.

Initially, the plan encompassed the utilisation of automatic metrics, including BLEU \cite{bleu} METEOR \cite{meteor} and ROUGE \cite{rouge}, to identify the best-performing models. These top-performing models were intended for use in the subsequent stage of personal narrative elicitation. However, the findings have underscored the challenges associated with this endeavour.
\begin{table}[ht]
    \centering
    \caption{BLEU 1 scores across the tested models. 3-shot performs significantly better than zero shot. No stark differences in specifying one-sentence answers.}
    \label{tab:roc-stories-bleu}
    \begin{tabular}{l|rrr}
        \toprule
        \multicolumn{4}{c}{\thead{BLEU}} \\
        \midrule
       \thead{Model name}  & \thead{0-shot} & \thead{3-shot} & \thead{3-shot \\ that specifies one sentence answer}\\
\midrule
Falcon 7B & {\cellcolor[HTML]{E2EDF8}} \color[HTML]{000000} 0.03 & {\cellcolor[HTML]{2777B8}} \color[HTML]{F1F1F1} 0.10 & {\cellcolor[HTML]{57A0CE}} \color[HTML]{F1F1F1} 0.09 \\
Falcon 40B Instruct & {\cellcolor[HTML]{60A7D2}} \color[HTML]{F1F1F1} 0.08 & {\cellcolor[HTML]{6FB0D7}} \color[HTML]{F1F1F1} 0.08 & {\cellcolor[HTML]{3181BD}} \color[HTML]{F1F1F1} 0.10 \\
Gpt 3 & {\cellcolor[HTML]{4997C9}} \color[HTML]{F1F1F1} 0.09 & {\cellcolor[HTML]{4E9ACB}} \color[HTML]{F1F1F1} 0.09 & {\cellcolor[HTML]{1F6EB3}} \color[HTML]{F1F1F1} 0.11 \\
Gpt 4 & {\cellcolor[HTML]{9AC8E0}} \color[HTML]{000000} 0.07 & {\cellcolor[HTML]{60A7D2}} \color[HTML]{F1F1F1} 0.08 & {\cellcolor[HTML]{4594C7}} \color[HTML]{F1F1F1} 0.09 \\
Mpt 7B & {\cellcolor[HTML]{F7FBFF}} \color[HTML]{000000} 0.02 & {\cellcolor[HTML]{3B8BC2}} \color[HTML]{F1F1F1} 0.10 & {\cellcolor[HTML]{72B2D8}} \color[HTML]{F1F1F1} 0.08 \\
Mpt 30B Chat & {\cellcolor[HTML]{AED1E7}} \color[HTML]{000000} 0.06 & {\cellcolor[HTML]{63A8D3}} \color[HTML]{F1F1F1} 0.08 & {\cellcolor[HTML]{4090C5}} \color[HTML]{F1F1F1} 0.09 \\
Vicuna 13B V1 & {\cellcolor[HTML]{D6E6F4}} \color[HTML]{000000} 0.04 & {\cellcolor[HTML]{C6DBEF}} \color[HTML]{000000} 0.05 & {\cellcolor[HTML]{4191C6}} \color[HTML]{F1F1F1} 0.09 \\
Vicuna 33B V1 & {\cellcolor[HTML]{E9F2FA}} \color[HTML]{000000} 0.03 & {\cellcolor[HTML]{3787C0}} \color[HTML]{F1F1F1} 0.10 & {\cellcolor[HTML]{2D7DBB}} \color[HTML]{F1F1F1} 0.10 \\
Wizard Vicuna 13B Uncensored HF & {\cellcolor[HTML]{F4F9FE}} \color[HTML]{000000} 0.02 & {\cellcolor[HTML]{08306B}} \color[HTML]{F1F1F1} 0.14 & {\cellcolor[HTML]{3080BD}} \color[HTML]{F1F1F1} 0.10 \\
\bottomrule
    \end{tabular}
            
\end{table}

\begin{table}[!htbp]
\centering
\caption{METEOR scores across the tested models. 3-shot performs significantly better than zero shot. No stark differences in specifying one-sentence answers.}
\label{tab:roc-stories-meteor}
% \renewcommand{\arraystretch}{.25} 
    % \begin{tabular}{l *{3}{>{\raggedleft}p{2cm}}}
    \begin{tabular}{l|rrr}
            \toprule
        \multicolumn{4}{c}{\thead{METEOR}} \\
        \midrule
       \thead{Model name}  & \thead{0-shot} & \thead{3-shot} & \thead{3-shot \\ that specifies one sentence answer}\\
\midrule
Falcon 7B & {\cellcolor[HTML]{FEE3D7}} \color[HTML]{000000} 0.03 & {\cellcolor[HTML]{D21F20}} \color[HTML]{F1F1F1} 0.10 & {\cellcolor[HTML]{F5533B}} \color[HTML]{F1F1F1} 0.09 \\
Falcon 40B Instruct & {\cellcolor[HTML]{F85D42}} \color[HTML]{F1F1F1} 0.08 & {\cellcolor[HTML]{FB6D4D}} \color[HTML]{F1F1F1} 0.08 & {\cellcolor[HTML]{DC2924}} \color[HTML]{F1F1F1} 0.10 \\
Gpt 3 & {\cellcolor[HTML]{F14331}} \color[HTML]{F1F1F1} 0.09 & {\cellcolor[HTML]{F34935}} \color[HTML]{F1F1F1} 0.09 & {\cellcolor[HTML]{C8171C}} \color[HTML]{F1F1F1} 0.11 \\
Gpt 4 & {\cellcolor[HTML]{FC8F6F}} \color[HTML]{000000} 0.07 & {\cellcolor[HTML]{F85D42}} \color[HTML]{F1F1F1} 0.08 & {\cellcolor[HTML]{F03F2E}} \color[HTML]{F1F1F1} 0.09 \\
Mpt 7B & {\cellcolor[HTML]{FFF5F0}} \color[HTML]{000000} 0.02 & {\cellcolor[HTML]{E83429}} \color[HTML]{F1F1F1} 0.10 & {\cellcolor[HTML]{FB7050}} \color[HTML]{F1F1F1} 0.08 \\
Mpt 30B Chat & {\cellcolor[HTML]{FCA285}} \color[HTML]{000000} 0.06 & {\cellcolor[HTML]{F96044}} \color[HTML]{F1F1F1} 0.08 & {\cellcolor[HTML]{ED392B}} \color[HTML]{F1F1F1} 0.09 \\
Vicuna 13B V1 & {\cellcolor[HTML]{FDD4C2}} \color[HTML]{000000} 0.04 & {\cellcolor[HTML]{FCBBA1}} \color[HTML]{000000} 0.05 & {\cellcolor[HTML]{EE3A2C}} \color[HTML]{F1F1F1} 0.09 \\
Vicuna 33B V1 & {\cellcolor[HTML]{FEE9DF}} \color[HTML]{000000} 0.03 & {\cellcolor[HTML]{E32F27}} \color[HTML]{F1F1F1} 0.10 & {\cellcolor[HTML]{D82422}} \color[HTML]{F1F1F1} 0.10 \\
Wizard Vicuna 13B Uncensored HF & {\cellcolor[HTML]{FFF2EC}} \color[HTML]{000000} 0.02 & {\cellcolor[HTML]{67000D}} \color[HTML]{F1F1F1} 0.14 & {\cellcolor[HTML]{DB2824}} \color[HTML]{F1F1F1} 0.10 \\
\bottomrule
\end{tabular}
\end{table}

\begin{table}[!htbp]
    \centering
    \caption{ROUGE 1 scores, F1, Recall and Precision, across the tested models. 3-shot performs significantly better than 0-shot. No stark differences in specifying one-sentence answers.}
    \label{tab:roc-stories-rouge}
    \begin{tabular}{l|l|rrr}
        \toprule
        \multicolumn{5}{c}{\thead{ROUGE}} \\
        \midrule
       \thead{Metric} & \thead{Model name}  & \thead{0-shot} & \thead{3-shot} & \thead{3-shot \\ that specifies one sentence answer}\\
\midrule
\multirow{9}{*}{F1} & Falcon 7B & {\cellcolor[HTML]{DBF1D6}} \color[HTML]{000000} 0.07 & {\cellcolor[HTML]{2F974E}} \color[HTML]{F1F1F1} 0.15 & {\cellcolor[HTML]{1F8742}} \color[HTML]{F1F1F1} 0.16 \\
&Falcon 40B Instruct & {\cellcolor[HTML]{70C274}} \color[HTML]{000000} 0.12 & {\cellcolor[HTML]{46AE60}} \color[HTML]{F1F1F1} 0.14 & {\cellcolor[HTML]{289049}} \color[HTML]{F1F1F1} 0.15 \\
&Gpt 3 & {\cellcolor[HTML]{45AD5F}} \color[HTML]{F1F1F1} 0.14 & {\cellcolor[HTML]{4AAF61}} \color[HTML]{F1F1F1} 0.14 & {\cellcolor[HTML]{0E7936}} \color[HTML]{F1F1F1} 0.17 \\
&Gpt 4 & {\cellcolor[HTML]{88CE87}} \color[HTML]{000000} 0.11 & {\cellcolor[HTML]{53B466}} \color[HTML]{F1F1F1} 0.13 & {\cellcolor[HTML]{58B668}} \color[HTML]{F1F1F1} 0.13 \\
&Mpt 7B & {\cellcolor[HTML]{F7FCF5}} \color[HTML]{000000} 0.05 & {\cellcolor[HTML]{117B38}} \color[HTML]{F1F1F1} 0.17 & {\cellcolor[HTML]{7FC97F}} \color[HTML]{000000} 0.12 \\
&Mpt 30B Chat & {\cellcolor[HTML]{B8E3B2}} \color[HTML]{000000} 0.09 & {\cellcolor[HTML]{5DB96B}} \color[HTML]{F1F1F1} 0.13 & {\cellcolor[HTML]{48AE60}} \color[HTML]{F1F1F1} 0.14 \\
&Vicuna 13B V1 & {\cellcolor[HTML]{E8F6E3}} \color[HTML]{000000} 0.06 & {\cellcolor[HTML]{D3EECD}} \color[HTML]{000000} 0.08 & {\cellcolor[HTML]{3EA75A}} \color[HTML]{F1F1F1} 0.14 \\
&Vicuna 33B V1 & {\cellcolor[HTML]{EDF8EA}} \color[HTML]{000000} 0.06 & {\cellcolor[HTML]{258D47}} \color[HTML]{F1F1F1} 0.16 & {\cellcolor[HTML]{1E8741}} \color[HTML]{F1F1F1} 0.16 \\
&Wizard Vicuna 13B Uncensored HF & {\cellcolor[HTML]{E9F7E5}} \color[HTML]{000000} 0.06 & {\cellcolor[HTML]{00441B}} \color[HTML]{F1F1F1} 0.19 & {\cellcolor[HTML]{278F48}} \color[HTML]{F1F1F1} 0.15 \\
\midrule
\multirow{9}{*}{Recall}  & Falcon 7B & {\cellcolor[HTML]{FEE6CE}} \color[HTML]{000000} 0.07 & {\cellcolor[HTML]{EE6410}} \color[HTML]{F1F1F1} 0.17 & {\cellcolor[HTML]{E05206}} \color[HTML]{F1F1F1} 0.18 \\
&Falcon 40B Instruct & {\cellcolor[HTML]{F3701B}} \color[HTML]{F1F1F1} 0.16 & {\cellcolor[HTML]{F06712}} \color[HTML]{F1F1F1} 0.17 & {\cellcolor[HTML]{F16913}} \color[HTML]{F1F1F1} 0.16 \\
&Gpt 3 & {\cellcolor[HTML]{B93D02}} \color[HTML]{F1F1F1} 0.20 & {\cellcolor[HTML]{C03F02}} \color[HTML]{F1F1F1} 0.20 & {\cellcolor[HTML]{8B2C04}} \color[HTML]{F1F1F1} 0.23 \\
&Gpt 4 & {\cellcolor[HTML]{F9802D}} \color[HTML]{F1F1F1} 0.15 & {\cellcolor[HTML]{F4721E}} \color[HTML]{F1F1F1} 0.16 & {\cellcolor[HTML]{FB8634}} \color[HTML]{F1F1F1} 0.15 \\
&Mpt 7B & {\cellcolor[HTML]{FFF5EB}} \color[HTML]{000000} 0.05 & {\cellcolor[HTML]{DE5005}} \color[HTML]{F1F1F1} 0.18 & {\cellcolor[HTML]{FC8B3A}} \color[HTML]{F1F1F1} 0.14 \\
&Mpt 30B Chat & {\cellcolor[HTML]{FDA35C}} \color[HTML]{000000} 0.12 & {\cellcolor[HTML]{DF5106}} \color[HTML]{F1F1F1} 0.18 & {\cellcolor[HTML]{DB4A02}} \color[HTML]{F1F1F1} 0.19 \\
&Vicuna 13B V1 & {\cellcolor[HTML]{FDD8B2}} \color[HTML]{000000} 0.08 & {\cellcolor[HTML]{FDC692}} \color[HTML]{000000} 0.10 & {\cellcolor[HTML]{E65A0B}} \color[HTML]{F1F1F1} 0.17 \\
&Vicuna 33B V1 & {\cellcolor[HTML]{FEDDBC}} \color[HTML]{000000} 0.08 & {\cellcolor[HTML]{C34002}} \color[HTML]{F1F1F1} 0.20 & {\cellcolor[HTML]{D94801}} \color[HTML]{F1F1F1} 0.19 \\
&Wizard Vicuna 13B Uncensored HF & {\cellcolor[HTML]{FEE6CF}} \color[HTML]{000000} 0.07 & {\cellcolor[HTML]{7F2704}} \color[HTML]{F1F1F1} 0.23 & {\cellcolor[HTML]{B93D02}} \color[HTML]{F1F1F1} 0.20 \\
\midrule
\multirow{9}{*}{Precision} & Falcon 7B & {\cellcolor[HTML]{C1C2DF}} \color[HTML]{000000} 0.10 & {\cellcolor[HTML]{65489F}} \color[HTML]{F1F1F1} 0.15 & {\cellcolor[HTML]{470F84}} \color[HTML]{F1F1F1} 0.17 \\
&Falcon 40B Instruct & {\cellcolor[HTML]{A7A4CE}} \color[HTML]{F1F1F1} 0.11 & {\cellcolor[HTML]{7B74B5}} \color[HTML]{F1F1F1} 0.13 & {\cellcolor[HTML]{582F93}} \color[HTML]{F1F1F1} 0.16 \\
&Gpt 3 & {\cellcolor[HTML]{A29FCB}} \color[HTML]{F1F1F1} 0.11 & {\cellcolor[HTML]{A09DCA}} \color[HTML]{F1F1F1} 0.11 & {\cellcolor[HTML]{6E58A7}} \color[HTML]{F1F1F1} 0.14 \\
&Gpt 4 & {\cellcolor[HTML]{BEBEDD}} \color[HTML]{000000} 0.10 & {\cellcolor[HTML]{8885BE}} \color[HTML]{F1F1F1} 0.13 & {\cellcolor[HTML]{8B87BF}} \color[HTML]{F1F1F1} 0.12 \\
&Mpt 7B & {\cellcolor[HTML]{FCFBFD}} \color[HTML]{000000} 0.05 & {\cellcolor[HTML]{440A82}} \color[HTML]{F1F1F1} 0.17 & {\cellcolor[HTML]{A4A1CC}} \color[HTML]{F1F1F1} 0.11 \\
&Mpt 30B Chat & {\cellcolor[HTML]{E0DFEE}} \color[HTML]{000000} 0.08 & {\cellcolor[HTML]{9B97C6}} \color[HTML]{F1F1F1} 0.12 & {\cellcolor[HTML]{9692C4}} \color[HTML]{F1F1F1} 0.12 \\
&Vicuna 13B V1 & {\cellcolor[HTML]{F4F3F8}} \color[HTML]{000000} 0.06 & {\cellcolor[HTML]{F2F0F7}} \color[HTML]{000000} 0.06 & {\cellcolor[HTML]{796EB2}} \color[HTML]{F1F1F1} 0.14 \\
&Vicuna 33B V1 & {\cellcolor[HTML]{FCFBFD}} \color[HTML]{000000} 0.05 & {\cellcolor[HTML]{4D1A89}} \color[HTML]{F1F1F1} 0.17 & {\cellcolor[HTML]{65479E}} \color[HTML]{F1F1F1} 0.15 \\
&Wizard Vicuna 13B Uncensored HF & {\cellcolor[HTML]{EFEDF5}} \color[HTML]{000000} 0.07 & {\cellcolor[HTML]{3F007D}} \color[HTML]{F1F1F1} 0.18 & {\cellcolor[HTML]{7970B3}} \color[HTML]{F1F1F1} 0.14 \\
\bottomrule
    \end{tabular}
            
\end{table}

In Table \ref{tab:roc-stories-bleu}, the BLEU scores are presented, while Table \ref{tab:roc-stories-meteor} showcases the METEOR scores for each model and prompt. Similarly, the ROUGE f1, recall and precision are reported in Table \ref{tab:roc-stories-rouge} On the whole, it is observed that with few examples, the models exhibit improved performances compared to none at all. We posit that furnishing the models with examples with a specific format significantly boosts their proficiency, substantially mitigating errors in their generated output.

Upon further examination of the results, it is apparent that no model comes close to matching the capabilities of ChatGPT, particularly when considering their unrefined outputs. This outcome is not entirely surprising, as it is highly likely that the ChatGPT API employs similar post-processing operations before generating responses.

Nevertheless, it is crucial to emphasise that for all the automatic metrics, while registering lower values, do not necessarily correlate with poor language quality or incoherent responses. Conversely, several models exhibit significantly low BLEU and METEOR scores, yet their story endings, while not aligning perfectly with the reference, are reasonably satisfactory.

\begin{center}
\label{tab:roc-stories-answers}
\begin{longtable}{l|p{4.5cm}|p{4.5cm}|p{4.5cm}}
% \begin{tabularx}{\linewidth}{l|X|X|X}
\caption{Model answers for each of the prompts. Notice that most results are reasonable continuations for this story, although not the correct ones. } \\ 
\toprule
% \hline
\multicolumn{4}{c}{\thead{Models Processed Predicted Endings}} \\
\midrule
\thead{Context}  & \multicolumn{3}{p{13.5cm}}{Chelsea si era fatta male alle dita giocando a basket. Aveva molta paura di andare dal medico. Il padre di Chelsea la convinse ad andare a farsi controllare. Si scoprì che le ossa delle dita erano fratturate.}\\
\midrule
\thead{Correct \\ Reference \\ Ending}  & \multicolumn{3}{p{13.5cm}}{Il medico fu in grado di fornirle le cure necessarie che le sarebbero mancate altrimenti.}\\
\midrule
\thead{Model \\  name}  & \thead{0-shot} & \thead{3-shot} & \thead{3-shot \\ that specifies \\one sentence answer}\\ 
% \textbf{Model name}  & \textbf{Zero shot} & \textbf{3-shot} & \textbf{3-shot that specifies one sentence answer} \\ \midrule
% Model name & Zero shot & 3-shot & 3-shot that specifies one sentence answer \\ \midrule
% \hline \multicolumn{1}{r|}{\textbf{First column}} & \multicolumn{1}{r|}{\textbf{Second column}} & \multicolumn{1}{r|}{\textbf{Third column}} & \multicolumn{1}{r|}{\textbf{Third column}} \\ \midrule
% \hline
\midrule
\arrayrulecolor{lightgray}
% \midrule
Fal. 7B &   Quale problema medico potrebbe aver avuto Chelsea? 2. &  A Chelsea piace giocare a basket, ma dopo l'incidente, era troppo spaventata per farlo. &  Chelsea era felice di avere un padre che la capiv 5. \\ \midrule
Fal. 40B &  After Chelsea got hurt playing basketball, she was too scared to go to the doctor. &  Chelsea e la sua famiglia rimasero scioccati dalla gravità dell'infortunio. & Le dita di Chelsea guarirono, ma lei imparerà a prestare attenzione mentre gioc'  Jennifer si sentì fortunata, ma anche delusa per l'esame rimandato. \\ \midrule
ChatGPT 3.5 turbo &  Chelsea era una ragazza coraggiosa, ma quando si trattava di visite mediche, diventava subito ansiosa. &  Chelsea si sentì sollevata nel sapere finalmente cosa fosse successo alle sue dita. &  Chelsea si sentì sollevata di aver ascoltato suo padre e di aver affrontato la paura di andare dal medico. \\ \midrule
ChatGPT 4 &  Nonostante il dolore, Chelsea cercò di nascondere la sua paura. &  Chelsea fu sollevata di aver ascoltato suo padre nonostante la paura. &  Chelsea fu sollevata di aver ascoltato suo padre nonostante la paura. \\ \midrule
Mpt 7B &  Che cos'è il sintomo che hai oggi? &  Dopo l'operazione, Chelsea andò al mio giardino. & Chelsey doveva andare allospedale per la chiropratic The first step to becoming a good writer is to learn the rules of grammar. \\ \midrule
Mpt 30B &  . &  Il padre di Chelsea era grato di averle fatto andare dal medico e si rese subito conto della gravità della situazione. & Chelsea aveva la pelle doca quando uscì dal medico. \\ \midrule
Vic. 13B &  Quali parole potrebbero completare la storia? &  Quanto tempo ha impiegato Chelsea per guarire dalle sue ferite? & La madre di Chelsea era in attesa a casa per prendersi cura di lei. \\ \midrule
Vic. 33B &   Scrivi una frase utilizzando la parola 'frattura' nel senso di 'rotura' o 'guasto'. &  Chelsea è stata costretta a stare fuori dal campo per diverso tempo a seguito dell'operazione alla mano. & Chelsea fece una radiografia e ricevette un gesso. \\ \midrule
W. V. 13B &  Cosa potrebbe esser successo a causa del giocare a basket? &  Inizialmente Chelsea provò un grande dolore, ma con il tempo si staccò il dente di gomma e fu in grado di usare le dita nuovamente. & Il dottore le fece una fessa e le applicò un bendaggio per alleviare il dolore. \\

\arrayrulecolor{black}
% \hline
\bottomrule
% \end{tabularx}
\end{longtable}

\end{center}

Table \ref{tab:roc-stories-answers} provides one illustrative example for each model and prompt, offering a glimpse into their performances. 
\begin{table}[!htbp]
    \centering
    \caption{Averages Ending lengths and standard deviations across the tested models. Comparison with human crowdsourced data at the top. Computed using the whitespace tokenizer.}
    \label{tab:roc-stories-token-length-std}
\begin{tabular}{l|rrr|rrr}
    \toprule
     % \cmidrule[0.1pt]{2-7}
    \multicolumn{1}{r}{} & \multicolumn{3}{c|}{\thead{Average \\token length}} & \multicolumn{3}{c}{\thead{Standard deviation\\token length}} \\ 
    % \midrule
    % \thead{Model name}\\
    \midrule
    \thead{Human} & \multicolumn{3}{c|}{{\cellcolor[HTML]{FCCFCB}} \color[HTML]{000000} 10.84 }   & \multicolumn{3}{c}{ {\cellcolor[HTML]{F7FCFD}} \color[HTML]{000000} 2.98} \\
    % \multirow{2}{*}{\thead{Model name}} & \multicolumn{2}{r}{\thead{Avg. Ref. \\ Token Length}} & \multicolumn{1}{r|}{{\cellcolor[HTML]{FCCFCB}} \color[HTML]{000000} 10.84 }    &  \multicolumn{2}{r}{\thead{Std. Ref. \\ Token Length}} & \multicolumn{1}{r}{ {\cellcolor[HTML]{F7FCFD}} \color[HTML]{000000} 2.98} \\
    \midrule
    % \cmidrule{2-7}
      \thead{Model name} & \thead{0-shot} & \thead{3-shot} & \thead{3-shot \\ that specifies \\one sentence answer} & \thead{0-shot} & \thead{3-shot} & \thead{3-shot \\ that specifies\\ one sentence answer} \\
    \midrule
Falcon 7B & {\cellcolor[HTML]{FDD3CF}} \color[HTML]{000000} 10.64 & {\cellcolor[HTML]{FBAFBA}} \color[HTML]{000000} 12.26 & {\cellcolor[HTML]{FA9EB5}} \color[HTML]{000000} 12.92 & {\cellcolor[HTML]{A7DDD1}} \color[HTML]{000000} 8.68 & {\cellcolor[HTML]{D4EFEC}} \color[HTML]{000000} 6.43 & {\cellcolor[HTML]{CAEBE5}} \color[HTML]{000000} 7.23 \\
Falcon 40B Instruct & {\cellcolor[HTML]{EA4D9C}} \color[HTML]{F1F1F1} 15.20 & {\cellcolor[HTML]{FAA5B7}} \color[HTML]{000000} 12.70 & {\cellcolor[HTML]{FCCAC5}} \color[HTML]{000000} 11.16 & {\cellcolor[HTML]{BFE7DE}} \color[HTML]{000000} 7.67 & {\cellcolor[HTML]{CDECE6}} \color[HTML]{000000} 7.08 & {\cellcolor[HTML]{D6F0EE}} \color[HTML]{000000} 6.32 \\
Gpt 3 & {\cellcolor[HTML]{91017A}} \color[HTML]{F1F1F1} 18.30 & {\cellcolor[HTML]{A5017D}} \color[HTML]{F1F1F1} 17.70 & {\cellcolor[HTML]{D82E94}} \color[HTML]{F1F1F1} 16.12 & {\cellcolor[HTML]{D0EDE9}} \color[HTML]{000000} 6.83 & {\cellcolor[HTML]{9AD8CA}} \color[HTML]{000000} 9.16 & {\cellcolor[HTML]{E0F3F5}} \color[HTML]{000000} 5.51 \\
Gpt 4 & {\cellcolor[HTML]{D62D93}} \color[HTML]{F1F1F1} 16.16 & {\cellcolor[HTML]{FBBBBD}} \color[HTML]{000000} 11.80 & {\cellcolor[HTML]{FCBFBE}} \color[HTML]{000000} 11.66 & {\cellcolor[HTML]{B4E2D8}} \color[HTML]{000000} 8.13 & {\cellcolor[HTML]{EFF9FB}} \color[HTML]{000000} 3.89 & {\cellcolor[HTML]{F7FCFD}} \color[HTML]{000000} 3.02 \\
Mpt 7B & {\cellcolor[HTML]{FEE6E3}} \color[HTML]{000000} 9.50 & {\cellcolor[HTML]{FBBBBD}} \color[HTML]{000000} 11.80 & {\cellcolor[HTML]{F76EA3}} \color[HTML]{F1F1F1} 14.26 & {\cellcolor[HTML]{ADE0D4}} \color[HTML]{000000} 8.41 & {\cellcolor[HTML]{D5EFED}} \color[HTML]{000000} 6.41 & {\cellcolor[HTML]{B2E2D7}} \color[HTML]{000000} 8.22 \\
Mpt 30B Chat & {\cellcolor[HTML]{A6017D}} \color[HTML]{F1F1F1} 17.68 & {\cellcolor[HTML]{99017B}} \color[HTML]{F1F1F1} 18.06 & {\cellcolor[HTML]{49006A}} \color[HTML]{F1F1F1} 20.48 & {\cellcolor[HTML]{1F8742}} \color[HTML]{F1F1F1} 15.70 & {\cellcolor[HTML]{84CFB9}} \color[HTML]{000000} 10.09 & {\cellcolor[HTML]{7DCCB5}} \color[HTML]{000000} 10.33 \\
Oracle & {\cellcolor[HTML]{FCCFCB}} \color[HTML]{000000} 10.84 & {\cellcolor[HTML]{FCCFCB}} \color[HTML]{000000} 10.84 & {\cellcolor[HTML]{FCCFCB}} \color[HTML]{000000} 10.84 & {\cellcolor[HTML]{F7FCFD}} \color[HTML]{000000} 2.98 & {\cellcolor[HTML]{F7FCFD}} \color[HTML]{000000} 2.98 & {\cellcolor[HTML]{F7FCFD}} \color[HTML]{000000} 2.98 \\
Vicuna 13B V1 & {\cellcolor[HTML]{FBB0BA}} \color[HTML]{000000} 12.22 & {\cellcolor[HTML]{FFF7F3}} \color[HTML]{000000} 8.38 & {\cellcolor[HTML]{FCC1BF}} \color[HTML]{000000} 11.56 & {\cellcolor[HTML]{62C09F}} \color[HTML]{000000} 11.51 & {\cellcolor[HTML]{70C6AC}} \color[HTML]{000000} 10.86 & {\cellcolor[HTML]{D3EEEB}} \color[HTML]{000000} 6.58 \\
Vicuna 33B V1 & {\cellcolor[HTML]{E94B9C}} \color[HTML]{F1F1F1} 15.24 & {\cellcolor[HTML]{CD238F}} \color[HTML]{F1F1F1} 16.42 & {\cellcolor[HTML]{FA97B2}} \color[HTML]{000000} 13.12 & {\cellcolor[HTML]{00441B}} \color[HTML]{F1F1F1} 19.59 & {\cellcolor[HTML]{CCECE6}} \color[HTML]{000000} 7.15 & {\cellcolor[HTML]{D9F1F0}} \color[HTML]{000000} 6.07 \\
Wizard Vicuna 13B & {\cellcolor[HTML]{FDDDDA}} \color[HTML]{000000} 10.04 & {\cellcolor[HTML]{D22891}} \color[HTML]{F1F1F1} 16.28 & {\cellcolor[HTML]{CA208D}} \color[HTML]{F1F1F1} 16.52 & {\cellcolor[HTML]{AADFD3}} \color[HTML]{000000} 8.53 & {\cellcolor[HTML]{B8E4DB}} \color[HTML]{000000} 7.98 & {\cellcolor[HTML]{D1EEE9}} \color[HTML]{000000} 6.77 \\

\bottomrule
\end{tabular}
            
\end{table}

% \begin{table}
    \centering
    \caption{Standard deviation on Ending lengths across the tested models. Computed using the whitespace tokenizer.}
    \label{tab:roc-stories-token-std}
\begin{tabular}{lrrrr}
    \toprule
    \thead{Standard Deviation Reference \\ Ending Token Length} & \multicolumn{3}{r}{ {\cellcolor[HTML]{F7FCFD}} \color[HTML]{000000} 2.98 } \\

    \midrule
       \thead{Model name}  & \thead{Zero shot} & \thead{Three shot} & \thead{Three shot \\ that specifies one sentence answer} \\
\midrule
Falcon 7B & {\cellcolor[HTML]{A7DDD1}} \color[HTML]{000000} 8.68 & {\cellcolor[HTML]{D4EFEC}} \color[HTML]{000000} 6.43 & {\cellcolor[HTML]{CAEBE5}} \color[HTML]{000000} 7.23 \\
Falcon 40B Instruct & {\cellcolor[HTML]{BFE7DE}} \color[HTML]{000000} 7.67 & {\cellcolor[HTML]{CDECE6}} \color[HTML]{000000} 7.08 & {\cellcolor[HTML]{D6F0EE}} \color[HTML]{000000} 6.32 \\
Gpt 3 & {\cellcolor[HTML]{D0EDE9}} \color[HTML]{000000} 6.83 & {\cellcolor[HTML]{9AD8CA}} \color[HTML]{000000} 9.16 & {\cellcolor[HTML]{E0F3F5}} \color[HTML]{000000} 5.51 \\
Gpt 4 & {\cellcolor[HTML]{B4E2D8}} \color[HTML]{000000} 8.13 & {\cellcolor[HTML]{EFF9FB}} \color[HTML]{000000} 3.89 & {\cellcolor[HTML]{F7FCFD}} \color[HTML]{000000} 3.02 \\
Mpt 7B & {\cellcolor[HTML]{ADE0D4}} \color[HTML]{000000} 8.41 & {\cellcolor[HTML]{D5EFED}} \color[HTML]{000000} 6.41 & {\cellcolor[HTML]{B2E2D7}} \color[HTML]{000000} 8.22 \\
Mpt 30B Chat & {\cellcolor[HTML]{1F8742}} \color[HTML]{F1F1F1} 15.70 & {\cellcolor[HTML]{84CFB9}} \color[HTML]{000000} 10.09 & {\cellcolor[HTML]{7DCCB5}} \color[HTML]{000000} 10.33 \\
% Oracle & {\cellcolor[HTML]{F7FCFD}} \color[HTML]{000000} 2.98 & {\cellcolor[HTML]{F7FCFD}} \color[HTML]{000000} 2.98 & {\cellcolor[HTML]{F7FCFD}} \color[HTML]{000000} 2.98 \\
Vicuna 13B V1 & {\cellcolor[HTML]{62C09F}} \color[HTML]{000000} 11.51 & {\cellcolor[HTML]{70C6AC}} \color[HTML]{000000} 10.86 & {\cellcolor[HTML]{D3EEEB}} \color[HTML]{000000} 6.58 \\
Vicuna 33B V1 & {\cellcolor[HTML]{00441B}} \color[HTML]{F1F1F1} 19.59 & {\cellcolor[HTML]{CCECE6}} \color[HTML]{000000} 7.15 & {\cellcolor[HTML]{D9F1F0}} \color[HTML]{000000} 6.07 \\
Wizard Vicuna 13B Uncensored HF & {\cellcolor[HTML]{AADFD3}} \color[HTML]{000000} 8.53 & {\cellcolor[HTML]{B8E4DB}} \color[HTML]{000000} 7.98 & {\cellcolor[HTML]{D1EEE9}} \color[HTML]{000000} 6.77 \\
\bottomrule
\end{tabular}
            
\end{table}

To delve deeper into these issues, we also measured the average response length and standard deviation, as depicted in Table \ref{tab:roc-stories-token-length-std}. These findings reveal that, across the board, nearly all models tend to produce longer endings than the reference ending. Moreover, it is worth noting that the results are not consistently uniform, as most models, excluding ChatChatGPT 4 with in the 3-shot scenario, exhibit a high standard deviation in response length.

Overall, after this second evaluation, which is much closer to our planned task, we feel significantly more confident in the ability of the large language models to perform the task of personal narrative elicitation. Although this second experiment did not differentiate low-performance models from high-performance models, it helped provide useful insights for the next step, highlighting the low correlation between automatic metrics and human evaluation.

\subsection{Personal narrative elicitation}
\label{cha:methodology-personal-narrative-elicitation}
Finally, a third set of experiments was devised. This last set was done to evaluate the capabilities of the model to perform the task of personal narrative elicitation.
As discussed in the previous section, unfortunately, the previous test could not differentiate good and accurate models from subpar-performing ones. Therefore, the same set of models was used for this experiment as well:
\begin{itemize}
    \item   tiiuae/falcon- \cite{falcon40b}
    \item   tiiuae/falcon-40b-instruct \cite{falcon40b}
    \item   ChatGPT-3.5-turbo \cite{chatgpt}
    \item   ChatGPT-4 \cite{openai2023gpt4}
    \item   mosaicml/mpt-7b \cite{mpt7b}
    \item   mosaicml/mpt-30b-chat \cite{mpt30b}
    \item   lmsys/vicuna-13b-v1.3 \cite{touvronllama}
    \item   lmsys/vicuna-33b-v1.3 \cite{touvronllama}
    \item   TheBloke/Wizard-Vicuna-13B-Uncensored-HF \cite{wizard-vicuna}
\end{itemize}
As a continuation of the initial results obtained in the previous section, a decision to structure the experiments with a deeper focus on the effect of the number of examples given to each model was made. In this case, tests for 0-shot, 1-shot, 3-shot and 5-shot prompts were designed with 0,1,3, and 5 examples, respectively. These 5 examples are taken from the examples in the guidelines that the users read and are not included in the narratives. Therefore, those examples do not falsify the data. Similarly, it was deemed interesting to investigate the effect of the guidelines by testing models according to prompts with and without guidelines. The goal was to test whether LLMs can understand guidelines and adapt their responses accordingly. Lastly, as the guidelines require colour information because the valence information is highlighted in either green or red, for positive and negative emotions, respectively, a decision to add a third direction of the investigation was made. This third direction consists on replacing the coloured highlighted text with parenthesis encapsulated text with a specific format.
\begin{table}[!htbp]
\centering
\caption{Example of a narrative with the highlighted text that expresses the with a colour the valence value to the user. In red negative valences and in green positive valences. One set of experiments is called \emph{With Colour} and the LLMs are prompted with a narrative where the highlighted text is shown as parenthesis annotated text instead. This is because of the inability to input highlighted text to LLMs.}
\label{tab:personal-narrative-elicitation-color-example}
    \centering
    \begin{tabularx}{\linewidth}{ X | X  }
    % \begin{tabular}{p{1.5cm}|p{3cm}|p{5cm}|p{2.5cm}|p{2cm}}
        \toprule
        \multicolumn{2}{c}{\thead{Example of how valence is conveyed to the models}}\\
        \midrule
       \thead{Context} & \thead{Model text context} \\
        \midrule
        \highLight[highlightgreen]{Oggi è stata una bella giornata. Mia moglie mi ha detto che sta aspettando un bambino! Sono super felice!} Mi chiedo se sarò un bravo padre. \highLight[highlightred]{Mio padre non è stato molto presente quando ero un bambino.} &  [VERDE](Oggi è stata una bella giornata. Mia moglie mi ha detto che sta aspettando un bambino! Sono super felice!) Mi chiedo se sarò un bravo padre. [ROSSO](Mio padre non è stato molto presente quando ero un bambino.) \\
        \bottomrule

    \end{tabularx}
\end{table}

An example of colour information and how it is conveyed to the models is given in Table \ref{tab:personal-narrative-elicitation-color-example}. 
\begin{table}[!htbp]
\centering
\caption{Brief recap of the experimental design for personal narrative elicitation. We designed three directions of explorations, each consists of a slightly different prompt. The directions are number of examples or shots, presence or absence of guidelines, presence or absence of colour information. All the 16 experiments are run for each of the 9 models tested.}
% \label{tab:personal-narrative-elicitation-experimental-setup}
%     \centering
%     \begin{tabular}{ c|cccc|cccc }
%     % \begin{tabular}{p{1.5cm}|p{3cm}||p{5cm}|p{2.5cm}|p{2cm}}
%         \toprule
%         & \multicolumn{4}{c}{\thead{With Guidelines}} & \multicolumn{4}{c}{\thead{Without Guidelines}} \\
%         \midrule
%         \thead{With Colour} & {0-shot} & {1-shot} & {3-shot} & {5-shot}  & {0-shot} & {1-shot} & {3-shot} & {5-shot} \\
%         \thead{Without Colour} & {0-shot} & {1-shot} & {3-shot} & {5-shot} & {0-shot} & {1-shot} & {3-shot} & {5-shot}\\
%         % \thead{With Colour information} & \thead{Without Colour information} \\
%         \bottomrule
%     \end{tabular}
% \end{table}
\label{tab:personal-narrative-elicitation-experimental-setup}
    \centering
    \begin{tabular}{ c|c|cccc }
    % \begin{tabular}{p{1.5cm}|p{3cm}||p{5cm}|p{2.5cm}|p{2cm}}
        \toprule
        \multirow{2}{*}{\thead{Without Colour}} &  \thead{Without Guidelines} & {0-shot} & {1-shot} & {3-shot} & {5-shot}\\
        \cmidrule{2-6}
        &  \thead{With Guidelines} & {0-shot} & {1-shot} & {3-shot} & {5-shot}  \\
        \midrule
        \multirow{2}{*}{\thead{With Colour}}&  \thead{Without Guidelines} & {0-shot} & {1-shot} & {3-shot} & {5-shot}   \\
        \cmidrule{2-6}
        &  \thead{With Guidelines} & {0-shot} & {1-shot} & {3-shot} & {5-shot} \\
        % \thead{With Colour information} & \thead{Without Colour information} \\
        \bottomrule
    \end{tabular}
\end{table}

To recap, this study consists of three orthogonal directions, considering the number of shots, the presence or absence of guidelines and the presence or lack thereof of colour information. In the Table \ref{tab:personal-narrative-elicitation-experimental-setup} is presented a brief recap of all the experiments for this section. 
Finally, in order to minimise the computational cost and, at the same time, reduce the complexity of the human evaluation, this set of experiments was run only on the test set of the previously defined narrative elicitation dataset. In future works, we plan on testing the train set as well.
\subsubsection{Prompts and experimental details}
% \input{assets/table/personal-narrative-elicitation-prompts}
The exact prompts used during the experiments are reported here for completeness:
\begin{itemize}
    \item \textbf{Without guidelines}: \\ \emph{Sei una AI che deve fare una domanda su un racconto in maniera tale da ottenere più informazioni su eventi accaduti nel racconto. A seguire degli esempi e successivamente una narrativa a cui dovrai fare una domanda in modo da ottenere più informazioni.\\
       NARRATIVA: " \highLight[highlightgreen]{Oggi è stata una bella giornata. Mia moglie mi ha detto che sta aspettando un bambino!} Sono super felice! Mi chiedo se sarò un bravo padre. \highLight[highlightred]Mio padre non è stato molto presente quando ero un bambino."\\
       DOMANDA: "Sono felice di sentirlo. Sapete già se si tratta di un maschio o di una femmina ?"\\
       NARRATIVA: "\highLight[highlightred]{Oggi ho litigato con Chiara, lei era arrabbiata con me perché secondo lei non io so fare le cose.}"\\
       DOMANDA: "Oh, mi spiace che tu abbia litigato. Secondo lei che cosa è che non sai fare ?"\\
       NARRATIVA: "\highLight[highlightgreen]{Oggi è una bella giornata. Ho pattinato sul ghiaccio e poi sono andato al cinema.}"\\
       DOMANDA: "Bello sentire che è stata una buona giornata per te. Dove sei stato a pattinare ?"\\
       NARRATIVA: "Pensavo sempre a mio figlio che doveva uscire nel pomeriggio, questo è il motivo che mi ha scatenato l’ansia."\\
       DOMANDA: "Capisco, dove doveva andare tuo figlio?"\\
       NARRATIVA: "Mia figlia si è lasciata con il suo fidanzato ed ora ho sensi di colpa e momenti di tristezza, mi dispiace tanto e mi sento incapace di supportarla in questo. Insomma giornate un po’ grigie. Non so se il sonno disturbato e qualche episodio di insonnia siano causati da questa confusione."\\
       DOMANDA: "Mi dispiace tanto, da quanto erano insieme?"\\
       NARRATIVA:  '\{prompt\}'\\
       DOMANDA: }
       \item \textbf{With guidelines}: \\ \emph{Sei una AI che deve fare una domanda su un racconto in maniera tale da ottenere più informazioni su eventi accaduti nel racconto. A seguire degli esempi e successivamente una narrativa a cui dovrai fare una domanda in modo da ottenere più informazioni.\\
       \\
       Istruzioni\\
       Di seguito ti verrà presentato un insieme di racconti personali e il tuo obiettivo è quello di proporre domande riguardanti alcuni aspetti degli eventi descritti nella narrativa. Queste domande hanno come obiettivo quello di approfondire il racconto e/o chiarire alcune sue parti.\\
       Nello specifico, le tue domande potranno avere uno o più dei seguenti obiettivi:\\
       Approfondire alcuni aspetti della narrativa per ottenere più informazioni riguardanti eventi, persone o altre entità menzionate nel racconto. Per esempio, se il narratore racconta di un generico problema a casa, una possibilità è approfondire il relativo problema. Vedi esempio 2 nella tabella 1.\\
       Parti del testo della narrativa potrebbero essere evidenziate di verde o rosso per sottolineare emozioni positive ( verde ) o negative ( rosso ). Usa queste indicazioni per concentrare le tue domande su eventi emotivamente carichi ed evidenziati dalle parti di testo colorate ( verde o rosso ).\\
       Usa segnali di feedback per cominciare la tua domanda. ( ad esempio "sì, capisco", "oh", "che bello" ) per dimostrare che si è capito la parte precedente e che si è attivamente interessati alla narrazione. Vedi esempio 4 nella tabella 1.\\
       È molto importante mantenere la narrazione centrata sul narratore riferendosi ad eventi accaduti.\\
       Mostrare empatia con le tue domande. Se il narratore esprime una emozione negativa, il tuo obiettivo è quello di essere comprensivo. Invece se il narratore mostra una emozione positiva cerca di mostrare interesse nell'evento positivo. Vedi esempio 5 nella tabella 1.\\
       Cerca di mantenere le domande sintetiche e puntuali. Troppe domande o una domanda troppo lunga può confondere il narratore e quindi avere un effetto negativo sulla narrazione. Vedi esempio 4 nella tabella 1.\\
       Le tue domande devono suonare naturali e coerenti con il contesto, ovvero la narrativa.\\
       Non da ultimo, verifica la correttezza grammaticale e sintattica delle tue domande.\\
       Domande da evitare:\\
       Richiesta di opinioni personali (ad esempio "cosa pensi ...", "come speri di fare per ..." e simili). Vedi esempio 3 nella tabella 1.\\
       Suggerimenti (ad esempio, "forse potresti ...", "dovresti ...", "perché non ..."). Vedi esempio 6 nella tabella 1.\\
       Esprimere eventi ipotetici (ad esempio, previsioni future, illazioni e immedesimazioni in altri ruoli). Vedi esempio 7 nella tabella 1.\\
       Evita domande generiche. Per evitare questo problema ti è consigliato di riportare testualmente un esempio della narrativa Vedi esempio 2 , tabella 1\\
       Evita di spostare il fulcro della conversazione su di te ( osservatore ) o fare domande che divagano in altri argomenti. Vedi esempio 4 nella tabella 1.\\
       A seguire la tabella 1 che riporta una serie di esempi\\
       Tabella 1, contenente un esempio corretto e molteplici esempi errati di domande per una narrativa. Ciascun esempio è numerato.\\
       NARRATIVA:\\
       \highLight[highlightgreen]{Oggi è stata una bella giornata. Mia moglie mi ha detto che sta aspettando un bambino!} Sono super felice! Mi chiedo se sarò un bravo padre. \highLight[highlightred]{Mio padre non è stato molto presente quando ero un bambino.}\\
       \\
       Tabella 1\\
       Esempio	Testo	Valutazione	Spiegazione\\
       1	Sono felice di sentirlo. Sapete già se si tratta di un maschio o di una femmina ?	CORRETTO	Segue tutte le linee guida\\
       2	Oh capisco. Cosa mi racconti? 	ERRATO	Non esplora la narrativa, troppo generica\\
       3	Sono felice di sentirlo. Cosa ne pensi di essere un genitore?	ERRATO	È una opinione personale\\
       4	Sapete già se si tratta di una femmina o maschio? Di quanti mesi è incinta? Sai che io ho una figlia, si chiama Chiara.	ERRATO	 Non inizia con un feedback. Sposta la conversazione dal narratore. Non è sintetico e puntuale\\
       5	Oh capisco, sono felice che tuo padre non sia stato molto presente.	ERRATO	Non mostra empatia\\
       6	Oh, capisco. Per evitare questo problema ti consiglio di spendere molto tempo assieme alla tua famiglia	ERRATO	Mostra un suggerimento\\
       7	Oh capisco, come ti immagini sarà la tua vita da genitore?	ERRATO	Si tratta di una domanda ipotetica\\
      NARRATIVA: "\highLight[highlightgreen]{Oggi è stata una bella giornata. Mia moglie mi ha detto che sta aspettando un bambino!} Sono super felice! Mi chiedo se sarò un bravo padre. \highLight[highlightred]{Mio padre non è stato molto presente quando ero un bambino.}"\\
       DOMANDA: "Sono felice di sentirlo. Sapete già se si tratta di un maschio o di una femmina ?"\\
       NARRATIVA: "\highLight[highlightred]{Oggi ho litigato con Chiara, lei era arrabbiata con me perché secondo lei non io so fare le cose.}"\\
       DOMANDA: "Oh, mi spiace che tu abbia litigato. Secondo lei che cosa è che non sai fare ?"\\
       NARRATIVA: "\highLight[highlightgreen]{Oggi è una bella giornata. Ho pattinato sul ghiaccio e poi sono andato al cinema.}"\\
       DOMANDA: "Bello sentire che è stata una buona giornata per te. Dove sei stato a pattinare ?"\\
       \\
       NARRATIVA: "Pensavo sempre a mio figlio che doveva uscire nel pomeriggio, questo è il motivo che mi ha scatenato l’ansia."\\
       DOMANDA: "Capisco, dove doveva andare tuo figlio?"\\
       NARRATIVA: "Mia figlia si è lasciata con il suo fidanzato ed ora ho sensi di colpa e momenti di tristezza, mi dispiace tanto e mi sento incapace di supportarla in questo. Insomma giornate un po’ grigie. Non so se il sonno disturbato e qualche episodio di insonnia siano causati da questa confusione."\\
       DOMANDA: "Mi dispiace tanto, da quanto erano insieme?"\\
       Completa questo task\\
       NARRATIVA:  '\{prompt\}'\\
       DOMANDA:}
\end{itemize}
On this page, two versions are provided, with and without guidelines, both of which contain 5 examples. The 0, 1, 3, and 5 shot experiments all use the same base prompts, with the difference that 0, 1, 3, or 5 examples are shown respectively. In the prompts that are reported here, for clarity purposes, the narratives all have highlighted sections representing the valence value when present. However, as previously mentioned, the colour information is either omitted or induced with special text formatting shown in Table \ref{tab:personal-narrative-elicitation-color-example}. The text \emph{"prompt"} is replaced with the correct narrative at every inference step.
It is also possible to notice that all prompts used in this set of experiments use impersonation. We found that impersonation was very effective \cite{impersonation}. This marks a difference compared to the previous set of story cloze experiments, in which impersonation was not applied. Finally, to satisfy the heavy GPU memory requirements of some of the models we tested, the models were run on a pair of A100s 80GB from Nvidia.
\subsubsection{Results}
\label{cha:methodology-personal-narrative-elicitation-results}
In this section, an analysis of the results is conducted with different automatic metrics in a similar manner to what was done in \ref{cha:methodology-LLMs-selection-story-cloze-test-results}. However, as a marked difference from the previous set of experiments, a comparison of the LLM-generated data with the human crowdsourced data is also made.

\begin{table}[!htbp]
\centering
\caption{Two examples of narratives and model outputs for different experimental settings are shown in each of the two rows. To evaluate the models, their outputs are processed by removing special invalid characters and only the first sentence, marked by the dot (\emph{?}), is considered. In the middle, it is possible to read the raw unprocessed model output, which is unsuitable for our task. Therefore some post-processing operations are automatically applied in order to extract a better elicitation, which is reported in the second rightmost column. On the rightmost column the human crowdsourced eliciting question for the same narrative as reference. }
\label{tab:personal-narrative-elicitation-example-answers}
    \centering
    \begin{tabularx}{\linewidth}{p{2.5cm}| p{1.5cm} | p{2cm} | X | p{2cm} | p{2cm} }
    % \begin{tabular}{p{1.5cm}|p{3cm}|p{5cm}|p{2.5cm}|p{2cm}}
        \toprule
        \thead{Experimental \\ Setting} & \thead{Model \\ name} & \thead{Narrative} & \thead{Unprocessed \\Model \\ eliciting question}& \thead{Processed \\ Model \\eliciting question} & \thead{Human \\ eliciting question} \\
        \midrule
\textbf{0-shot without guidelines without colour} & Mpt 7B & Oggi sono molto più tranquilla . . . fine settimana gratificante in famiglia. Era il mio compleanno e finalmente avevo tutta la mia famiglia riunita.  &  Perché la tua vita è molto più tranquilla oggi? Risposta: 'Domani ho un esame. ' 'Domani mi è stato detto che la mia madre mi sta cercando, ma non è più la mia città.' 'Domani ho un prossimo di casa. ' 'Domani è la mia festa. ' 'Domani è l'anno nuovo e tutta la mia famiglia è qui. ' 'Domani devo pagare il mio bollo. ' 'Domani sono le vacanze. ' 'Domani sono i miei amici. ' 'Domani sono il mare. ' 'Domani sono le mie amiche. ' 'Domani sono le mie scuole. ' 'Domani sono le mie parole. ' 'Domani sono le mie gioie. ' 'Domani sono le mie parole...' & Perché la tua vita è molto più tranquilla oggi? & Mi fa piacere sentirlo, come avete celebrato? \\
\midrule
\textbf{5-shot with guidelines without colour} &  Vicuna 33B V1  & Ritornata dal lavoro mia figlia mi dice di aver chiamato il medico perché ha dei dolori alla testa non ha il senso dell'olfatto e del gusto per cui ci siamo un po' allarmati. & 1. Quali altri segni o sintomi ha segnato la tua figlia oltre al mal di testa, alla perdita del senso dell'olfatto e del gusto? 2. In quale momento della giornata ha chiamato il medico e quali esami o accertamenti ha richiesto? 3. Quali sono le tue preoccupazioni in merito a queste condizioni della tua figlia? 4. È stata presa alcuna decision & Quali altri segni o sintomi ha segnato la tua figlia oltre al mal di testa, alla perdita del senso dell'olfatto e del gusto? &Mi dispiace di sentirlo, da quanto tempo tua figlia ha questi sintomi? \\
        \bottomrule
    \end{tabularx}
\end{table}

Looking at the elicitation for continuation produced by the models, it is possible to witness that the problem of the answers of the model containing special non-text characters is still present, as shown in Table \ref{tab:personal-narrative-elicitation-example-answers}. The same behaviour of degeneration in the quality of the text is also present. Therefore, a similar answer pattern to the previous experiments was applied in order to retrieve the answers of the models, with the difference of using \emph{?} instead of \emph{.} as a sentence delimitator mark. This difference is motivated by the fact that our elicitations should be questions.
% 
\begin{center}
\label{tab:personal-narrative-elicitation-answers}
\begin{longtable}{l|l|l|p{3cm}|p{8cm}}
\caption{Model answers for each of the experimental setup. Notice that most 0 shot results are unusable and they get better with the number of shots and presence of guidelines. Notice how colour information negatively affects the models' responses. } \\ 
\toprule
\multicolumn{2}{l}{\thead{Narrative}} & \multicolumn{3}{}{Oggi sono molto più tranquilla . . . fine settimana gratificante in famiglia. Era il mio compleanno e finalmente avevo tutta la mia famiglia riunita}
\multicolumn{2}{l}{\thead{Human eliciting question}} & \multicolumn{3}{l}{Mi fa piacere sentirlo, come avete celebrato?}
\midrule
 \thead{Colour}& \thead{Guidelines} & \thead{Shot} & \thead{Model name} & \thead{Model eliciting question} \\
\midrule
\multirow[c]{70}{*}{\thead{Without Colour}} & \multirow[c]{36}{*}{\thead{Without Guidelines}} & \multirow[c]{9}{*}{0-shot} & Falcon 7B & Oggi era il compleanno. Quali presentimenti avevi prima del tuo compleanno? \\
 &  &  & Falcon 40B Instruct & Mi puoi dire di più sul compleanno della protagonista del racconto e quali attività hanno svolto la famiglia in quel fine settimana? \\
 &  &  & ChatGPT 3.5 turbo & Che tipo di attività hai svolto con la tua famiglia durante il tuo compleanno? \\
 &  &  & ChatGPT 4 & Come hai festeggiato il tuo compleanno con la tua famiglia? \\
 &  &  & Mpt 7B & Perché la tua vita è molto più tranquilla oggi? \\
 &  &  & Mpt 30B Chat & Cosa ti ha regalato la tua famiglia per il tuo compleanno? \\
 &  &  & Vicuna 13B V1 &  Cè un modo in cui mi potresti chiedere se vuoi ottenere ulteriori informazioni sulla tua esperienza di compleanno? \\
 &  &  & Vicuna 33B V1 &  Quali attività hanno organizzato per festeggiare il compleanno della protagonista? \\
 &  &  & Wizard Vicuna 13B Uncensored HF & Quali eventi accaddero durante il fine settimana che fecero del tuo compleanno una gratificante" esperienza per te? \\
\cline{3-5}
 &  & \multirow[c]{9}{*}{1-shot} & Falcon 7B & Mi piacerebbe sapere se siete andati tutti insieme in vacanza o se cè stato qualcuno in più. "NARRATIVA: 'Oggi è stato un giorno bellissimo. I miei figli hanno organizzato una festa di compleanno per me. Hanno invitato tutti i nostri amici, la famiglia e anche una persona che conosco da poco... "DOMANDA: ``Mi piacerebbe sapere se è stato un party a sorpresa o se è venuta una persona in più. "NARRATIVA: 'La mia bambina ha fatto la sua prima giornata di scuol Sono molto orgoglioso del fatto che sia così coraggiosa. Mi ha anche detto di avere imparato molte cose. 'DOMANDA: ``Mi piacerebbe sapere se la bambina ha imparato a suonare un'instruments o qualcosa di simile. "NARRATIVA: ``Il tempo è stato splendido stamattina. Mi sono alzato presto e ho fatto un giro in bicicletta. 'DOMANDA: ``Mi piacerebbe sapere se è stato il caso di fare un giro in bicicletta. " A questo punto, è possibile procedere come si è visto nel capitolo precedente e, selezionando l'opzione _Selezionare elementi a caso_, fare un'indagine sul racconto. \\
 &  &  & Falcon 40B Instruct & Che cosa è stato fatto nel weekend per rendere speciale il vostro compleanno? \\
 &  &  & ChatGPT 3.5 turbo & Che cosa avete fatto per festeggiare il vostro compleanno? \\
 &  &  & ChatGPT 4 & Che tipo di attività avete svolto durante il fine settimana per celebrare il tuo compleanno? \\
 &  &  & Mpt 7B & Come siete riusciti a organizzare il compleanno? \\
 &  &  & Mpt 30B Chat & Come hai passato il tuo compleanno? \\
 &  &  & Vicuna 13B V1 &  Siet \\
 &  &  & Vicuna 33B V1 &  Quanto dura che la famiglia è riunita insieme in questo weekend? \\
 &  &  & Wizard Vicuna 13B Uncensored HF & Come ti è andata il compleanno? \\
\cline{3-5}
 &  & \multirow[c]{9}{*}{3-shot} & Falcon 7B & Oh che bello! Come ti senti adesso? \\
 &  &  & Falcon 40B Instruct & Mi chiedevo se puoi dirmi cosa ha reso speciale la tua famiglia per questa volt' I hope this helps! \\
 &  &  & ChatGPT 3.5 turbo & Mi fa piacere sentire che hai trascorso un fine settimana gratificante in famigli Mi chiedo, cosa avete fatto per festeggiare il tuo compleanno? \\
 &  &  & ChatGPT 4 & Sembra che tu abbia avuto un bel compleanno. Cè stato un momento particolare che ti ha reso felice durante la tua festa di compleanno? \\
 &  &  & Mpt 7B & Sì! Quando è stato tuo compleanno? \\
 &  &  & Mpt 30B Chat & Cosa volete dire con finita settimana gratificante in famiglia"? \\
 &  &  & Vicuna 13B V1 & Mi fa piacere che tu abbia trascorso una giornata piacevole con la tua famigli Quale era il tuo compleanno? \\
 &  &  & Vicuna 33B V1 & Sapete cosa hai fatto o avuto per il tuo compleanno? \\
 &  &  & Wizard Vicuna 13B Uncensored HF & Oh, sono felice per te che hai avuto una giornata così speciale per il tuo compleanno. Hai avuto la tua famiglia intera qui? \\
\cline{3-5}
 &  & \multirow[c]{9}{*}{5-shot} & Falcon 7B & Che bello. Che cosa facevate? \\
 &  &  & Falcon 40B Instruct & Sono felice per te, sai che hai un grande valore e meriti di stare bene e di essere circondata dalle persone che ami. Mi chiedo, hai pensato a festeggiare il compleanno con un evento particolare, tipo un viaggio? \\
 &  &  & ChatGPT 3.5 turbo & Mi fa piacere sentire che hai trascorso un fine settimana gratificante in famigli Mi chiedo, quali membri della tua famiglia erano presenti per festeggiare il tuo compleanno? \\
 &  &  & ChatGPT 4 & Che bello! Cosa avete fatto per festeggiare il tuo compleanno? \\
 &  &  & Mpt 7B & Mi dispiace tanto, che hai avuto un ottimo compleanno e ora sono felice che sia riunita la tua famigli Perché non avete mai visto ogni tanto? \\
 &  &  & Mpt 30B Chat & Che cosa ha fatto la vostra famiglia per festeggiare il tuo compleanno? \\
 &  &  & Vicuna 13B V1 & Spero tu abbia passato un buon compleanno. Chi c'era la tua famiglia? \\
 &  &  & Vicuna 33B V1 &  Che cosa hai fatto durante il compleanno? \\
 &  &  & Wizard Vicuna 13B Uncensored HF & Hai avuto un compleanno meraviglioso. E dove siete state riunite? \\
\cline{2-5} \cline{3-5}
 & \multirow[c]{34}{*}{\thead{With Guidelines}} & \multirow[c]{9}{*}{0-shot} & Falcon 7B & Ti senti più tranquilla? \\
 &  &  & Falcon 40B Instruct & Qual è il tuo rapporto con le altre persone della tua famiglia? \\
 &  &  & ChatGPT 3.5 turbo & Che cosa avete fatto per festeggiare il tuo compleanno con tutta la tua famiglia riunita? \\
 &  &  & ChatGPT 4 & Sembra che tu abbia avuto un bel compleanno! Puoi condividere qualche dettaglio speciale di come hai festeggiato con la tua famiglia? \\
 &  &  & Mpt 7B & 1) Questa è la nostra ultima visita di famigli Perché la trovi così bella? \\
 &  &  & Mpt 30B Chat &  Che cosa vuoi dire con tranquilla'? \\
 &  &  & Vicuna 13B V1 & ``` ruby from telegram import InlineKeyboardButton, InlineKeyboardMarkup, ReplyKeyboardMarkup def start(update, context): keyboard = [[InlineKeyboardButton(Domande", callback_data=quer ``` \\
 &  &  & Vicuna 33B V1 & Sì, capisco. Che bello avere una famiglia così unit Raccontami di come hai passato il tuo compleanno con loro. \\
 &  &  & Wizard Vicuna 13B Uncensored HF & Per approfondire la narrazione, le mie domande riguardano il significato della frase finally all my family was reunited'. Potrei chiedere se questa è la prima volta che la tua famiglia è riunita o perché hai aspettato a festeggiare il tuo compleanno con loro? \\
\cline{3-5}
 &  & \multirow[c]{9}{*}{1-shot} & Falcon 7B & Che ne pensi di essere un genitore? \\
 &  &  & Falcon 40B Instruct & Sono contenta che tu abbia passato un buon compleanno. Mi dispiace di non potere essere stata con voi. Com'è stato il tempo? \\
 &  &  & ChatGPT 3.5 turbo & Oh, capisco. Sono felice che hai avuto un fine settimana gratificante in famigli Cosa avete fatto per festeggiare il tuo compleanno? \\
 &  &  & ChatGPT 4 & Sembra che tu abbia avuto un bel compleanno! Puoi condividere qualche dettaglio in più su come hai festeggiato? \\
 &  &  & Mpt 7B & Quando hai avuto il tuo compleanno? \\
 &  &  & Mpt 30B Chat & 1) Come è avuto inizio il vostro compleanno? \\
 &  &  & Vicuna 13B V1 & Mi chiedo perché sei così felice in questa occasione. \\
 &  &  & Vicuna 33B V1 & Cosa ti ha reso più tranquilla durante il fine settimana in famiglia? \\
 &  &  & Wizard Vicuna 13B Uncensored HF & Che cosa è successo quest weekend '? \\
\cline{3-5}
 &  & \multirow[c]{8}{*}{3-shot} & Falcon 7B & Che bella notizia, Felice! "DOMANDA: ``Che cosa ne pensa tua moglie della tua decisione? \\
 &  &  & Falcon 40B Instruct & Che emozione ti ha dato riunire tutta la tua famiglia? \\
 &  &  & ChatGPT 3.5 turbo & Che cosa hai fatto nel fine settimana con la tua famiglia per festeggiare il tuo compleanno? \\
 &  &  & ChatGPT 4 & Sembra che tu abbia avuto un bel compleanno in famigli Cè stato un momento particolare che ti ha reso particolarmente felice? \\
 &  &  & Mpt 30B Chat & ... sì, capisco. E come hai passato il tuo compleanno? \\
 &  &  & Vicuna 13B V1 & A) Cosa hai fatto per celebrare il tuo compleanno? \\
 &  &  & Vicuna 33B V1 & È bello passare del tempo tutto insieme. È successo qualche bello evento nel corso del weekend? \\
 &  &  & Wizard Vicuna 13B Uncensored HF & Oh, è bello che tu abbia avuto una giornata piacevole. Mi piacerebbe sapere che cosa hai fatto per festeggiare il tuo compleanno." \\
\cline{3-5}
 &  & \multirow[c]{8}{*}{5-shot} & Falcon 7B & Come hai passato il compleanno? \\
 &  &  & Falcon 40B Instruct & Che cosa vi ha uniti? \\
 &  &  & ChatGPT 3.5 turbo & Che cosa avete fatto per festeggiare il tuo compleanno? \\
 &  &  & ChatGPT 4 & Sembra che tu abbia avuto un bel compleanno. Cera qualche membro della famiglia che non vedevi da molto tempo? \\
 &  &  & Mpt 30B Chat & Oh, capisco.... E che cosa è stato il regalo che preferiresti di tutto? \\
 &  &  & Vicuna 13B V1 & NARR \\
 &  &  & Vicuna 33B V1 & Sì, capisco. Era il tuo compleanno, avevi organizzato qualche evento o ti avevano regalato qualcosa che ti ha fatto felice? \\
 &  &  & Wizard Vicuna 13B Uncensored HF & Sono contento che tu stia sentendoti più tranquill Come hai trascorso il tuo compleanno? \\
\cline{1-5} \cline{2-5} \cline{3-5}
\multirow[c]{68}{*}{\thead{With Colour}} & \multirow[c]{36}{*}{\thead{Without Guidelines}} & \multirow[c]{9}{*}{0-shot} & Falcon 7B & [VERDE](Per favore, qual è il nome del tuo fidanzato? \\
 &  &  & Falcon 40B Instruct & What was the most memorable event that occurred during the familys gathering on the narrator's birthday? \\
 &  &  & ChatGPT 3.5 turbo & Quali sono state le attività che hai svolto con la tua famiglia durante il tuo compleanno? \\
 &  &  & ChatGPT 4 & Quali sono stati gli eventi o le attività che avete svolto durante il fine settimana del tuo compleanno che lo hanno reso così gratificante? \\
 &  &  & Mpt 7B & [VERDE](Cosa pensi di fare con i tuoi parenti per il compleanno? \\
 &  &  & Mpt 30B Chat & Mmm scusa, mi ha colpito che eravate il compleanno, quanti anni avete? \\
 &  &  & Vicuna 13B V1 & Quali sono i momenti di gioia e gratitudine che l'hanno resa così tranquilla? \\
 &  &  & Vicuna 33B V1 &  Qual è la data del compleanno della persona? \\
 &  &  & Wizard Vicuna 13B Uncensored HF & In che modo hai saputo che era il compleanno di qualcuno nella tua famiglia? \\
\cline{3-5}
 &  & \multirow[c]{9}{*}{1-shot} & Falcon 7B & [MAROON](Mi chiedo se il tuo compleanno sia diverso dal mio. Per me è solo il giorno delle mie nozze di santo, quando mio padre è morto. [VERDE](Oggi sono molto più tranquill' DOMANDA: '[VERDE](Oggi è stata una bella giornat Mia moglie mi ha detto che sta aspettando un bambino! Sono super felice!) Io penso che sia una ragazza perché mia mamma mi ha detto che il primo figlio di lei era una femmina. [VERDE](Mia madre è andata a trovare mio padre.') DOMAINE: [VERDE](Mio padre è un medico molto bravo.') NARRATIVA: '[VERDE](Oggi sono molto più tranquilla)] DOMAINE: '[VERDE](Oggi sono molto più tranquilla)' NARRATIVA: '[ROSSO](Ieri sono passato a salutare mia moglie e il mio primo nipote. Sono molto soddisfatto, sono contento dei miei due figli.' DOMANDA: '[MAROON](' [ROSSO](Ieri sono passato a salutare mia moglie e il mio primo nipote.')' \\
 &  &  & Falcon 40B Instruct & È bello sapere che il tuo compleanno è stato un successo. Quali sono stati i momenti migliori della giornata? \\
 &  &  & ChatGPT 3.5 turbo & Come hai festeggiato il tuo compleanno con la tua famiglia? \\
 &  &  & ChatGPT 4 & Mi fa piacere che tu sia tranquill Come hai festeggiato il tuo compleanno con la tua famiglia? \\
 &  &  & Mpt 7B & NARRATIVA DOMANDA NARRATIVA DOMANDA NARRATIVA DOMANDA NARRATIVA DOMANDA NARRATIVA DOMANDA \\
 &  &  & Mpt 30B Chat & Quante persone sono presenti insieme per il vostro compleanno? \\
 &  &  & Vicuna 13B V1 & Sieti più felice quando ci sei tutta la vostra famiglia insieme? \\
 &  &  & Vicuna 33B V1 & E... come ha passato il tuo compleanno? \\
 &  &  & Wizard Vicuna 13B Uncensored HF &  Quale è stato il tuo compleanno? \\
\cline{3-5}
 &  & \multirow[c]{9}{*}{3-shot} & Falcon 7B & Che bello sapere che è andata così bene per te. Come si chiama la tua famiglia? \\
 &  &  & Falcon 40B Instruct & Mi fa piacere sapere che ti sei sentita più tranquill Hai passato un weekend in famiglia? \\
 &  &  & ChatGPT 3.5 turbo & Mi fa piacere sentire che sei più tranquilla oggi. Mi chiedo, cosa hai fatto durante il tuo compleanno con tutta la tua famiglia riunita? \\
 &  &  & ChatGPT 4 & Sono contenta che tu sia riuscita a riunire tutta la tua famiglia per il tuo compleanno. Cè stato un momento particolare che ti ha reso felice durante la celebrazione? \\
 &  &  & Mpt 7B & NARRATIVA: [ROSSO](Oggi è stata una bella giornat Ho chiamato il medico per parlare di una operazione.)" DOMANDA: ``Dove vive il medico? \\
 &  &  & Mpt 30B Chat & Che tipo di regalo ti hai fatto fare per il tuo compleanno? \\
 &  &  & Vicuna 13B V1 & Oh, mi fa piacere sentire che sei più tranquill Dovè che hai passato la tua festa di compleanno? \\
 &  &  & Vicuna 33B V1 &  Mi scusi, lei ha specificato di essere'molto più tranquilla' oggi. Ciò è dovuto a qualcosa che è accaduto di recente? \\
 &  &  & Wizard Vicuna 13B Uncensored HF & Hai passato una giornata molto piacevole? \\
\cline{3-5}
 &  & \multirow[c]{9}{*}{5-shot} & Falcon 7B & NARRATIVA: [VERDE](Oggi non sono felice perché mia moglie non fa parte della mia festa di compleanno. Mi sento un po solo. ') DOMAINA: '[VERDE](Sono felice per il fatto di essere una nuova generazione di genitori e di crescere il bambino con le migliori tecnologie disponibili.' DOMANDA: '[VERDE](Come mai siete diventati una famiglia numerosa? \\
 &  &  & Falcon 40B Instruct & Sono molto contento di ascoltare che sei più tranquilla, ma posso chiederti quale è stata la cosa più bella del tuo compleanno? \\
 &  &  & ChatGPT 3.5 turbo & Che cosa avete fatto per festeggiare il tuo compleanno? \\
 &  &  & ChatGPT 4 & Sono contento che tu abbia trascorso un bel compleanno in famigli Quanti anni hai compiuto? \\
 &  &  & Mpt 7B & Molto bene, ora che hai tutta la tua famiglia riunita, cosa ti è piaciuta di più? \\
 &  &  & Mpt 30B Chat & Che meraviglia! Che genere di regalo hai ricevuto per il tuo compleanno? \\
 &  &  & Vicuna 13B V1 & E tu sei felice per la prima volta in un po ? \\
 &  &  & Vicuna 33B V1 &  \\
 &  &  & Wizard Vicuna 13B Uncensored HF & Bene, mi sembra che hai avuto un bel compleanno. Sembra davvero soddisfacente avere la tua famiglia riunita per festeggiare insieme. Hai avuto lopportunità di mangiare qualcosa di speciale per il compleanno? \\
\cline{2-5} \cline{3-5}
 & \multirow[c]{32}{*}{\thead{With Guidelines}} & \multirow[c]{9}{*}{0-shot} & Falcon 7B & [ROSSO](Ti ha fatto una sorpresa? \\
 &  &  & Falcon 40B Instruct & [ROSSO] Perché ti sentivi tesa prima del tuo compleanno? \\
 &  &  & ChatGPT 3.5 turbo & Che cosa hai fatto nel tuo compleanno con la tua famiglia riunita? \\
 &  &  & ChatGPT 4 & Sembra che tu abbia avuto un bel compleanno. Puoi condividere qualche dettaglio speciale di come hai festeggiato con la tua famiglia? \\
 &  &  & Mpt 7B & E molto bello che ora ha il completo supporto di tutta la famiglia? \\
 &  &  & Mpt 30B Chat & / \\
 &  &  & Vicuna 13B V1 &  Ciao, mi fa piacere sentire che sei più tranquillo or Cosa ti ha portato a sentirti così? \\
 &  &  & Vicuna 33B V1 &  VERDE: Per quale motivo hai avuto una giornata molto più tranquilla? \\
 &  &  & Wizard Vicuna 13B Uncensored HF &  \\
\cline{3-5}
 &  & \multirow[c]{9}{*}{1-shot} & Falcon 7B & Oh capisco. Cosa ti fa sentire tranquillo? \\
 &  &  & Falcon 40B Instruct & Come ti sei sentita dopo il compleanno? \\
 &  &  & ChatGPT 3.5 turbo & Che cosa avete fatto per festeggiare il tuo compleanno con tutta la tua famiglia riunita? \\
 &  &  & ChatGPT 4 & Sembra che tu abbia avuto un bel compleanno. Cera qualche membro della famiglia che non vedevi da molto tempo? \\
 &  &  & Mpt 7B &  Sapete se è stata la prima volta che tuo papà si trova a casa? \\
 &  &  & Mpt 30B Chat &  Cosa ti ha fatto felice di più durante la tua [ROSSO](solitari [VERDE](felice) [ROSSO](solitaria) fin [VERDE](tanta) [ROSSO](soltanto) [VERDE](bellissima) [ROSSO](solitaria) [VERDE](famigliare) [ROSSO](vacanza)? \\
 &  &  & Vicuna 13B V1 &  Che fa il compleanno? \\
 &  &  & Vicuna 33B V1 &  \\
 &  &  & Wizard Vicuna 13B Uncensored HF & Mi chiedo se hai avuto un buon compleanno, non trovi la mia domanda come un tentativo di fare una sorpresa? \\
\cline{3-5}
 &  & \multirow[c]{7}{*}{3-shot} & Falcon 7B &  [ROSSO](Spero di essere una brava madre... Spero di essere una brava madre... Spero di essere una brava madre... Spero di essere una brava madre... Spero di essere una brava madre... [ROSSO](fine settimana gratificante in famigli. [ROSSO] )... [ROSSO]...... fine settimana gratificante in \\
 &  &  & Falcon 40B Instruct & NARRATIVA: [ROSSO](Ho litigato con Chiara, lei era arrabbiata con me perché secondo lei non io so fare le cose.)" DOMANDA: NARRATIVA: ``[VERDE](Oggi è stata una bella giornat Ho pattinato sul ghiaccio e poi sono andato al cinema.)" DOMANDA: The following table provides an example of an incorrect question for a narrative, as well as examples of how to correctly ask questions based on the content of the narrative.  Are you happy to hear that \\
 &  &  & ChatGPT 3.5 turbo & Che cosa hai fatto nel tuo compleanno con la tua famiglia riunita? \\
 &  &  & ChatGPT 4 & Sembra che tu abbia avuto un bel compleanno. Cera qualche evento speciale o tradizione che hai condiviso con la tua famiglia durante la celebrazione? \\
 &  &  & Vicuna 13B V1 & ) A) Sono felice di sentirlo. Quale è stato il momento più bello della tua festa di compleanno? \\
 &  &  & Vicuna 33B V1 & ) \\
 &  &  & Wizard Vicuna 13B Uncensored HF &  \\
\cline{3-5}
 &  & \multirow[c]{7}{*}{5-shot} & Falcon 7B & Che tipo di regalo ti sarebbe piaciuto di più? \\
 &  &  & Falcon 40B Instruct & NARRATIVA: [ROSSO](Non vedo lora che arrivi il mio anniversario di matrimonio). [ROSSO] (Mi sento un po’ triste).' DOMANDA: NARRATIVA: Mi chiedo se il mio rapporto di amicizia con [ROSSO] sia a rischio." DOMANDA: NARRATIVA: ``Mi sento un po’ triste perché la mia migliore amica oggi mi ha fatto un torto" DOMANDA: NARRAT \\
 &  &  & ChatGPT 3.5 turbo & Che cosa avete fatto per festeggiare il tuo compleanno con la tua famiglia riunita? \\
 &  &  & ChatGPT 4 & Sono contento che tu sia tranquilla e che tu abbia trascorso un bel fine settiman Quanti anni hai compiuto nel tuo compleanno? \\
 &  &  & Vicuna 13B V1 & 48. Usa la tecnologia per cre \\
 &  &  & Vicuna 33B V1 &  \\
 &  &  & Wizard Vicuna 13B Uncensored HF &  \\
% \cline{1-5} \cline{2-5} \cline{3-5}
\bottomrule

\end{longtable}

\end{center}

% Due to size constraints, in the appendix, Table \ref{tab:personal-narrative-elicitation-answers} reports examples of elicitations for a single narrative across all different models. % The three tables presented should highlight that the raw output of all models, except for ChatGPT models, is unsuitable for this task. These results are in line with our previous experiments.

As a side note, two models, mpt 7b and mpt 30b, had issues with their tokenisers for longer prompts. Due to time constraints, we were unable to fix and retest their metrics.

% \begin{table}[!htbp]
\centering
\caption{Examples of eliciting questions for two narratives are reported on each row. Narratives are reported in the first column and corresponding eliciting questions are reported in the second column. On the top a longer narrative with two different eliciting questions. Notice that eliciting questions for the longer narrative pursue different topics. On the bottom is a shorter narrative, with more eliciting questions but on the same few set of topics.}
\label{tab:personal-narrative-elicitation-continuations-example}
    \centering
    \begin{tabularx}{\linewidth}{ l|X | X  }
    % \begin{tabular}{p{1.5cm}|p{3cm}|p{5cm}|p{2.5cm}|p{2cm}}
        \toprule
        \multicolumn{3}{c}{\thead{Examples of narrative and respective eliciting questions}} \\
        \midrule
        % \thead{Mode}
        % \midrule
        \thead{Example}& \thead{Narrative} & \thead{Crowdsourced Eliciting Questions} \\
        \midrule
        \thead{1} & \multirow{2}{7cm}{Giornata piacevole ma stancante. Comunione di una nipotina. Oggi non è stata una giornata abbastanza calda ... mangiare al freddo non è il massimo. Non ero emozionata sapendo dove era il posto ... mi sono coperta per quanto possibile visto il periodo.} &  Congratulazioni per il bellissimo evento. La tua nipotina è stata felice? \\
 [2em]
       %        \cmidrule{2-2}
        && Se non altro hai allenato il tuo spirito di adattamento. Spero che la giornata sia andata bene, eri contenta alla fine della giornata? \\
        \arrayrulecolor{black}
        \midrule
        \thead{2 } & \multirow[t]{10}{*}{Che noia finiranno le feste?} & Mi spiace tu ti annoi, come mai?\\
 [1em]
       %        \cmidrule{2-2}
        && Ti capisco. Come mai non ti piacciono?\\
 [1em]
       %        \cmidrule{2-2}
        && Capisco, immagino che hai trascorso delle belle giornate!\\
 [2em]
       %        \cmidrule{2-2}
        && Mi dispiace non ti piacciano, perché vuoi che finiscano?\\
 [2em]
       %        \cmidrule{2-2}
        && Mi dispiace, che cosa hai fatto durante le feste che ti ha annoiato?\\
 [2em]
       %        \cmidrule{2-2}
        && Mi spiace tu stia così, che cosa ti causa noia in queste feste?\\
 [2em]
       %        \cmidrule{2-2}
        && Mi dispiace che ti annoi, perché vuoi che finiscano?\\     
 [1em]
       %        \cmidrule{2-2}
        && Concordo, quando ricominci a lavorare tu?\\      
 [1em]
       %        \cmidrule{2-2}
        && Ti capisco, per quale motivo ti senti annoiato?\\      
 [1em]
       %        \cmidrule{2-2}
        && Mi dispiace che le feste ti annoino, come mai non ti piacciono?\\
       \arrayrulecolor{black}
        \bottomrule

    \end{tabularx}
\end{table}
              % "input": "Giornata piacevole ma stancante. Comunione di una nipotina. Oggi non \u00e8 stata una giornata abbastanza calda ... mangiare al freddo non \u00e8 il massimo. Non ero emozionata sapendo dove era il posto ... mi sono coperta per quanto possibile visto il periodo. ",
              %   "output": [
              %       {
              %           "generated_text": "Congratulazioni per il bellissimo evento. La tua nipotina \u00e8 stata felice? "
              %       },
              %       {
              %           "generated_text": "Se non altro hai allenato il tuo spirito di adattamento. Spero che la giornata sia andata bene, eri contenta alla fine della giornata?"
              %       }
              %   ],
              %   "id": "264"
              %               {
              %   "input": "Che noia finiranno le feste? ",
              %   "output": [
              %       {
              %           "generated_text": "Mi spiace tu ti annoi, come mai?"
              %       },
              %       {
              %           "generated_text": "Ti capisco. Come mai non ti piacciono?"
              %       },
              %       {
              %           "generated_text": "Capisco, immagino che hai trascorso delle belle giornate!"
              %       },
              %       {
              %           "generated_text": "Mi dispiace non ti piacciano, perch\u00e9 vuoi che finiscano?"
              %       },
              %       {
              %           "generated_text": "Mi dispiace, che cosa hai fatto durante le feste che ti ha annoiato?"
              %       },
              %       {
              %           "generated_text": "Mi spiace tu stia cos\u00ec, che cosa ti causa noia in queste feste?"
              %       },
              %       {
              %           "generated_text": "Mi dispiace che ti annoi, perch\u00e9 vuoi che finiscano?"
              %       },
              %       {
              %           "generated_text": "Concordo, quando ricominci a lavorare tu?"
              %       },
              %       {
              %           "generated_text": "Ti capisco, per quale motivo ti senti annoiato?"
              %       },
              %       {
              %           "generated_text": "Mi dispiace che le feste ti annoino, come mai non ti piacciono?"
              %       }
              %   ],
The first type of evaluation results are automatic evaluation metrics, such as BLEU, METEOR and ROUGE, which were previously used in the story cloze test. In this case, there is no actual ground truth data in the task of personal narrative elicitation; as ground truth, the crowdsourced elicitation data from section \ref{cha:methodology-data-collection} is used instead. In a similar way to the story cloze test, there is also a problem with different possibilities of elicitation. It is clear that for a generic narrative, there are different directions of eliciting the narrator, as shown in the examples provided in Table \ref{tab:personal-narrative-elicitation-continuations-example}. This issue can be partially mitigated by analysing shorter narratives, as their shorter nature tends to limit the directions of a natural elicitation. This fact aligns nicely with the fact that in the data collection, more than one elicitation was sampled for a few select short narratives. 

\begin{table}
    \centering
    \caption{BLEU 1 scores across the tested models for all experiments. OpenAI models score consistently higher than the rest. Observe that increasing the number of shots on average does yield better results.}
    \label{tab:personal-narrative-elicitation-bleu}
\begin{tabular}{l|l|l|rrrr}
\toprule
\multicolumn{7}{c}{\thead{BLEU}}\\
\midrule
\thead{Colour} & \thead{Guidelines} & \thead{Model name} & \thead{0-shot} & \thead{1-shot} & \thead{3-shot} & \thead{5-shot} \\
\midrule
\multirow[c]{18}{*}{\thead{Without\\ Colour}} & \multirow[c]{9}{*}{\thead{Without\\ Guidelines}} & Falcon 7B & {\cellcolor[HTML]{D4E4F4}} \color[HTML]{000000} 0.041 & {\cellcolor[HTML]{D7E6F5}} \color[HTML]{000000} 0.039 & {\cellcolor[HTML]{BDD7EC}} \color[HTML]{000000} 0.056 & {\cellcolor[HTML]{CADEF0}} \color[HTML]{000000} 0.048 \\
 &  & Falcon 40B Instruct & {\cellcolor[HTML]{B9D6EA}} \color[HTML]{000000} 0.058 & {\cellcolor[HTML]{99C7E0}} \color[HTML]{000000} 0.072 & {\cellcolor[HTML]{75B4D8}} \color[HTML]{000000} 0.085 & {\cellcolor[HTML]{66ABD4}} \color[HTML]{F1F1F1} 0.091 \\
 &  & Gpt 3 & {\cellcolor[HTML]{4F9BCB}} \color[HTML]{F1F1F1} 0.1 & {\cellcolor[HTML]{3E8EC4}} \color[HTML]{F1F1F1} 0.11 & {\cellcolor[HTML]{125EA6}} \color[HTML]{F1F1F1} 0.14 & {\cellcolor[HTML]{3181BD}} \color[HTML]{F1F1F1} 0.12 \\
 &  & Gpt 4 & {\cellcolor[HTML]{2C7CBA}} \color[HTML]{F1F1F1} 0.12 & {\cellcolor[HTML]{4896C8}} \color[HTML]{F1F1F1} 0.11 & {\cellcolor[HTML]{2070B4}} \color[HTML]{F1F1F1} 0.13 & {\cellcolor[HTML]{1C6BB0}} \color[HTML]{F1F1F1} 0.13 \\
 &  & Mpt 7B & {\cellcolor[HTML]{D0E1F2}} \color[HTML]{000000} 0.045 & {\cellcolor[HTML]{B5D4E9}} \color[HTML]{000000} 0.059 & {\cellcolor[HTML]{A1CBE2}} \color[HTML]{000000} 0.069 & {\cellcolor[HTML]{CADDF0}} \color[HTML]{000000} 0.049 \\
 &  & Mpt 30B Chat & {\cellcolor[HTML]{8CC0DD}} \color[HTML]{000000} 0.077 & {\cellcolor[HTML]{B0D2E7}} \color[HTML]{000000} 0.062 & {\cellcolor[HTML]{B0D2E7}} \color[HTML]{000000} 0.062 & {\cellcolor[HTML]{C7DCEF}} \color[HTML]{000000} 0.051 \\
 &  & Vicuna 13B V1 & {\cellcolor[HTML]{4896C8}} \color[HTML]{F1F1F1} 0.1 & {\cellcolor[HTML]{9CC9E1}} \color[HTML]{000000} 0.071 & {\cellcolor[HTML]{77B5D9}} \color[HTML]{000000} 0.085 & {\cellcolor[HTML]{91C3DE}} \color[HTML]{000000} 0.075 \\
 &  & Vicuna 33B V1 & {\cellcolor[HTML]{A8CEE4}} \color[HTML]{000000} 0.066 & {\cellcolor[HTML]{CCDFF1}} \color[HTML]{000000} 0.047 & {\cellcolor[HTML]{8CC0DD}} \color[HTML]{000000} 0.077 & {\cellcolor[HTML]{7CB7DA}} \color[HTML]{000000} 0.083 \\
 &  & Wizard Vicuna 13B Uncensored HF & {\cellcolor[HTML]{A0CBE2}} \color[HTML]{000000} 0.07 & {\cellcolor[HTML]{B7D4EA}} \color[HTML]{000000} 0.059 & {\cellcolor[HTML]{99C7E0}} \color[HTML]{000000} 0.073 & {\cellcolor[HTML]{8FC2DE}} \color[HTML]{000000} 0.076 \\
\cmidrule{2-7}
 & \multirow[c]{9}{*}{\thead{With \\Guidelines}} & Falcon 7B & {\cellcolor[HTML]{C8DCF0}} \color[HTML]{000000} 0.05 & {\cellcolor[HTML]{C7DBEF}} \color[HTML]{000000} 0.051 & {\cellcolor[HTML]{AFD1E7}} \color[HTML]{000000} 0.063 & {\cellcolor[HTML]{C3DAEE}} \color[HTML]{000000} 0.053 \\
 &  & Falcon 40B Instruct & {\cellcolor[HTML]{D6E5F4}} \color[HTML]{000000} 0.04 & {\cellcolor[HTML]{B4D3E9}} \color[HTML]{000000} 0.06 & {\cellcolor[HTML]{71B1D7}} \color[HTML]{F1F1F1} 0.087 & {\cellcolor[HTML]{A8CEE4}} \color[HTML]{000000} 0.066 \\
 &  & Gpt 3 & {\cellcolor[HTML]{125DA6}} \color[HTML]{F1F1F1} 0.14 & {\cellcolor[HTML]{08488E}} \color[HTML]{F1F1F1} 0.15 & {\cellcolor[HTML]{084A91}} \color[HTML]{F1F1F1} 0.15 & {\cellcolor[HTML]{084488}} \color[HTML]{F1F1F1} 0.15 \\
 &  & Gpt 4 & {\cellcolor[HTML]{3787C0}} \color[HTML]{F1F1F1} 0.11 & {\cellcolor[HTML]{5AA2CF}} \color[HTML]{F1F1F1} 0.097 & {\cellcolor[HTML]{2171B5}} \color[HTML]{F1F1F1} 0.13 & {\cellcolor[HTML]{1865AC}} \color[HTML]{F1F1F1} 0.13 \\
 &  & Mpt 7B & {\cellcolor[HTML]{D1E2F3}} \color[HTML]{000000} 0.043 & {\cellcolor[HTML]{D3E3F3}} \color[HTML]{000000} 0.042 & {\cellcolor[HTML]{000000}} \color[HTML]{F1F1F1} nan & {\cellcolor[HTML]{000000}} \color[HTML]{F1F1F1} nan \\
 &  & Mpt 30B Chat & {\cellcolor[HTML]{B5D4E9}} \color[HTML]{000000} 0.059 & {\cellcolor[HTML]{79B5D9}} \color[HTML]{000000} 0.084 & {\cellcolor[HTML]{AACFE5}} \color[HTML]{000000} 0.065 & {\cellcolor[HTML]{ADD0E6}} \color[HTML]{000000} 0.064 \\
 &  & Vicuna 13B V1 & {\cellcolor[HTML]{AFD1E7}} \color[HTML]{000000} 0.062 & {\cellcolor[HTML]{C7DCEF}} \color[HTML]{000000} 0.051 & {\cellcolor[HTML]{9AC8E0}} \color[HTML]{000000} 0.072 & {\cellcolor[HTML]{75B4D8}} \color[HTML]{000000} 0.085 \\
 &  & Vicuna 33B V1 & {\cellcolor[HTML]{BED8EC}} \color[HTML]{000000} 0.056 & {\cellcolor[HTML]{B5D4E9}} \color[HTML]{000000} 0.059 & {\cellcolor[HTML]{8ABFDD}} \color[HTML]{000000} 0.077 & {\cellcolor[HTML]{B0D2E7}} \color[HTML]{000000} 0.062 \\
 &  & Wizard Vicuna 13B Uncensored HF & {\cellcolor[HTML]{DEEBF7}} \color[HTML]{000000} 0.034 & {\cellcolor[HTML]{DAE8F6}} \color[HTML]{000000} 0.036 & {\cellcolor[HTML]{9AC8E0}} \color[HTML]{000000} 0.072 & {\cellcolor[HTML]{BED8EC}} \color[HTML]{000000} 0.056 \\
 \midrule
\multirow[c]{18}{*}{\thead{With\\ Colour}} & \multirow[c]{9}{*}{\thead{Without\\ Guidelines}} & Falcon 7B & {\cellcolor[HTML]{DBE9F6}} \color[HTML]{000000} 0.036 & {\cellcolor[HTML]{D1E2F3}} \color[HTML]{000000} 0.043 & {\cellcolor[HTML]{D0E1F2}} \color[HTML]{000000} 0.044 & {\cellcolor[HTML]{BAD6EB}} \color[HTML]{000000} 0.057 \\
 &  & Falcon 40B Instruct & {\cellcolor[HTML]{BCD7EB}} \color[HTML]{000000} 0.057 & {\cellcolor[HTML]{A1CBE2}} \color[HTML]{000000} 0.069 & {\cellcolor[HTML]{8CC0DD}} \color[HTML]{000000} 0.077 & {\cellcolor[HTML]{68ACD5}} \color[HTML]{F1F1F1} 0.09 \\
 &  & Gpt 3 & {\cellcolor[HTML]{4B98CA}} \color[HTML]{F1F1F1} 0.1 & {\cellcolor[HTML]{4997C9}} \color[HTML]{F1F1F1} 0.1 & {\cellcolor[HTML]{084285}} \color[HTML]{F1F1F1} 0.15 & {\cellcolor[HTML]{084F99}} \color[HTML]{F1F1F1} 0.15 \\
 &  & Gpt 4 & {\cellcolor[HTML]{084285}} \color[HTML]{F1F1F1} 0.15 & {\cellcolor[HTML]{2B7BBA}} \color[HTML]{F1F1F1} 0.12 & {\cellcolor[HTML]{206FB4}} \color[HTML]{F1F1F1} 0.13 & {\cellcolor[HTML]{2F7FBC}} \color[HTML]{F1F1F1} 0.12 \\
 &  & Mpt 7B & {\cellcolor[HTML]{DFECF7}} \color[HTML]{000000} 0.032 & {\cellcolor[HTML]{E5EFF9}} \color[HTML]{000000} 0.028 & {\cellcolor[HTML]{D7E6F5}} \color[HTML]{000000} 0.038 & {\cellcolor[HTML]{D9E8F5}} \color[HTML]{000000} 0.037 \\
 &  & Mpt 30B Chat & {\cellcolor[HTML]{AACFE5}} \color[HTML]{000000} 0.065 & {\cellcolor[HTML]{CDE0F1}} \color[HTML]{000000} 0.046 & {\cellcolor[HTML]{BED8EC}} \color[HTML]{000000} 0.056 & {\cellcolor[HTML]{C3DAEE}} \color[HTML]{000000} 0.053 \\
 &  & Vicuna 13B V1 & {\cellcolor[HTML]{94C4DF}} \color[HTML]{000000} 0.074 & {\cellcolor[HTML]{A3CCE3}} \color[HTML]{000000} 0.069 & {\cellcolor[HTML]{ABD0E6}} \color[HTML]{000000} 0.064 & {\cellcolor[HTML]{8ABFDD}} \color[HTML]{000000} 0.078 \\
 &  & Vicuna 33B V1 & {\cellcolor[HTML]{9FCAE1}} \color[HTML]{000000} 0.07 & {\cellcolor[HTML]{B0D2E7}} \color[HTML]{000000} 0.062 & {\cellcolor[HTML]{C2D9EE}} \color[HTML]{000000} 0.054 & {\cellcolor[HTML]{9CC9E1}} \color[HTML]{000000} 0.071 \\
 &  & Wizard Vicuna 13B Uncensored HF & {\cellcolor[HTML]{56A0CE}} \color[HTML]{F1F1F1} 0.099 & {\cellcolor[HTML]{C8DCF0}} \color[HTML]{000000} 0.05 & {\cellcolor[HTML]{AED1E7}} \color[HTML]{000000} 0.063 & {\cellcolor[HTML]{9CC9E1}} \color[HTML]{000000} 0.071 \\
 \cmidrule{2-7}
 & \multirow[c]{9}{*}{\thead{With \\Guidelines}} & Falcon 7B & {\cellcolor[HTML]{C1D9ED}} \color[HTML]{000000} 0.054 & {\cellcolor[HTML]{D6E5F4}} \color[HTML]{000000} 0.04 & {\cellcolor[HTML]{C9DDF0}} \color[HTML]{000000} 0.05 & {\cellcolor[HTML]{D2E3F3}} \color[HTML]{000000} 0.043 \\
 &  & Falcon 40B Instruct & {\cellcolor[HTML]{D2E3F3}} \color[HTML]{000000} 0.043 & {\cellcolor[HTML]{ABD0E6}} \color[HTML]{000000} 0.064 & {\cellcolor[HTML]{A5CDE3}} \color[HTML]{000000} 0.067 & {\cellcolor[HTML]{C1D9ED}} \color[HTML]{000000} 0.054 \\
 &  & Gpt 3 & {\cellcolor[HTML]{083471}} \color[HTML]{F1F1F1} 0.16 & {\cellcolor[HTML]{08306B}} \color[HTML]{F1F1F1} 0.16 & {\cellcolor[HTML]{08478D}} \color[HTML]{F1F1F1} 0.15 & {\cellcolor[HTML]{083776}} \color[HTML]{F1F1F1} 0.16 \\
 &  & Gpt 4 & {\cellcolor[HTML]{4594C7}} \color[HTML]{F1F1F1} 0.11 & {\cellcolor[HTML]{3989C1}} \color[HTML]{F1F1F1} 0.11 & {\cellcolor[HTML]{2474B7}} \color[HTML]{F1F1F1} 0.12 & {\cellcolor[HTML]{2676B8}} \color[HTML]{F1F1F1} 0.12 \\
 &  & Mpt 7B & {\cellcolor[HTML]{DCEAF6}} \color[HTML]{000000} 0.034 & {\cellcolor[HTML]{E3EEF9}} \color[HTML]{000000} 0.029 & {\cellcolor[HTML]{000000}} \color[HTML]{F1F1F1} nan & {\cellcolor[HTML]{000000}} \color[HTML]{F1F1F1} nan \\
 &  & Mpt 30B Chat & {\cellcolor[HTML]{D4E4F4}} \color[HTML]{000000} 0.041 & {\cellcolor[HTML]{D6E5F4}} \color[HTML]{000000} 0.04 & {\cellcolor[HTML]{000000}} \color[HTML]{F1F1F1} nan & {\cellcolor[HTML]{000000}} \color[HTML]{F1F1F1} nan \\
 &  & Vicuna 13B V1 & {\cellcolor[HTML]{C1D9ED}} \color[HTML]{000000} 0.055 & {\cellcolor[HTML]{BDD7EC}} \color[HTML]{000000} 0.056 & {\cellcolor[HTML]{C9DDF0}} \color[HTML]{000000} 0.05 & {\cellcolor[HTML]{B0D2E7}} \color[HTML]{000000} 0.062 \\
 &  & Vicuna 33B V1 & {\cellcolor[HTML]{DDEAF7}} \color[HTML]{000000} 0.034 & {\cellcolor[HTML]{E7F0FA}} \color[HTML]{000000} 0.027 & {\cellcolor[HTML]{EEF5FC}} \color[HTML]{000000} 0.022 & {\cellcolor[HTML]{F7FBFF}} \color[HTML]{000000} 0.014 \\
 &  & Wizard Vicuna 13B Uncensored HF & {\cellcolor[HTML]{CADEF0}} \color[HTML]{000000} 0.049 & {\cellcolor[HTML]{E6F0F9}} \color[HTML]{000000} 0.028 & {\cellcolor[HTML]{CCDFF1}} \color[HTML]{000000} 0.048 & {\cellcolor[HTML]{DFEBF7}} \color[HTML]{000000} 0.033 \\

\bottomrule
\end{tabular}
            
\end{table}

\begin{table}[!htbp]
    \centering
    \caption{METEOR scores across the tested models for all experiments. On the left is reported the experimental setup, with/without guidelines/colour information for each of the models that were tested. On the right are shown the BLEU 1 scores for a different number of examples (shots) given in the prompt. %OpenAI models score consistently higher than the rest. Observe that increasing the number of shots on average does yield better results.}
    \label{tab:personal-narrative-elicitation-meteor}
\begin{tabular}{l|l|l|rrrr}
\toprule
\multicolumn{7}{c}{\thead{METEOR}}\\
\midrule
\thead{Colour} & \thead{Guidelines} & \thead{Model name} & \thead{0-shot} & \thead{1-shot} & \thead{3-shot} & \thead{5-shot} \\
\midrule
\multirow[c]{18}{*}{\thead{Without\\ Colour}} & \multirow[c]{9}{*}{\thead{Without\\ Guidelines}} &Falcon 7B & {\cellcolor[HTML]{FCB99F}} \color[HTML]{000000} 0.097 & {\cellcolor[HTML]{FCB89E}} \color[HTML]{000000} 0.097 & {\cellcolor[HTML]{FC8D6D}} \color[HTML]{F1F1F1} 0.13 & {\cellcolor[HTML]{FCA486}} \color[HTML]{000000} 0.11 \\
 &  & Falcon 40B Instruct & {\cellcolor[HTML]{FCAB8F}} \color[HTML]{000000} 0.11 & {\cellcolor[HTML]{FC9474}} \color[HTML]{000000} 0.13 & {\cellcolor[HTML]{F96044}} \color[HTML]{F1F1F1} 0.17 & {\cellcolor[HTML]{F44D38}} \color[HTML]{F1F1F1} 0.18 \\
 &  & ChatGPT 3.5 turbo & {\cellcolor[HTML]{FC7F5F}} \color[HTML]{F1F1F1} 0.14 & {\cellcolor[HTML]{F96245}} \color[HTML]{F1F1F1} 0.17 & {\cellcolor[HTML]{C9181D}} \color[HTML]{F1F1F1} 0.23 & {\cellcolor[HTML]{BC141A}} \color[HTML]{F1F1F1} 0.24 \\
 &  & ChatGPT 4 & {\cellcolor[HTML]{F24734}} \color[HTML]{F1F1F1} 0.18 & {\cellcolor[HTML]{F44F39}} \color[HTML]{F1F1F1} 0.18 & {\cellcolor[HTML]{B81419}} \color[HTML]{F1F1F1} 0.24 & {\cellcolor[HTML]{B31218}} \color[HTML]{F1F1F1} 0.24 \\
 &  & Mpt 7B & {\cellcolor[HTML]{FDCBB6}} \color[HTML]{000000} 0.081 & {\cellcolor[HTML]{FCA98C}} \color[HTML]{000000} 0.11 & {\cellcolor[HTML]{FCA082}} \color[HTML]{000000} 0.12 & {\cellcolor[HTML]{FC9D7F}} \color[HTML]{000000} 0.12 \\
 &  & Mpt 30B Chat & {\cellcolor[HTML]{FC8161}} \color[HTML]{F1F1F1} 0.14 & {\cellcolor[HTML]{FCAF93}} \color[HTML]{000000} 0.1 & {\cellcolor[HTML]{FCA588}} \color[HTML]{000000} 0.11 & {\cellcolor[HTML]{FCAE92}} \color[HTML]{000000} 0.11 \\
 &  & Vicuna 13B V1 & {\cellcolor[HTML]{FC8B6B}} \color[HTML]{F1F1F1} 0.13 & {\cellcolor[HTML]{FC9879}} \color[HTML]{000000} 0.12 & {\cellcolor[HTML]{FC8262}} \color[HTML]{F1F1F1} 0.14 & {\cellcolor[HTML]{FC9C7D}} \color[HTML]{000000} 0.12 \\
 &  & Vicuna 33B V1 & {\cellcolor[HTML]{FCBCA2}} \color[HTML]{000000} 0.094 & {\cellcolor[HTML]{FCC4AD}} \color[HTML]{000000} 0.087 & {\cellcolor[HTML]{FC8B6B}} \color[HTML]{F1F1F1} 0.13 & {\cellcolor[HTML]{FC9879}} \color[HTML]{000000} 0.12 \\
 &  & Wizard Vicuna 13B Uncensored HF & {\cellcolor[HTML]{FCA689}} \color[HTML]{000000} 0.11 & {\cellcolor[HTML]{FCAA8D}} \color[HTML]{000000} 0.11 & {\cellcolor[HTML]{FC9E80}} \color[HTML]{000000} 0.12 & {\cellcolor[HTML]{FC9474}} \color[HTML]{000000} 0.13 \\
 \cmidrule{2-7}
 & \multirow[c]{9}{*}{\thead{With\\ Guidelines}}& Falcon 7B & {\cellcolor[HTML]{FCB296}} \color[HTML]{000000} 0.1 & {\cellcolor[HTML]{FCAA8D}} \color[HTML]{000000} 0.11 & {\cellcolor[HTML]{FC8A6A}} \color[HTML]{F1F1F1} 0.13 & {\cellcolor[HTML]{FCAA8D}} \color[HTML]{000000} 0.11 \\
 &  & Falcon 40B Instruct & {\cellcolor[HTML]{FDC9B3}} \color[HTML]{000000} 0.084 & {\cellcolor[HTML]{FC8969}} \color[HTML]{F1F1F1} 0.14 & {\cellcolor[HTML]{F96346}} \color[HTML]{F1F1F1} 0.16 & {\cellcolor[HTML]{FC9070}} \color[HTML]{000000} 0.13 \\
 &  & ChatGPT 3.5 turbo & {\cellcolor[HTML]{BF151B}} \color[HTML]{F1F1F1} 0.23 & {\cellcolor[HTML]{9C0D14}} \color[HTML]{F1F1F1} 0.26 & {\cellcolor[HTML]{960B13}} \color[HTML]{F1F1F1} 0.26 & {\cellcolor[HTML]{79040F}} \color[HTML]{F1F1F1} 0.28 \\
 &  & ChatGPT 4 & {\cellcolor[HTML]{CB181D}} \color[HTML]{F1F1F1} 0.22 & {\cellcolor[HTML]{EC382B}} \color[HTML]{F1F1F1} 0.2 & {\cellcolor[HTML]{B31218}} \color[HTML]{F1F1F1} 0.24 & {\cellcolor[HTML]{AA1016}} \color[HTML]{F1F1F1} 0.25 \\
 &  & Mpt 7B & {\cellcolor[HTML]{FCBDA4}} \color[HTML]{000000} 0.093 & {\cellcolor[HTML]{FCC2AA}} \color[HTML]{000000} 0.09 & {\cellcolor[HTML]{000000}} \color[HTML]{F1F1F1} nan & {\cellcolor[HTML]{000000}} \color[HTML]{F1F1F1} nan \\
 &  & Mpt 30B Chat & {\cellcolor[HTML]{FCAA8D}} \color[HTML]{000000} 0.11 & {\cellcolor[HTML]{FC9474}} \color[HTML]{000000} 0.13 & {\cellcolor[HTML]{FC9373}} \color[HTML]{000000} 0.13 & {\cellcolor[HTML]{FCA183}} \color[HTML]{000000} 0.12 \\
 &  & Vicuna 13B V1 & {\cellcolor[HTML]{FCA486}} \color[HTML]{000000} 0.11 & {\cellcolor[HTML]{FCBEA5}} \color[HTML]{000000} 0.093 & {\cellcolor[HTML]{FC9272}} \color[HTML]{000000} 0.13 & {\cellcolor[HTML]{FC8F6F}} \color[HTML]{000000} 0.13 \\
 &  & Vicuna 33B V1 & {\cellcolor[HTML]{FDC7B2}} \color[HTML]{000000} 0.085 & {\cellcolor[HTML]{FCAA8D}} \color[HTML]{000000} 0.11 & {\cellcolor[HTML]{FC8D6D}} \color[HTML]{F1F1F1} 0.13 & {\cellcolor[HTML]{FC9C7D}} \color[HTML]{000000} 0.12 \\
 &  & Wizard Vicuna 13B Uncensored HF & {\cellcolor[HTML]{FEE6DA}} \color[HTML]{000000} 0.054 & {\cellcolor[HTML]{FCC4AD}} \color[HTML]{000000} 0.088 & {\cellcolor[HTML]{FC9272}} \color[HTML]{000000} 0.13 & {\cellcolor[HTML]{FC9D7F}} \color[HTML]{000000} 0.12 \\
 \midrule
\multirow[c]{18}{*}{\thead{With\\ Colour}} & \multirow[c]{9}{*}{\thead{Without\\ Guidelines}}& Falcon 7B & {\cellcolor[HTML]{FCBEA5}} \color[HTML]{000000} 0.093 & {\cellcolor[HTML]{FCBBA1}} \color[HTML]{000000} 0.096 & {\cellcolor[HTML]{FCAF93}} \color[HTML]{000000} 0.1 & {\cellcolor[HTML]{FC9576}} \color[HTML]{000000} 0.12 \\
 &  & Falcon 40B Instruct & {\cellcolor[HTML]{FCAB8F}} \color[HTML]{000000} 0.11 & {\cellcolor[HTML]{FCA285}} \color[HTML]{000000} 0.11 & {\cellcolor[HTML]{FB7D5D}} \color[HTML]{F1F1F1} 0.14 & {\cellcolor[HTML]{FA6648}} \color[HTML]{F1F1F1} 0.16 \\
 &  & ChatGPT 3.5 turbo & {\cellcolor[HTML]{FB7B5B}} \color[HTML]{F1F1F1} 0.15 & {\cellcolor[HTML]{FA6547}} \color[HTML]{F1F1F1} 0.16 & {\cellcolor[HTML]{79040F}} \color[HTML]{F1F1F1} 0.28 & {\cellcolor[HTML]{6F020E}} \color[HTML]{F1F1F1} 0.28 \\
 &  & ChatGPT 4 & {\cellcolor[HTML]{D82422}} \color[HTML]{F1F1F1} 0.21 & {\cellcolor[HTML]{E43027}} \color[HTML]{F1F1F1} 0.2 & {\cellcolor[HTML]{AB1016}} \color[HTML]{F1F1F1} 0.25 & {\cellcolor[HTML]{BB141A}} \color[HTML]{F1F1F1} 0.24 \\
 &  & Mpt 7B & {\cellcolor[HTML]{FDCCB8}} \color[HTML]{000000} 0.081 & {\cellcolor[HTML]{FDD7C6}} \color[HTML]{000000} 0.071 & {\cellcolor[HTML]{FDCBB6}} \color[HTML]{000000} 0.081 & {\cellcolor[HTML]{FDCCB8}} \color[HTML]{000000} 0.08 \\
 &  & Mpt 30B Chat & {\cellcolor[HTML]{FCB69B}} \color[HTML]{000000} 0.1 & {\cellcolor[HTML]{FCB69B}} \color[HTML]{000000} 0.1 & {\cellcolor[HTML]{FCC1A8}} \color[HTML]{000000} 0.091 & {\cellcolor[HTML]{FCA082}} \color[HTML]{000000} 0.12 \\
 &  & Vicuna 13B V1 & {\cellcolor[HTML]{FCA486}} \color[HTML]{000000} 0.11 & {\cellcolor[HTML]{FCA689}} \color[HTML]{000000} 0.11 & {\cellcolor[HTML]{FC997A}} \color[HTML]{000000} 0.12 & {\cellcolor[HTML]{FCA183}} \color[HTML]{000000} 0.12 \\
 &  & Vicuna 33B V1 & {\cellcolor[HTML]{FCB69B}} \color[HTML]{000000} 0.1 & {\cellcolor[HTML]{FCBBA1}} \color[HTML]{000000} 0.096 & {\cellcolor[HTML]{FDC6B0}} \color[HTML]{000000} 0.085 & {\cellcolor[HTML]{FCA98C}} \color[HTML]{000000} 0.11 \\
 &  & Wizard Vicuna 13B Uncensored HF & {\cellcolor[HTML]{FC8565}} \color[HTML]{F1F1F1} 0.14 & {\cellcolor[HTML]{FCB499}} \color[HTML]{000000} 0.1 & {\cellcolor[HTML]{FCA98C}} \color[HTML]{000000} 0.11 & {\cellcolor[HTML]{FC9879}} \color[HTML]{000000} 0.12 \\
 \cmidrule{2-7}
 & \multirow[c]{9}{*}{\thead{With\\ Guidelines}}& Falcon 7B & {\cellcolor[HTML]{FCAF93}} \color[HTML]{000000} 0.11 & {\cellcolor[HTML]{FCA689}} \color[HTML]{000000} 0.11 & {\cellcolor[HTML]{FCBEA5}} \color[HTML]{000000} 0.093 & {\cellcolor[HTML]{FCBDA4}} \color[HTML]{000000} 0.094 \\
 &  & Falcon 40B Instruct & {\cellcolor[HTML]{FCB89E}} \color[HTML]{000000} 0.098 & {\cellcolor[HTML]{FC8969}} \color[HTML]{F1F1F1} 0.14 & {\cellcolor[HTML]{FC9E80}} \color[HTML]{000000} 0.12 & {\cellcolor[HTML]{FC9C7D}} \color[HTML]{000000} 0.12 \\
 &  & ChatGPT 3.5 turbo & {\cellcolor[HTML]{8A0812}} \color[HTML]{F1F1F1} 0.27 & {\cellcolor[HTML]{71020E}} \color[HTML]{F1F1F1} 0.28 & {\cellcolor[HTML]{7E0610}} \color[HTML]{F1F1F1} 0.28 & {\cellcolor[HTML]{67000D}} \color[HTML]{F1F1F1} 0.29 \\
 &  & ChatGPT 4 & {\cellcolor[HTML]{E22E27}} \color[HTML]{F1F1F1} 0.2 & {\cellcolor[HTML]{E02C26}} \color[HTML]{F1F1F1} 0.21 & {\cellcolor[HTML]{AC1117}} \color[HTML]{F1F1F1} 0.25 & {\cellcolor[HTML]{A10E15}} \color[HTML]{F1F1F1} 0.26 \\
 &  & Mpt 7B & {\cellcolor[HTML]{FCC2AA}} \color[HTML]{000000} 0.09 & {\cellcolor[HTML]{FEE1D3}} \color[HTML]{000000} 0.063 & {\cellcolor[HTML]{000000}} \color[HTML]{F1F1F1} nan & {\cellcolor[HTML]{000000}} \color[HTML]{F1F1F1} nan \\
 &  & Mpt 30B Chat & {\cellcolor[HTML]{FCBBA1}} \color[HTML]{000000} 0.096 & {\cellcolor[HTML]{FDD5C4}} \color[HTML]{000000} 0.072 & {\cellcolor[HTML]{000000}} \color[HTML]{F1F1F1} nan & {\cellcolor[HTML]{000000}} \color[HTML]{F1F1F1} nan \\
 &  & Vicuna 13B V1 & {\cellcolor[HTML]{FCAA8D}} \color[HTML]{000000} 0.11 & {\cellcolor[HTML]{FCA183}} \color[HTML]{000000} 0.12 & {\cellcolor[HTML]{FCC4AD}} \color[HTML]{000000} 0.088 & {\cellcolor[HTML]{FCB99F}} \color[HTML]{000000} 0.097 \\
 &  & Vicuna 33B V1 & {\cellcolor[HTML]{FEE1D4}} \color[HTML]{000000} 0.062 & {\cellcolor[HTML]{FEE9DF}} \color[HTML]{000000} 0.049 & {\cellcolor[HTML]{FFF0E8}} \color[HTML]{000000} 0.04 & {\cellcolor[HTML]{FFF5F0}} \color[HTML]{000000} 0.031 \\
 &  & Wizard Vicuna 13B Uncensored HF & {\cellcolor[HTML]{FDC9B3}} \color[HTML]{000000} 0.084 & {\cellcolor[HTML]{FEE5D8}} \color[HTML]{000000} 0.056 & {\cellcolor[HTML]{FCAA8D}} \color[HTML]{000000} 0.11 & {\cellcolor[HTML]{FDCBB6}} \color[HTML]{000000} 0.081 \\
\bottomrule
\end{tabular}
            
\end{table}

\begin{table}
    \centering
    \caption{ROUGE 1 scores across the tested models for all experiments. OpenAI models score consistently higher than the rest. Observe that increasing the number of shots on average does yield better results.}
    \label{tab:personal-narrative-elicitation-rouge}
\setlength{\tabcolsep}{3pt}
\begin{tabular}{l|l|l|rrrr|rrrr|rrrr}
\toprule
\multicolumn{15}{c}{\thead{ROUGE}}\\
\midrule
\multirow{2}{*}{\rotatebox[origin=l]{270}{\thead{Colour}}} & \multirow{2}{*}{\rotatebox[origin=l]{270}{\thead{Guidelines}}} & \multirow{2}{*}{\rotatebox[origin=l]{270}{\thead{Model name}}} & \multicolumn{4}{c|}{\rotatebox[origin=l]{270}{\thead{F1}}} & \multicolumn{4}{c|}{\thead{\rotatebox[origin=l]{270}{Recall}}} & \multicolumn{4}{c}{\thead{\rotatebox[origin=l]{270}{Precision}}} \\
% \cmidrule{4-15}
&&&\thead{\rotatebox[origin=l]{270}{0-shot}} & \thead{\rotatebox[origin=l]{270}{1-shot}} & \thead{\rotatebox[origin=l]{270}{3-shot}} & \thead{\rotatebox[origin=l]{270}{5-shot}} & \thead{\rotatebox[origin=l]{270}{0-shot}} & \thead{\rotatebox[origin=l]{270}{1-shot}} & \thead{\rotatebox[origin=l]{270}{3-shot}} & \thead{\rotatebox[origin=l]{270}{5-shot}}&
\thead{\rotatebox[origin=l]{270}{0-shot}} & \thead{\rotatebox[origin=l]{270}{1-shot}} & \thead{\rotatebox[origin=l]{270}{3-shot}} & \thead{\rotatebox[origin=l]{270}{5-shot}}   \\
\midrule
\multirow[c]{18}{*}{\rotatebox[origin=l]{270}{\thead{Without Colour}}} & \multirow[c]{9}{*}{\rotatebox[origin=l]{270}{\thead{Without Guidelines}}} & F. 7B & {\cellcolor[HTML]{CEECC8}} \color[HTML]{000000} 0.067 & {\cellcolor[HTML]{D5EFCF}} \color[HTML]{000000} 0.062 & {\cellcolor[HTML]{9BD696}} \color[HTML]{000000} 0.1 & {\cellcolor[HTML]{BAE3B3}} \color[HTML]{000000} 0.082 & {\cellcolor[HTML]{FDAB66}} \color[HTML]{000000} 0.15 & {\cellcolor[HTML]{FDAF6C}} \color[HTML]{000000} 0.14 & {\cellcolor[HTML]{FD9040}} \color[HTML]{000000} 0.18 & {\cellcolor[HTML]{FDB373}} \color[HTML]{000000} 0.14 & {\cellcolor[HTML]{D4D4E8}} \color[HTML]{000000} 0.069 & {\cellcolor[HTML]{DCDCEC}} \color[HTML]{000000} 0.063 & {\cellcolor[HTML]{A4A1CC}} \color[HTML]{F1F1F1} 0.1 & {\cellcolor[HTML]{C1C2DF}} \color[HTML]{000000} 0.083 \\
 &  & F. 40B  & {\cellcolor[HTML]{BDE5B6}} \color[HTML]{000000} 0.08 & {\cellcolor[HTML]{A2D99C}} \color[HTML]{000000} 0.098 & {\cellcolor[HTML]{81CA81}} \color[HTML]{000000} 0.12 & {\cellcolor[HTML]{79C67A}} \color[HTML]{000000} 0.12 & {\cellcolor[HTML]{FEDCB9}} \color[HTML]{000000} 0.089 & {\cellcolor[HTML]{FDC088}} \color[HTML]{000000} 0.13 & {\cellcolor[HTML]{FD9243}} \color[HTML]{000000} 0.17 & {\cellcolor[HTML]{FD9C51}} \color[HTML]{000000} 0.16 & {\cellcolor[HTML]{C0C1DE}} \color[HTML]{000000} 0.084 & {\cellcolor[HTML]{B2B2D5}} \color[HTML]{000000} 0.094 & {\cellcolor[HTML]{AAA8D0}} \color[HTML]{000000} 0.099 & {\cellcolor[HTML]{8F8CC1}} \color[HTML]{F1F1F1} 0.12 \\
 &  & Gpt 3 & {\cellcolor[HTML]{76C578}} \color[HTML]{000000} 0.12 & {\cellcolor[HTML]{40AA5D}} \color[HTML]{F1F1F1} 0.15 & {\cellcolor[HTML]{0B7734}} \color[HTML]{F1F1F1} 0.19 & {\cellcolor[HTML]{0A7633}} \color[HTML]{F1F1F1} 0.19 & {\cellcolor[HTML]{FDB475}} \color[HTML]{000000} 0.14 & {\cellcolor[HTML]{FD9C51}} \color[HTML]{000000} 0.16 & {\cellcolor[HTML]{C84202}} \color[HTML]{F1F1F1} 0.26 & {\cellcolor[HTML]{9C3203}} \color[HTML]{F1F1F1} 0.29 & {\cellcolor[HTML]{8784BD}} \color[HTML]{F1F1F1} 0.12 & {\cellcolor[HTML]{6950A3}} \color[HTML]{F1F1F1} 0.15 & {\cellcolor[HTML]{582F93}} \color[HTML]{F1F1F1} 0.17 & {\cellcolor[HTML]{603E9A}} \color[HTML]{F1F1F1} 0.16 \\
 &  & Gpt 4 & {\cellcolor[HTML]{319A50}} \color[HTML]{F1F1F1} 0.16 & {\cellcolor[HTML]{3FA95C}} \color[HTML]{F1F1F1} 0.15 & {\cellcolor[HTML]{077331}} \color[HTML]{F1F1F1} 0.19 & {\cellcolor[HTML]{03702E}} \color[HTML]{F1F1F1} 0.19 & {\cellcolor[HTML]{F16913}} \color[HTML]{F1F1F1} 0.21 & {\cellcolor[HTML]{F87F2C}} \color[HTML]{F1F1F1} 0.19 & {\cellcolor[HTML]{A23503}} \color[HTML]{F1F1F1} 0.29 & {\cellcolor[HTML]{BE3F02}} \color[HTML]{F1F1F1} 0.27 & {\cellcolor[HTML]{7262AC}} \color[HTML]{F1F1F1} 0.14 & {\cellcolor[HTML]{776AB0}} \color[HTML]{F1F1F1} 0.14 & {\cellcolor[HTML]{65479E}} \color[HTML]{F1F1F1} 0.15 & {\cellcolor[HTML]{5C3696}} \color[HTML]{F1F1F1} 0.16 \\
 &  & Mpt 7B & {\cellcolor[HTML]{D6EFD0}} \color[HTML]{000000} 0.061 & {\cellcolor[HTML]{AADDA4}} \color[HTML]{000000} 0.092 & {\cellcolor[HTML]{B7E2B1}} \color[HTML]{000000} 0.084 & {\cellcolor[HTML]{C0E6B9}} \color[HTML]{000000} 0.078 & {\cellcolor[HTML]{FDD9B4}} \color[HTML]{000000} 0.094 & {\cellcolor[HTML]{FDBD83}} \color[HTML]{000000} 0.13 & {\cellcolor[HTML]{FDD7B1}} \color[HTML]{000000} 0.096 & {\cellcolor[HTML]{FDB475}} \color[HTML]{000000} 0.14 & {\cellcolor[HTML]{DBDBEC}} \color[HTML]{000000} 0.065 & {\cellcolor[HTML]{9E9AC8}} \color[HTML]{F1F1F1} 0.11 & {\cellcolor[HTML]{BDBEDC}} \color[HTML]{000000} 0.086 & {\cellcolor[HTML]{C4C5E0}} \color[HTML]{000000} 0.081 \\
 &  & Mpt 30B & {\cellcolor[HTML]{72C375}} \color[HTML]{000000} 0.12 & {\cellcolor[HTML]{A5DB9F}} \color[HTML]{000000} 0.095 & {\cellcolor[HTML]{97D492}} \color[HTML]{000000} 0.1 & {\cellcolor[HTML]{C1E6BA}} \color[HTML]{000000} 0.077 & {\cellcolor[HTML]{FDAE6A}} \color[HTML]{000000} 0.14 & {\cellcolor[HTML]{FDD1A4}} \color[HTML]{000000} 0.11 & {\cellcolor[HTML]{FDCA99}} \color[HTML]{000000} 0.11 & {\cellcolor[HTML]{FDD1A4}} \color[HTML]{000000} 0.11 & {\cellcolor[HTML]{7669AF}} \color[HTML]{F1F1F1} 0.14 & {\cellcolor[HTML]{8D89C0}} \color[HTML]{F1F1F1} 0.12 & {\cellcolor[HTML]{9390C3}} \color[HTML]{F1F1F1} 0.12 & {\cellcolor[HTML]{BFC0DE}} \color[HTML]{000000} 0.084 \\
 &  & V. 13B  & {\cellcolor[HTML]{6DC072}} \color[HTML]{000000} 0.13 & {\cellcolor[HTML]{8ACE88}} \color[HTML]{000000} 0.11 & {\cellcolor[HTML]{90D18D}} \color[HTML]{000000} 0.11 & {\cellcolor[HTML]{9CD797}} \color[HTML]{000000} 0.1 & {\cellcolor[HTML]{FDB06E}} \color[HTML]{000000} 0.14 & {\cellcolor[HTML]{FDC088}} \color[HTML]{000000} 0.13 & {\cellcolor[HTML]{FDB87C}} \color[HTML]{000000} 0.13 & {\cellcolor[HTML]{FDC088}} \color[HTML]{000000} 0.13 & {\cellcolor[HTML]{7970B3}} \color[HTML]{F1F1F1} 0.14 & {\cellcolor[HTML]{8481BC}} \color[HTML]{F1F1F1} 0.13 & {\cellcolor[HTML]{A09DCA}} \color[HTML]{F1F1F1} 0.11 & {\cellcolor[HTML]{A19ECA}} \color[HTML]{F1F1F1} 0.11 \\
 &  & V. 33B  & {\cellcolor[HTML]{C3E7BC}} \color[HTML]{000000} 0.076 & {\cellcolor[HTML]{CBEAC4}} \color[HTML]{000000} 0.071 & {\cellcolor[HTML]{87CD86}} \color[HTML]{000000} 0.11 & {\cellcolor[HTML]{9CD797}} \color[HTML]{000000} 0.1 & {\cellcolor[HTML]{FEE1C4}} \color[HTML]{000000} 0.081 & {\cellcolor[HTML]{FEE5CC}} \color[HTML]{000000} 0.074 & {\cellcolor[HTML]{FDB77A}} \color[HTML]{000000} 0.13 & {\cellcolor[HTML]{FDD0A2}} \color[HTML]{000000} 0.11 & {\cellcolor[HTML]{C3C4E0}} \color[HTML]{000000} 0.082 & {\cellcolor[HTML]{BEBFDD}} \color[HTML]{000000} 0.085 & {\cellcolor[HTML]{8986BE}} \color[HTML]{F1F1F1} 0.12 & {\cellcolor[HTML]{9390C3}} \color[HTML]{F1F1F1} 0.12 \\
 &  & W. V. 13B  & {\cellcolor[HTML]{AEDEA7}} \color[HTML]{000000} 0.09 & {\cellcolor[HTML]{B2E0AC}} \color[HTML]{000000} 0.087 & {\cellcolor[HTML]{A2D99C}} \color[HTML]{000000} 0.098 & {\cellcolor[HTML]{A2D99C}} \color[HTML]{000000} 0.098 & {\cellcolor[HTML]{FDD5AD}} \color[HTML]{000000} 0.099 & {\cellcolor[HTML]{FDD2A6}} \color[HTML]{000000} 0.11 & {\cellcolor[HTML]{FDC088}} \color[HTML]{000000} 0.13 & {\cellcolor[HTML]{FDC794}} \color[HTML]{000000} 0.12 & {\cellcolor[HTML]{AEACD2}} \color[HTML]{000000} 0.097 & {\cellcolor[HTML]{ACAAD1}} \color[HTML]{000000} 0.098 & {\cellcolor[HTML]{B4B4D7}} \color[HTML]{000000} 0.092 & {\cellcolor[HTML]{ABA9D0}} \color[HTML]{000000} 0.099 \\
 \cmidrule{2-15}
 & \multirow[c]{9}{*}{\rotatebox[origin=lenter]{270}{\thead{With Guidelines}}}& F. 7B & {\cellcolor[HTML]{B4E1AD}} \color[HTML]{000000} 0.086 & {\cellcolor[HTML]{BAE3B3}} \color[HTML]{000000} 0.082 & {\cellcolor[HTML]{A9DCA3}} \color[HTML]{000000} 0.093 & {\cellcolor[HTML]{ABDDA5}} \color[HTML]{000000} 0.091 & {\cellcolor[HTML]{FDCFA0}} \color[HTML]{000000} 0.11 & {\cellcolor[HTML]{FDBB81}} \color[HTML]{000000} 0.13 & {\cellcolor[HTML]{FDAE6A}} \color[HTML]{000000} 0.14 & {\cellcolor[HTML]{FDB271}} \color[HTML]{000000} 0.14 & {\cellcolor[HTML]{8D89C0}} \color[HTML]{F1F1F1} 0.12 & {\cellcolor[HTML]{C6C7E1}} \color[HTML]{000000} 0.08 & {\cellcolor[HTML]{ADABD2}} \color[HTML]{000000} 0.097 & {\cellcolor[HTML]{AEACD2}} \color[HTML]{000000} 0.097 \\
 &  & F. 40B  & {\cellcolor[HTML]{DDF2D8}} \color[HTML]{000000} 0.055 & {\cellcolor[HTML]{A3DA9D}} \color[HTML]{000000} 0.097 & {\cellcolor[HTML]{76C578}} \color[HTML]{000000} 0.12 & {\cellcolor[HTML]{ABDDA5}} \color[HTML]{000000} 0.091 & {\cellcolor[HTML]{FDD7AF}} \color[HTML]{000000} 0.097 & {\cellcolor[HTML]{FDB373}} \color[HTML]{000000} 0.14 & {\cellcolor[HTML]{FB8735}} \color[HTML]{F1F1F1} 0.19 & {\cellcolor[HTML]{FDC189}} \color[HTML]{000000} 0.12 & {\cellcolor[HTML]{E8E6F2}} \color[HTML]{000000} 0.052 & {\cellcolor[HTML]{A5A2CD}} \color[HTML]{F1F1F1} 0.1 & {\cellcolor[HTML]{9C98C7}} \color[HTML]{F1F1F1} 0.11 & {\cellcolor[HTML]{A19ECA}} \color[HTML]{F1F1F1} 0.11 \\
 &  & Gpt 3 & {\cellcolor[HTML]{05712F}} \color[HTML]{F1F1F1} 0.19 & {\cellcolor[HTML]{005B25}} \color[HTML]{F1F1F1} 0.21 & {\cellcolor[HTML]{005522}} \color[HTML]{F1F1F1} 0.21 & {\cellcolor[HTML]{00441B}} \color[HTML]{F1F1F1} 0.22 & {\cellcolor[HTML]{D54601}} \color[HTML]{F1F1F1} 0.25 & {\cellcolor[HTML]{AE3903}} \color[HTML]{F1F1F1} 0.28 & {\cellcolor[HTML]{8F2D04}} \color[HTML]{F1F1F1} 0.31 & {\cellcolor[HTML]{832804}} \color[HTML]{F1F1F1} 0.32 & {\cellcolor[HTML]{51218C}} \color[HTML]{F1F1F1} 0.17 & {\cellcolor[HTML]{460C83}} \color[HTML]{F1F1F1} 0.19 & {\cellcolor[HTML]{4A1587}} \color[HTML]{F1F1F1} 0.18 & {\cellcolor[HTML]{440981}} \color[HTML]{F1F1F1} 0.19 \\
 &  & Gpt 4 & {\cellcolor[HTML]{29914A}} \color[HTML]{F1F1F1} 0.17 & {\cellcolor[HTML]{46AE60}} \color[HTML]{F1F1F1} 0.15 & {\cellcolor[HTML]{19833E}} \color[HTML]{F1F1F1} 0.18 & {\cellcolor[HTML]{0D7836}} \color[HTML]{F1F1F1} 0.19 & {\cellcolor[HTML]{DB4A02}} \color[HTML]{F1F1F1} 0.25 & {\cellcolor[HTML]{F26D17}} \color[HTML]{F1F1F1} 0.21 & {\cellcolor[HTML]{BB3D02}} \color[HTML]{F1F1F1} 0.27 & {\cellcolor[HTML]{CB4302}} \color[HTML]{F1F1F1} 0.26 & {\cellcolor[HTML]{786DB2}} \color[HTML]{F1F1F1} 0.14 & {\cellcolor[HTML]{8C88BF}} \color[HTML]{F1F1F1} 0.12 & {\cellcolor[HTML]{6E58A7}} \color[HTML]{F1F1F1} 0.15 & {\cellcolor[HTML]{5E3A98}} \color[HTML]{F1F1F1} 0.16 \\
 &  & Mpt 7B & {\cellcolor[HTML]{C4E8BD}} \color[HTML]{000000} 0.075 & {\cellcolor[HTML]{CCEBC6}} \color[HTML]{000000} 0.069 & {\cellcolor[HTML]{000000}} \color[HTML]{F1F1F1} nan & {\cellcolor[HTML]{000000}} \color[HTML]{F1F1F1} nan & {\cellcolor[HTML]{FDDAB6}} \color[HTML]{000000} 0.091 & {\cellcolor[HTML]{FDDAB6}} \color[HTML]{000000} 0.092 & {\cellcolor[HTML]{000000}} \color[HTML]{F1F1F1} nan & {\cellcolor[HTML]{000000}} \color[HTML]{F1F1F1} nan & {\cellcolor[HTML]{BBBBDB}} \color[HTML]{000000} 0.088 & {\cellcolor[HTML]{CBCBE3}} \color[HTML]{000000} 0.076 & {\cellcolor[HTML]{000000}} \color[HTML]{F1F1F1} nan & {\cellcolor[HTML]{000000}} \color[HTML]{F1F1F1} nan \\
 &  & Mpt 30B & {\cellcolor[HTML]{B1E0AB}} \color[HTML]{000000} 0.088 & {\cellcolor[HTML]{A0D99B}} \color[HTML]{000000} 0.098 & {\cellcolor[HTML]{92D28F}} \color[HTML]{000000} 0.11 & {\cellcolor[HTML]{AADDA4}} \color[HTML]{000000} 0.092 & {\cellcolor[HTML]{FDCFA0}} \color[HTML]{000000} 0.11 & {\cellcolor[HTML]{FDD5AB}} \color[HTML]{000000} 0.1 & {\cellcolor[HTML]{FDB678}} \color[HTML]{000000} 0.14 & {\cellcolor[HTML]{FDC088}} \color[HTML]{000000} 0.12 & {\cellcolor[HTML]{B6B6D8}} \color[HTML]{000000} 0.091 & {\cellcolor[HTML]{8683BD}} \color[HTML]{F1F1F1} 0.12 & {\cellcolor[HTML]{8D89C0}} \color[HTML]{F1F1F1} 0.12 & {\cellcolor[HTML]{BBBBDB}} \color[HTML]{000000} 0.088 \\
 &  & V. 13B  & {\cellcolor[HTML]{A7DBA0}} \color[HTML]{000000} 0.095 & {\cellcolor[HTML]{B8E3B2}} \color[HTML]{000000} 0.083 & {\cellcolor[HTML]{9FD899}} \color[HTML]{000000} 0.099 & {\cellcolor[HTML]{8BCF89}} \color[HTML]{000000} 0.11 & {\cellcolor[HTML]{FDC088}} \color[HTML]{000000} 0.13 & {\cellcolor[HTML]{FDCD9C}} \color[HTML]{000000} 0.11 & {\cellcolor[HTML]{FDB576}} \color[HTML]{000000} 0.14 & {\cellcolor[HTML]{FDB97D}} \color[HTML]{000000} 0.13 & {\cellcolor[HTML]{A29FCB}} \color[HTML]{F1F1F1} 0.1 & {\cellcolor[HTML]{BBBBDB}} \color[HTML]{000000} 0.088 & {\cellcolor[HTML]{ACAAD1}} \color[HTML]{000000} 0.098 & {\cellcolor[HTML]{908DC2}} \color[HTML]{F1F1F1} 0.12 \\
 &  & V. 33B  & {\cellcolor[HTML]{BBE4B4}} \color[HTML]{000000} 0.081 & {\cellcolor[HTML]{A4DA9E}} \color[HTML]{000000} 0.096 & {\cellcolor[HTML]{97D492}} \color[HTML]{000000} 0.1 & {\cellcolor[HTML]{B2E0AC}} \color[HTML]{000000} 0.087 & {\cellcolor[HTML]{FDD4AA}} \color[HTML]{000000} 0.1 & {\cellcolor[HTML]{FDC189}} \color[HTML]{000000} 0.12 & {\cellcolor[HTML]{FDC38D}} \color[HTML]{000000} 0.12 & {\cellcolor[HTML]{FDD5AD}} \color[HTML]{000000} 0.099 & {\cellcolor[HTML]{B3B3D6}} \color[HTML]{000000} 0.093 & {\cellcolor[HTML]{A9A7CF}} \color[HTML]{F1F1F1} 0.1 & {\cellcolor[HTML]{A8A6CF}} \color[HTML]{F1F1F1} 0.1 & {\cellcolor[HTML]{B3B3D6}} \color[HTML]{000000} 0.093 \\
 &  & W. V. 13B  & {\cellcolor[HTML]{EBF7E7}} \color[HTML]{000000} 0.04 & {\cellcolor[HTML]{CDECC7}} \color[HTML]{000000} 0.068 & {\cellcolor[HTML]{A4DA9E}} \color[HTML]{000000} 0.096 & {\cellcolor[HTML]{A3DA9D}} \color[HTML]{000000} 0.097 & {\cellcolor[HTML]{FFF2E6}} \color[HTML]{000000} 0.044 & {\cellcolor[HTML]{FEE5CC}} \color[HTML]{000000} 0.073 & {\cellcolor[HTML]{FDB97D}} \color[HTML]{000000} 0.13 & {\cellcolor[HTML]{FDC28B}} \color[HTML]{000000} 0.12 & {\cellcolor[HTML]{EAE8F2}} \color[HTML]{000000} 0.05 & {\cellcolor[HTML]{B6B6D8}} \color[HTML]{000000} 0.091 & {\cellcolor[HTML]{B6B6D8}} \color[HTML]{000000} 0.091 & {\cellcolor[HTML]{9C98C7}} \color[HTML]{F1F1F1} 0.11 \\
\midrule
\multirow[c]{18}{*}{\rotatebox[origin=l]{270}{\thead{With Colour}}} & \multirow[c]{9}{*}{\rotatebox[origin=l]{270}{\thead{Without Guidelines}}} & F. 7B & {\cellcolor[HTML]{D8F0D2}} \color[HTML]{000000} 0.06 & {\cellcolor[HTML]{CCEBC6}} \color[HTML]{000000} 0.068 & {\cellcolor[HTML]{BEE5B8}} \color[HTML]{000000} 0.079 & {\cellcolor[HTML]{AADDA4}} \color[HTML]{000000} 0.092 & {\cellcolor[HTML]{FDA45D}} \color[HTML]{000000} 0.15 & {\cellcolor[HTML]{FDB06E}} \color[HTML]{000000} 0.14 & {\cellcolor[HTML]{FDB576}} \color[HTML]{000000} 0.14 & {\cellcolor[HTML]{FD9446}} \color[HTML]{000000} 0.17 & {\cellcolor[HTML]{E1E0EE}} \color[HTML]{000000} 0.059 & {\cellcolor[HTML]{D2D2E7}} \color[HTML]{000000} 0.071 & {\cellcolor[HTML]{CDCDE4}} \color[HTML]{000000} 0.075 & {\cellcolor[HTML]{B6B6D8}} \color[HTML]{000000} 0.091 \\
 &  & F. 40B  & {\cellcolor[HTML]{B1E0AB}} \color[HTML]{000000} 0.088 & {\cellcolor[HTML]{9FD899}} \color[HTML]{000000} 0.1 & {\cellcolor[HTML]{80CA80}} \color[HTML]{000000} 0.12 & {\cellcolor[HTML]{7CC87C}} \color[HTML]{000000} 0.12 & {\cellcolor[HTML]{FDC794}} \color[HTML]{000000} 0.12 & {\cellcolor[HTML]{FDC48F}} \color[HTML]{000000} 0.12 & {\cellcolor[HTML]{FD9C51}} \color[HTML]{000000} 0.16 & {\cellcolor[HTML]{FDAD69}} \color[HTML]{000000} 0.14 & {\cellcolor[HTML]{C6C7E1}} \color[HTML]{000000} 0.08 & {\cellcolor[HTML]{9C98C7}} \color[HTML]{F1F1F1} 0.11 & {\cellcolor[HTML]{9390C3}} \color[HTML]{F1F1F1} 0.12 & {\cellcolor[HTML]{7C75B6}} \color[HTML]{F1F1F1} 0.13 \\
 &  & Gpt 3 & {\cellcolor[HTML]{5BB86A}} \color[HTML]{F1F1F1} 0.14 & {\cellcolor[HTML]{4BB062}} \color[HTML]{F1F1F1} 0.14 & {\cellcolor[HTML]{005321}} \color[HTML]{F1F1F1} 0.21 & {\cellcolor[HTML]{00682A}} \color[HTML]{F1F1F1} 0.2 & {\cellcolor[HTML]{FDA057}} \color[HTML]{000000} 0.16 & {\cellcolor[HTML]{FDA057}} \color[HTML]{000000} 0.16 & {\cellcolor[HTML]{7F2704}} \color[HTML]{F1F1F1} 0.32 & {\cellcolor[HTML]{963003}} \color[HTML]{F1F1F1} 0.3 & {\cellcolor[HTML]{7A71B4}} \color[HTML]{F1F1F1} 0.13 & {\cellcolor[HTML]{6B53A4}} \color[HTML]{F1F1F1} 0.15 & {\cellcolor[HTML]{52238D}} \color[HTML]{F1F1F1} 0.17 & {\cellcolor[HTML]{5B3495}} \color[HTML]{F1F1F1} 0.16 \\
 &  & Gpt 4 & {\cellcolor[HTML]{026F2E}} \color[HTML]{F1F1F1} 0.2 & {\cellcolor[HTML]{2F974E}} \color[HTML]{F1F1F1} 0.16 & {\cellcolor[HTML]{006C2C}} \color[HTML]{F1F1F1} 0.2 & {\cellcolor[HTML]{16803C}} \color[HTML]{F1F1F1} 0.18 & {\cellcolor[HTML]{DB4B03}} \color[HTML]{F1F1F1} 0.25 & {\cellcolor[HTML]{EF6612}} \color[HTML]{F1F1F1} 0.22 & {\cellcolor[HTML]{963003}} \color[HTML]{F1F1F1} 0.3 & {\cellcolor[HTML]{D34601}} \color[HTML]{F1F1F1} 0.25 & {\cellcolor[HTML]{4C1888}} \color[HTML]{F1F1F1} 0.18 & {\cellcolor[HTML]{705CA9}} \color[HTML]{F1F1F1} 0.14 & {\cellcolor[HTML]{5B3495}} \color[HTML]{F1F1F1} 0.16 & {\cellcolor[HTML]{66499F}} \color[HTML]{F1F1F1} 0.15 \\
 &  & Mpt 7B & {\cellcolor[HTML]{DEF2D9}} \color[HTML]{000000} 0.054 & {\cellcolor[HTML]{EEF8EA}} \color[HTML]{000000} 0.036 & {\cellcolor[HTML]{E0F3DB}} \color[HTML]{000000} 0.053 & {\cellcolor[HTML]{D8F0D2}} \color[HTML]{000000} 0.059 & {\cellcolor[HTML]{FEE6CE}} \color[HTML]{000000} 0.072 & {\cellcolor[HTML]{FDD1A3}} \color[HTML]{000000} 0.11 & {\cellcolor[HTML]{FEE5CB}} \color[HTML]{000000} 0.074 & {\cellcolor[HTML]{FDD8B2}} \color[HTML]{000000} 0.095 & {\cellcolor[HTML]{CDCDE4}} \color[HTML]{000000} 0.075 & {\cellcolor[HTML]{FAF8FB}} \color[HTML]{000000} 0.027 & {\cellcolor[HTML]{EAE8F2}} \color[HTML]{000000} 0.05 & {\cellcolor[HTML]{E8E6F2}} \color[HTML]{000000} 0.052 \\
 &  & Mpt 30B & {\cellcolor[HTML]{BAE3B3}} \color[HTML]{000000} 0.082 & {\cellcolor[HTML]{CFECC9}} \color[HTML]{000000} 0.067 & {\cellcolor[HTML]{C3E7BC}} \color[HTML]{000000} 0.076 & {\cellcolor[HTML]{B4E1AD}} \color[HTML]{000000} 0.086 & {\cellcolor[HTML]{FDD5AB}} \color[HTML]{000000} 0.1 & {\cellcolor[HTML]{FDD7AF}} \color[HTML]{000000} 0.098 & {\cellcolor[HTML]{FDD9B5}} \color[HTML]{000000} 0.092 & {\cellcolor[HTML]{FDB576}} \color[HTML]{000000} 0.14 & {\cellcolor[HTML]{BABADB}} \color[HTML]{000000} 0.088 & {\cellcolor[HTML]{DCDCEC}} \color[HTML]{000000} 0.064 & {\cellcolor[HTML]{C6C7E1}} \color[HTML]{000000} 0.08 & {\cellcolor[HTML]{BCBDDC}} \color[HTML]{000000} 0.087 \\
 &  & V. 13B  & {\cellcolor[HTML]{A4DA9E}} \color[HTML]{000000} 0.096 & {\cellcolor[HTML]{A4DA9E}} \color[HTML]{000000} 0.096 & {\cellcolor[HTML]{ABDDA5}} \color[HTML]{000000} 0.091 & {\cellcolor[HTML]{ACDEA6}} \color[HTML]{000000} 0.091 & {\cellcolor[HTML]{FDD1A4}} \color[HTML]{000000} 0.11 & {\cellcolor[HTML]{FDC794}} \color[HTML]{000000} 0.12 & {\cellcolor[HTML]{FDCB9B}} \color[HTML]{000000} 0.11 & {\cellcolor[HTML]{FDC38D}} \color[HTML]{000000} 0.12 & {\cellcolor[HTML]{A19ECA}} \color[HTML]{F1F1F1} 0.11 & {\cellcolor[HTML]{ACAAD1}} \color[HTML]{000000} 0.098 & {\cellcolor[HTML]{9692C4}} \color[HTML]{F1F1F1} 0.11 & {\cellcolor[HTML]{BFC0DE}} \color[HTML]{000000} 0.084 \\
 &  & V. 33B  & {\cellcolor[HTML]{ACDEA6}} \color[HTML]{000000} 0.091 & {\cellcolor[HTML]{CBEBC5}} \color[HTML]{000000} 0.07 & {\cellcolor[HTML]{D2EDCC}} \color[HTML]{000000} 0.064 & {\cellcolor[HTML]{B1E0AB}} \color[HTML]{000000} 0.087 & {\cellcolor[HTML]{FDD7AF}} \color[HTML]{000000} 0.098 & {\cellcolor[HTML]{FEDEBF}} \color[HTML]{000000} 0.085 & {\cellcolor[HTML]{FEE3C8}} \color[HTML]{000000} 0.077 & {\cellcolor[HTML]{FDD3A7}} \color[HTML]{000000} 0.1 & {\cellcolor[HTML]{ACAAD1}} \color[HTML]{000000} 0.098 & {\cellcolor[HTML]{CECEE5}} \color[HTML]{000000} 0.074 & {\cellcolor[HTML]{CECEE5}} \color[HTML]{000000} 0.074 & {\cellcolor[HTML]{B0AFD4}} \color[HTML]{000000} 0.095 \\
 &  & W. V. 13B  & {\cellcolor[HTML]{63BC6E}} \color[HTML]{F1F1F1} 0.13 & {\cellcolor[HTML]{CCEBC6}} \color[HTML]{000000} 0.069 & {\cellcolor[HTML]{BCE4B5}} \color[HTML]{000000} 0.081 & {\cellcolor[HTML]{A0D99B}} \color[HTML]{000000} 0.098 & {\cellcolor[HTML]{FDAE6A}} \color[HTML]{000000} 0.14 & {\cellcolor[HTML]{FEE1C4}} \color[HTML]{000000} 0.08 & {\cellcolor[HTML]{FDD1A4}} \color[HTML]{000000} 0.11 & {\cellcolor[HTML]{FDBA7F}} \color[HTML]{000000} 0.13 & {\cellcolor[HTML]{7970B3}} \color[HTML]{F1F1F1} 0.14 & {\cellcolor[HTML]{CFD0E6}} \color[HTML]{000000} 0.073 & {\cellcolor[HTML]{CECEE5}} \color[HTML]{000000} 0.074 & {\cellcolor[HTML]{B3B3D6}} \color[HTML]{000000} 0.093 \\

  \cmidrule{2-15}
 & \multirow[c]{9}{*}{\rotatebox[origin=l]{270}{\thead{With Guidelines}}} & F. 7B & {\cellcolor[HTML]{A7DBA0}} \color[HTML]{000000} 0.094 & {\cellcolor[HTML]{BDE5B6}} \color[HTML]{000000} 0.079 & {\cellcolor[HTML]{C2E7BB}} \color[HTML]{000000} 0.077 & {\cellcolor[HTML]{AFDFA8}} \color[HTML]{000000} 0.089 & {\cellcolor[HTML]{FDC38D}} \color[HTML]{000000} 0.12 & {\cellcolor[HTML]{FDC088}} \color[HTML]{000000} 0.12 & {\cellcolor[HTML]{FDD8B2}} \color[HTML]{000000} 0.095 & {\cellcolor[HTML]{FDC997}} \color[HTML]{000000} 0.11 & {\cellcolor[HTML]{A19ECA}} \color[HTML]{F1F1F1} 0.11 & {\cellcolor[HTML]{B7B7D9}} \color[HTML]{000000} 0.09 & {\cellcolor[HTML]{B5B5D7}} \color[HTML]{000000} 0.092 & {\cellcolor[HTML]{A09DCA}} \color[HTML]{F1F1F1} 0.11 \\
 &  & F. 40B  & {\cellcolor[HTML]{D6EFD0}} \color[HTML]{000000} 0.061 & {\cellcolor[HTML]{B5E1AE}} \color[HTML]{000000} 0.085 & {\cellcolor[HTML]{A9DCA3}} \color[HTML]{000000} 0.093 & {\cellcolor[HTML]{BCE4B5}} \color[HTML]{000000} 0.081 & {\cellcolor[HTML]{FDD5AB}} \color[HTML]{000000} 0.1 & {\cellcolor[HTML]{FDC590}} \color[HTML]{000000} 0.12 & {\cellcolor[HTML]{FDA762}} \color[HTML]{000000} 0.15 & {\cellcolor[HTML]{FDAD69}} \color[HTML]{000000} 0.14 & {\cellcolor[HTML]{E2E2EF}} \color[HTML]{000000} 0.057 & {\cellcolor[HTML]{C8C8E2}} \color[HTML]{000000} 0.079 & {\cellcolor[HTML]{C7C8E1}} \color[HTML]{000000} 0.079 & {\cellcolor[HTML]{CECFE5}} \color[HTML]{000000} 0.073 \\
 &  & Gpt 3 & {\cellcolor[HTML]{004D1F}} \color[HTML]{F1F1F1} 0.22 & {\cellcolor[HTML]{004E1F}} \color[HTML]{F1F1F1} 0.22 & {\cellcolor[HTML]{005F26}} \color[HTML]{F1F1F1} 0.21 & {\cellcolor[HTML]{005020}} \color[HTML]{F1F1F1} 0.22 & {\cellcolor[HTML]{993103}} \color[HTML]{F1F1F1} 0.3 & {\cellcolor[HTML]{832804}} \color[HTML]{F1F1F1} 0.32 & {\cellcolor[HTML]{942F03}} \color[HTML]{F1F1F1} 0.3 & {\cellcolor[HTML]{7F2704}} \color[HTML]{F1F1F1} 0.32 & {\cellcolor[HTML]{3F007D}} \color[HTML]{F1F1F1} 0.19 & {\cellcolor[HTML]{4A1587}} \color[HTML]{F1F1F1} 0.18 & {\cellcolor[HTML]{582E92}} \color[HTML]{F1F1F1} 0.17 & {\cellcolor[HTML]{4F1D8B}} \color[HTML]{F1F1F1} 0.18 \\
 &  & Gpt 4 & {\cellcolor[HTML]{3BA458}} \color[HTML]{F1F1F1} 0.15 & {\cellcolor[HTML]{2F984F}} \color[HTML]{F1F1F1} 0.16 & {\cellcolor[HTML]{067230}} \color[HTML]{F1F1F1} 0.19 & {\cellcolor[HTML]{117B38}} \color[HTML]{F1F1F1} 0.19 & {\cellcolor[HTML]{E75C0C}} \color[HTML]{F1F1F1} 0.23 & {\cellcolor[HTML]{EA5F0E}} \color[HTML]{F1F1F1} 0.22 & {\cellcolor[HTML]{973003}} \color[HTML]{F1F1F1} 0.3 & {\cellcolor[HTML]{C34002}} \color[HTML]{F1F1F1} 0.26 & {\cellcolor[HTML]{8481BC}} \color[HTML]{F1F1F1} 0.13 & {\cellcolor[HTML]{776AB0}} \color[HTML]{F1F1F1} 0.14 & {\cellcolor[HTML]{674CA1}} \color[HTML]{F1F1F1} 0.15 & {\cellcolor[HTML]{62429C}} \color[HTML]{F1F1F1} 0.16 \\
 &  & Mpt 7B & {\cellcolor[HTML]{D4EECE}} \color[HTML]{000000} 0.063 & {\cellcolor[HTML]{E5F5E0}} \color[HTML]{000000} 0.049 & {\cellcolor[HTML]{000000}} \color[HTML]{F1F1F1} nan & {\cellcolor[HTML]{000000}} \color[HTML]{F1F1F1} nan & {\cellcolor[HTML]{FEE2C6}} \color[HTML]{000000} 0.079 & {\cellcolor[HTML]{FEE6CF}} \color[HTML]{000000} 0.071 & {\cellcolor[HTML]{000000}} \color[HTML]{F1F1F1} nan & {\cellcolor[HTML]{000000}} \color[HTML]{F1F1F1} nan & {\cellcolor[HTML]{D6D6E9}} \color[HTML]{000000} 0.069 & {\cellcolor[HTML]{E2E2EF}} \color[HTML]{000000} 0.057 & {\cellcolor[HTML]{000000}} \color[HTML]{F1F1F1} nan & {\cellcolor[HTML]{000000}} \color[HTML]{F1F1F1} nan \\
 &  & Mpt 30B & {\cellcolor[HTML]{C8E9C1}} \color[HTML]{000000} 0.073 & {\cellcolor[HTML]{DAF0D4}} \color[HTML]{000000} 0.058 & {\cellcolor[HTML]{000000}} \color[HTML]{F1F1F1} nan & {\cellcolor[HTML]{000000}} \color[HTML]{F1F1F1} nan & {\cellcolor[HTML]{FDC28B}} \color[HTML]{000000} 0.12 & {\cellcolor[HTML]{FDD5AB}} \color[HTML]{000000} 0.1 & {\cellcolor[HTML]{000000}} \color[HTML]{F1F1F1} nan & {\cellcolor[HTML]{000000}} \color[HTML]{F1F1F1} nan & {\cellcolor[HTML]{D7D7E9}} \color[HTML]{000000} 0.068 & {\cellcolor[HTML]{E8E6F2}} \color[HTML]{000000} 0.052 & {\cellcolor[HTML]{000000}} \color[HTML]{F1F1F1} nan & {\cellcolor[HTML]{000000}} \color[HTML]{F1F1F1} nan \\
 &  & V. 13B  & {\cellcolor[HTML]{B5E1AE}} \color[HTML]{000000} 0.085 & {\cellcolor[HTML]{9ED798}} \color[HTML]{000000} 0.1 & {\cellcolor[HTML]{BBE4B4}} \color[HTML]{000000} 0.081 & {\cellcolor[HTML]{A9DCA3}} \color[HTML]{000000} 0.093 & {\cellcolor[HTML]{FDCB9B}} \color[HTML]{000000} 0.11 & {\cellcolor[HTML]{FDC692}} \color[HTML]{000000} 0.12 & {\cellcolor[HTML]{FEDCB9}} \color[HTML]{000000} 0.09 & {\cellcolor[HTML]{FDD5AB}} \color[HTML]{000000} 0.1 & {\cellcolor[HTML]{A9A7CF}} \color[HTML]{F1F1F1} 0.1 & {\cellcolor[HTML]{A3A0CB}} \color[HTML]{F1F1F1} 0.1 & {\cellcolor[HTML]{BCBDDC}} \color[HTML]{000000} 0.087 & {\cellcolor[HTML]{A29FCB}} \color[HTML]{F1F1F1} 0.1 \\
 &  & V. 33B  & {\cellcolor[HTML]{E5F5E1}} \color[HTML]{000000} 0.048 & {\cellcolor[HTML]{E7F6E3}} \color[HTML]{000000} 0.046 & {\cellcolor[HTML]{F3FAF0}} \color[HTML]{000000} 0.029 & {\cellcolor[HTML]{F7FCF5}} \color[HTML]{000000} 0.023 & {\cellcolor[HTML]{FEE7D1}} \color[HTML]{000000} 0.069 & {\cellcolor[HTML]{FEEDDC}} \color[HTML]{000000} 0.056 & {\cellcolor[HTML]{FFF5EB}} \color[HTML]{000000} 0.037 & {\cellcolor[HTML]{FFF5EB}} \color[HTML]{000000} 0.037 & {\cellcolor[HTML]{EEECF4}} \color[HTML]{000000} 0.046 & {\cellcolor[HTML]{E9E8F2}} \color[HTML]{000000} 0.051 & {\cellcolor[HTML]{F9F8FB}} \color[HTML]{000000} 0.028 & {\cellcolor[HTML]{FCFBFD}} \color[HTML]{000000} 0.023 \\
 &  & W. V. 13B  & {\cellcolor[HTML]{B7E2B1}} \color[HTML]{000000} 0.083 & {\cellcolor[HTML]{E9F7E5}} \color[HTML]{000000} 0.042 & {\cellcolor[HTML]{C0E6B9}} \color[HTML]{000000} 0.078 & {\cellcolor[HTML]{D7EFD1}} \color[HTML]{000000} 0.06 & {\cellcolor[HTML]{FEDCBB}} \color[HTML]{000000} 0.089 & {\cellcolor[HTML]{FEEDDC}} \color[HTML]{000000} 0.055 & {\cellcolor[HTML]{FDCD9C}} \color[HTML]{000000} 0.11 & {\cellcolor[HTML]{FEE6CE}} \color[HTML]{000000} 0.073 & {\cellcolor[HTML]{AFAED4}} \color[HTML]{000000} 0.095 & {\cellcolor[HTML]{F1F0F6}} \color[HTML]{000000} 0.04 & {\cellcolor[HTML]{C4C5E0}} \color[HTML]{000000} 0.081 & {\cellcolor[HTML]{DADAEB}} \color[HTML]{000000} 0.066 \\

\bottomrule

\end{tabular}
\setlength{\tabcolsep}{6pt}
\end{table}


Looking at BLEU, METEOR and ROGUE scores, it is possible that some models improve on average with the number of examples. Although this is not always the case, most models exhibit clearly improved performances with the number of examples. This effect is particularly strongly observed with ChatGPT 3.5 turbo. This confirms the initial findings on the positive effect of the number of examples given to the model in the quality of the answers.
On the other hand, by focusing on the presence and/or absence of guidelines, there are some mixed results. Some models seem to perform better with guidelines, while others do not. We hypothesise that smaller models may have a hard time understanding longer contexts since those are near their length limits. It has to be noted that although the longest prompt is no longer than a 1000 words, the tokenisers of most models split words in more than one token. Additionally, special characters that are used often are tokenised individually. This means that many models tokenise the longest prompts at $\sim$1700 tokens, which, combined with the text of the longest narrative, is slightly over $\sim$1900 tokens. This high value is indeed very close to the maximum context window of many open-source models, which have 2048 tokens as context.

It is also possible to observe that the presence of guidelines for both OpenAI models improves the baseline 0-shot experimental setting, although the 5-shot results are mostly unchanged. We believe this result is because in the guidelines, there are a few examples that contribute as learning examples for the models.  

While initially, our belief was that including colour information in some way would very likely increase the performance of our models and in the worst case, get the same result, that belief was unexpectedly wrong. Except for ChatGPT models, all models perform worse when given colour information than when not given colour information. By analysing their outputs, it is possible to make an educated guess as to what is happening. The text formatting used to represent colour information is likely confusing the models, which assimilate parenthesis to source code. Therefore the models try to generate source code of some programming language. It is unclear why ChatGPT models do not have this issue. Another confirmation of this issue is given by the number of null answers. 
\begin{table}
    \centering
    \caption{Percentage of null answers across the tested models. A null answer is an answer that contains only null or whitespace characters or punctuation marks. Notice how the presence of colour information increases the percentage of null answers.}
    \label{tab:personal-narrative-elicitation-null-answers}
    \begin{tabular}{l|l|l|rrrr}
    \toprule
\multicolumn{7}{c}{\thead{Null answers}} \\
    \midrule
    \thead{Colour} & \thead{Guidelines} & \thead{Model name} & \thead{0-shot} & \thead{1-shot} & \thead{3-shot} & \thead{5-shot} \\
    \midrule
    \multirow[c]{18}{*}{\thead{Without\\ Colour}} & \multirow[c]{9}{*}{\thead{Without\\ Guidelines}} & Falcon 7B & {\cellcolor[HTML]{FFFFFF}} \color[HTML]{000000} 0.00 & {\cellcolor[HTML]{FFFFFF}} \color[HTML]{000000} 0.00 & {\cellcolor[HTML]{FDFDF8}} \color[HTML]{000000} 0.02 & {\cellcolor[HTML]{FFFFFF}} \color[HTML]{000000} 0.00 \\
 &  & Falcon 40B Instruct & {\cellcolor[HTML]{FFFFFF}} \color[HTML]{000000} 0.00 & {\cellcolor[HTML]{FFFFFF}} \color[HTML]{000000} 0.00 & {\cellcolor[HTML]{FDFDF8}} \color[HTML]{000000} 0.02 & {\cellcolor[HTML]{FFFFFF}} \color[HTML]{000000} 0.00 \\
 &  & Gpt 3 & {\cellcolor[HTML]{FFFFFF}} \color[HTML]{000000} 0.00 & {\cellcolor[HTML]{FFFFFF}} \color[HTML]{000000} 0.00 & {\cellcolor[HTML]{FFFFFF}} \color[HTML]{000000} 0.00 & {\cellcolor[HTML]{FFFFFF}} \color[HTML]{000000} 0.00 \\
 &  & Gpt 4 & {\cellcolor[HTML]{FFFFFF}} \color[HTML]{000000} 0.00 & {\cellcolor[HTML]{FFFFFF}} \color[HTML]{000000} 0.00 & {\cellcolor[HTML]{FFFFFF}} \color[HTML]{000000} 0.00 & {\cellcolor[HTML]{FFFFFF}} \color[HTML]{000000} 0.00 \\
 &  & Mpt 7B & {\cellcolor[HTML]{FDFDF8}} \color[HTML]{000000} 0.02 & {\cellcolor[HTML]{FDFDF8}} \color[HTML]{000000} 0.02 & {\cellcolor[HTML]{ECECC2}} \color[HTML]{000000} 0.12 & {\cellcolor[HTML]{FDFDF8}} \color[HTML]{000000} 0.02 \\
 &  & Mpt 30B Chat & {\cellcolor[HTML]{FFFFFF}} \color[HTML]{000000} 0.00 & {\cellcolor[HTML]{FDFDF8}} \color[HTML]{000000} 0.02 & {\cellcolor[HTML]{E9E9B8}} \color[HTML]{000000} 0.14 & {\cellcolor[HTML]{FDFDF8}} \color[HTML]{000000} 0.02 \\
 &  & Vicuna 13B V1 & {\cellcolor[HTML]{FFFFFF}} \color[HTML]{000000} 0.00 & {\cellcolor[HTML]{FFFFFF}} \color[HTML]{000000} 0.00 & {\cellcolor[HTML]{F2F2D6}} \color[HTML]{000000} 0.09 & {\cellcolor[HTML]{FFFFFF}} \color[HTML]{000000} 0.00 \\
 &  & Vicuna 33B V1 & {\cellcolor[HTML]{FFFFFF}} \color[HTML]{000000} 0.00 & {\cellcolor[HTML]{FAFAF0}} \color[HTML]{000000} 0.03 & {\cellcolor[HTML]{F2F2D6}} \color[HTML]{000000} 0.09 & {\cellcolor[HTML]{FFFFFF}} \color[HTML]{000000} 0.00 \\
 &  & Wizard Vicuna 13B Uncensored HF & {\cellcolor[HTML]{FFFFFF}} \color[HTML]{000000} 0.00 & {\cellcolor[HTML]{FFFFFF}} \color[HTML]{000000} 0.00 & {\cellcolor[HTML]{F2F2D6}} \color[HTML]{000000} 0.09 & {\cellcolor[HTML]{FFFFFF}} \color[HTML]{000000} 0.00 \\
     \cmidrule{2-7}
     & \multirow[c]{9}{*}{\thead{With\\ Guidelines}} & Falcon 7B & {\cellcolor[HTML]{FFFFFF}} \color[HTML]{000000} 0.00 & {\cellcolor[HTML]{FFFFFF}} \color[HTML]{000000} 0.00 & {\cellcolor[HTML]{FFFFFF}} \color[HTML]{000000} 0.00 & {\cellcolor[HTML]{FFFFFF}} \color[HTML]{000000} 0.00 \\
 &  & Falcon 40B Instruct & {\cellcolor[HTML]{FDFDF8}} \color[HTML]{000000} 0.02 & {\cellcolor[HTML]{FDFDF8}} \color[HTML]{000000} 0.02 & {\cellcolor[HTML]{FDFDF8}} \color[HTML]{000000} 0.02 & {\cellcolor[HTML]{FFFFFF}} \color[HTML]{000000} 0.00 \\
 &  & Gpt 3 & {\cellcolor[HTML]{FFFFFF}} \color[HTML]{000000} 0.00 & {\cellcolor[HTML]{FFFFFF}} \color[HTML]{000000} 0.00 & {\cellcolor[HTML]{FFFFFF}} \color[HTML]{000000} 0.00 & {\cellcolor[HTML]{FFFFFF}} \color[HTML]{000000} 0.00 \\
 &  & Gpt 4 & {\cellcolor[HTML]{FFFFFF}} \color[HTML]{000000} 0.00 & {\cellcolor[HTML]{FFFFFF}} \color[HTML]{000000} 0.00 & {\cellcolor[HTML]{FFFFFF}} \color[HTML]{000000} 0.00 & {\cellcolor[HTML]{FFFFFF}} \color[HTML]{000000} 0.00 \\
 &  & Mpt 7B & {\cellcolor[HTML]{F7F7E7}} \color[HTML]{000000} 0.05 & {\cellcolor[HTML]{F4F4DE}} \color[HTML]{000000} 0.07 & {\cellcolor[HTML]{000000}} \color[HTML]{F1F1F1} nan & {\cellcolor[HTML]{000000}} \color[HTML]{F1F1F1} nan \\
 &  & Mpt 30B Chat & {\cellcolor[HTML]{E4DEAE}} \color[HTML]{000000} 0.17 & {\cellcolor[HTML]{FDFDF8}} \color[HTML]{000000} 0.02 & {\cellcolor[HTML]{FFFFFF}} \color[HTML]{000000} 0.00 & {\cellcolor[HTML]{FAFAF0}} \color[HTML]{000000} 0.03 \\
 &  & Vicuna 13B V1 & {\cellcolor[HTML]{F4F4DE}} \color[HTML]{000000} 0.07 & {\cellcolor[HTML]{EFEFCC}} \color[HTML]{000000} 0.10 & {\cellcolor[HTML]{EFEFCC}} \color[HTML]{000000} 0.10 & {\cellcolor[HTML]{EFEFCC}} \color[HTML]{000000} 0.10 \\
 &  & Vicuna 33B V1 & {\cellcolor[HTML]{CFA791}} \color[HTML]{000000} 0.29 & {\cellcolor[HTML]{F2F2D6}} \color[HTML]{000000} 0.09 & {\cellcolor[HTML]{E4DEAE}} \color[HTML]{000000} 0.17 & {\cellcolor[HTML]{F4F4DE}} \color[HTML]{000000} 0.07 \\
 &  & Wizard Vicuna 13B Uncensored HF & {\cellcolor[HTML]{9F6767}} \color[HTML]{F1F1F1} 0.43 & {\cellcolor[HTML]{E9E9B8}} \color[HTML]{000000} 0.14 & {\cellcolor[HTML]{F2F2D6}} \color[HTML]{000000} 0.09 & {\cellcolor[HTML]{EFEFCC}} \color[HTML]{000000} 0.10 \\
\midrule
    \multirow[c]{18}{*}{\thead{With\\ Colour}} & \multirow[c]{9}{*}{\thead{Without\\ Guidelines}}  & Falcon 7B & {\cellcolor[HTML]{FFFFFF}} \color[HTML]{000000} 0.00 & {\cellcolor[HTML]{FFFFFF}} \color[HTML]{000000} 0.00 & {\cellcolor[HTML]{F4F4DE}} \color[HTML]{000000} 0.07 & {\cellcolor[HTML]{FFFFFF}} \color[HTML]{000000} 0.00 \\
 &  & Falcon 40B Instruct & {\cellcolor[HTML]{FFFFFF}} \color[HTML]{000000} 0.00 & {\cellcolor[HTML]{FFFFFF}} \color[HTML]{000000} 0.00 & {\cellcolor[HTML]{F7F7E7}} \color[HTML]{000000} 0.05 & {\cellcolor[HTML]{FDFDF8}} \color[HTML]{000000} 0.02 \\
 &  & Gpt 3 & {\cellcolor[HTML]{FFFFFF}} \color[HTML]{000000} 0.00 & {\cellcolor[HTML]{FFFFFF}} \color[HTML]{000000} 0.00 & {\cellcolor[HTML]{FFFFFF}} \color[HTML]{000000} 0.00 & {\cellcolor[HTML]{FFFFFF}} \color[HTML]{000000} 0.00 \\
 &  & Gpt 4 & {\cellcolor[HTML]{FFFFFF}} \color[HTML]{000000} 0.00 & {\cellcolor[HTML]{FFFFFF}} \color[HTML]{000000} 0.00 & {\cellcolor[HTML]{FFFFFF}} \color[HTML]{000000} 0.00 & {\cellcolor[HTML]{FFFFFF}} \color[HTML]{000000} 0.00 \\
 &  & Mpt 7B & {\cellcolor[HTML]{FDFDF8}} \color[HTML]{000000} 0.02 & {\cellcolor[HTML]{E7E5B2}} \color[HTML]{000000} 0.16 & {\cellcolor[HTML]{D5B89A}} \color[HTML]{000000} 0.26 & {\cellcolor[HTML]{D5B89A}} \color[HTML]{000000} 0.26 \\
 &  & Mpt 30B Chat & {\cellcolor[HTML]{FFFFFF}} \color[HTML]{000000} 0.00 & {\cellcolor[HTML]{FDFDF8}} \color[HTML]{000000} 0.02 & {\cellcolor[HTML]{D5B89A}} \color[HTML]{000000} 0.26 & {\cellcolor[HTML]{FDFDF8}} \color[HTML]{000000} 0.02 \\
 &  & Vicuna 13B V1 & {\cellcolor[HTML]{FFFFFF}} \color[HTML]{000000} 0.00 & {\cellcolor[HTML]{F4F4DE}} \color[HTML]{000000} 0.07 & {\cellcolor[HTML]{DBC8A2}} \color[HTML]{000000} 0.22 & {\cellcolor[HTML]{E7E5B2}} \color[HTML]{000000} 0.16 \\
 &  & Vicuna 33B V1 & {\cellcolor[HTML]{FDFDF8}} \color[HTML]{000000} 0.02 & {\cellcolor[HTML]{ECECC2}} \color[HTML]{000000} 0.12 & {\cellcolor[HTML]{CFA791}} \color[HTML]{000000} 0.29 & {\cellcolor[HTML]{E9E9B8}} \color[HTML]{000000} 0.14 \\
 &  & Wizard Vicuna 13B Uncensored HF & {\cellcolor[HTML]{FFFFFF}} \color[HTML]{000000} 0.00 & {\cellcolor[HTML]{F7F7E7}} \color[HTML]{000000} 0.05 & {\cellcolor[HTML]{D5B89A}} \color[HTML]{000000} 0.26 & {\cellcolor[HTML]{F4F4DE}} \color[HTML]{000000} 0.07 \\
     \cmidrule{2-7}
     & \multirow[c]{9}{*}{\thead{With\\ Guidelines}}& Falcon 7B & {\cellcolor[HTML]{FFFFFF}} \color[HTML]{000000} 0.00 & {\cellcolor[HTML]{FFFFFF}} \color[HTML]{000000} 0.00 & {\cellcolor[HTML]{FFFFFF}} \color[HTML]{000000} 0.00 & {\cellcolor[HTML]{FDFDF8}} \color[HTML]{000000} 0.02 \\
 &  & Falcon 40B Instruct & {\cellcolor[HTML]{FAFAF0}} \color[HTML]{000000} 0.03 & {\cellcolor[HTML]{FDFDF8}} \color[HTML]{000000} 0.02 & {\cellcolor[HTML]{F2F2D6}} \color[HTML]{000000} 0.09 & {\cellcolor[HTML]{FDFDF8}} \color[HTML]{000000} 0.02 \\
 &  & Gpt 3 & {\cellcolor[HTML]{FFFFFF}} \color[HTML]{000000} 0.00 & {\cellcolor[HTML]{FFFFFF}} \color[HTML]{000000} 0.00 & {\cellcolor[HTML]{FFFFFF}} \color[HTML]{000000} 0.00 & {\cellcolor[HTML]{FFFFFF}} \color[HTML]{000000} 0.00 \\
 &  & Gpt 4 & {\cellcolor[HTML]{FFFFFF}} \color[HTML]{000000} 0.00 & {\cellcolor[HTML]{FFFFFF}} \color[HTML]{000000} 0.00 & {\cellcolor[HTML]{FFFFFF}} \color[HTML]{000000} 0.00 & {\cellcolor[HTML]{FFFFFF}} \color[HTML]{000000} 0.00 \\
 &  & Mpt 7B & {\cellcolor[HTML]{FDFDF8}} \color[HTML]{000000} 0.02 & {\cellcolor[HTML]{EFEFCC}} \color[HTML]{000000} 0.10 & {\cellcolor[HTML]{000000}} \color[HTML]{F1F1F1} nan & {\cellcolor[HTML]{000000}} \color[HTML]{F1F1F1} nan \\
 &  & Mpt 30B Chat & {\cellcolor[HTML]{ECECC2}} \color[HTML]{000000} 0.12 & {\cellcolor[HTML]{E7E5B2}} \color[HTML]{000000} 0.16 & {\cellcolor[HTML]{000000}} \color[HTML]{F1F1F1} nan & {\cellcolor[HTML]{000000}} \color[HTML]{F1F1F1} nan \\
 &  & Vicuna 13B V1 & {\cellcolor[HTML]{F4F4DE}} \color[HTML]{000000} 0.07 & {\cellcolor[HTML]{ECECC2}} \color[HTML]{000000} 0.12 & {\cellcolor[HTML]{E1D7AA}} \color[HTML]{000000} 0.19 & {\cellcolor[HTML]{E7E5B2}} \color[HTML]{000000} 0.16 \\
 &  & Vicuna 33B V1 & {\cellcolor[HTML]{C89387}} \color[HTML]{F1F1F1} 0.33 & {\cellcolor[HTML]{C27E7E}} \color[HTML]{F1F1F1} 0.36 & {\cellcolor[HTML]{643E3E}} \color[HTML]{F1F1F1} 0.52 & {\cellcolor[HTML]{1E0000}} \color[HTML]{F1F1F1} 0.57 \\
 &  & Wizard Vicuna 13B Uncensored HF & {\cellcolor[HTML]{C89387}} \color[HTML]{F1F1F1} 0.33 & {\cellcolor[HTML]{956060}} \color[HTML]{F1F1F1} 0.45 & {\cellcolor[HTML]{CFA791}} \color[HTML]{000000} 0.29 & {\cellcolor[HTML]{B17272}} \color[HTML]{F1F1F1} 0.40 \\
    \bottomrule
    \end{tabular}
                
\end{table}

Null answers, i.e, answers that contain only non-text-based characters, were more pronounced with colour information. See Table \ref{tab:personal-narrative-elicitation-null-answers}.

% [ONLY SHORT NARRATIVES]

\begin{table}[!htbp]
    \centering
    \caption{Wasserstein divergence metric to the reference human token distribution. It is unable to highlight significant differences, except for Falcon models, for which the distributions are completely different from the human. Tokens distributions are computed using Spacy NLP tokenizer, escluding punctuation, stop words and digits.}
    \label{tab:personal-narrative-elicitation-wasserstein}
\begin{tabular}{l|l|l|rrrr}
\toprule
\multicolumn{7}{c}{\thead{Wasserstein divergence metric}}\\
\midrule
\thead{Colour} & \thead{Guidelines} & \thead{Model name} & \thead{0-shot} & \thead{1-shot} & \thead{3-shot} & \thead{5-shot} \\
\midrule
\multirow[c]{18}{*}{\thead{Without\\ Colour}} & \multirow[c]{9}{*}{\thead{Without\\ Guidelines}} & Falcon 7B & {\cellcolor[HTML]{852D90}} \color[HTML]{F1F1F1} 0.17 & {\cellcolor[HTML]{99B1D4}} \color[HTML]{000000} 0.09 & {\cellcolor[HTML]{ACC6DF}} \color[HTML]{000000} 0.07 & {\cellcolor[HTML]{DCE9F2}} \color[HTML]{000000} 0.03 \\
 &  & Falcon 40B Instruct & {\cellcolor[HTML]{E0ECF4}} \color[HTML]{000000} 0.03 & {\cellcolor[HTML]{D4E3EF}} \color[HTML]{000000} 0.04 & {\cellcolor[HTML]{D8E6F0}} \color[HTML]{000000} 0.04 & {\cellcolor[HTML]{DCE9F2}} \color[HTML]{000000} 0.03 \\
 &  & ChatGPT 3.5 turbo & {\cellcolor[HTML]{F1F8FB}} \color[HTML]{000000} 0.01 & {\cellcolor[HTML]{F0F7FA}} \color[HTML]{000000} 0.01 & {\cellcolor[HTML]{E4EEF5}} \color[HTML]{000000} 0.02 & {\cellcolor[HTML]{DAE7F1}} \color[HTML]{000000} 0.03 \\
 &  & ChatGPT 4 & {\cellcolor[HTML]{E2EDF5}} \color[HTML]{000000} 0.03 & {\cellcolor[HTML]{E4EFF6}} \color[HTML]{000000} 0.02 & {\cellcolor[HTML]{D2E2EE}} \color[HTML]{000000} 0.04 & {\cellcolor[HTML]{DDEAF3}} \color[HTML]{000000} 0.03 \\
 &  & Mpt 7B & {\cellcolor[HTML]{D1E1EE}} \color[HTML]{000000} 0.04 & {\cellcolor[HTML]{E2EDF5}} \color[HTML]{000000} 0.03 & {\cellcolor[HTML]{F3F9FC}} \color[HTML]{000000} 0.01 & {\cellcolor[HTML]{E0ECF4}} \color[HTML]{000000} 0.03 \\
 &  & Mpt 30B Chat & {\cellcolor[HTML]{CFDFED}} \color[HTML]{000000} 0.04 & {\cellcolor[HTML]{EEF5F9}} \color[HTML]{000000} 0.01 & {\cellcolor[HTML]{F7FCFD}} \color[HTML]{000000} 0.00 & {\cellcolor[HTML]{E4EFF6}} \color[HTML]{000000} 0.02 \\
 &  & Vicuna 13B V1 & {\cellcolor[HTML]{EEF5F9}} \color[HTML]{000000} 0.01 & {\cellcolor[HTML]{EEF6FA}} \color[HTML]{000000} 0.01 & {\cellcolor[HTML]{F1F7FA}} \color[HTML]{000000} 0.01 & {\cellcolor[HTML]{EFF6FA}} \color[HTML]{000000} 0.01 \\
 &  & Vicuna 33B V1 & {\cellcolor[HTML]{F3F9FC}} \color[HTML]{000000} 0.01 & {\cellcolor[HTML]{F4FAFC}} \color[HTML]{000000} 0.01 & {\cellcolor[HTML]{F4FAFC}} \color[HTML]{000000} 0.01 & {\cellcolor[HTML]{F2F8FB}} \color[HTML]{000000} 0.01 \\
 &  & Wizard Vicuna 13B Uncensored HF & {\cellcolor[HTML]{F0F7FA}} \color[HTML]{000000} 0.01 & {\cellcolor[HTML]{F3F9FB}} \color[HTML]{000000} 0.01 & {\cellcolor[HTML]{F1F7FA}} \color[HTML]{000000} 0.01 & {\cellcolor[HTML]{EAF3F8}} \color[HTML]{000000} 0.02 \\
\cmidrule{2-7}
 & \multirow[c]{9}{*}{\thead{With\\ Guidelines}} & Falcon 7B & {\cellcolor[HTML]{EAF3F8}} \color[HTML]{000000} 0.02 & {\cellcolor[HTML]{C3D6E8}} \color[HTML]{000000} 0.05 & {\cellcolor[HTML]{D3E2EF}} \color[HTML]{000000} 0.04 & {\cellcolor[HTML]{DBE8F2}} \color[HTML]{000000} 0.03 \\
 &  & Falcon 40B Instruct & {\cellcolor[HTML]{4D004B}} \color[HTML]{F1F1F1} 0.21 & {\cellcolor[HTML]{B5CCE3}} \color[HTML]{000000} 0.06 & {\cellcolor[HTML]{C5D8E9}} \color[HTML]{000000} 0.05 & {\cellcolor[HTML]{C8DAEA}} \color[HTML]{000000} 0.05 \\
 &  & ChatGPT 3.5 turbo & {\cellcolor[HTML]{E1ECF4}} \color[HTML]{000000} 0.03 & {\cellcolor[HTML]{E1ECF4}} \color[HTML]{000000} 0.03 & {\cellcolor[HTML]{DAE7F1}} \color[HTML]{000000} 0.03 & {\cellcolor[HTML]{DCE9F2}} \color[HTML]{000000} 0.03 \\
 &  & ChatGPT 4 & {\cellcolor[HTML]{D1E1EE}} \color[HTML]{000000} 0.04 & {\cellcolor[HTML]{D7E5F0}} \color[HTML]{000000} 0.04 & {\cellcolor[HTML]{CFDFED}} \color[HTML]{000000} 0.04 & {\cellcolor[HTML]{DDEAF3}} \color[HTML]{000000} 0.03 \\
 &  & Mpt 7B & {\cellcolor[HTML]{E4EFF6}} \color[HTML]{000000} 0.02 & {\cellcolor[HTML]{E1ECF4}} \color[HTML]{000000} 0.03 & {\cellcolor[HTML]{000000}} \color[HTML]{F1F1F1} nan & {\cellcolor[HTML]{000000}} \color[HTML]{F1F1F1} nan \\
 &  & Mpt 30B Chat & {\cellcolor[HTML]{E4EFF6}} \color[HTML]{000000} 0.02 & {\cellcolor[HTML]{F1F8FB}} \color[HTML]{000000} 0.01 & {\cellcolor[HTML]{EFF6FA}} \color[HTML]{000000} 0.01 & {\cellcolor[HTML]{EEF5F9}} \color[HTML]{000000} 0.01 \\
 &  & Vicuna 13B V1 & {\cellcolor[HTML]{E7F1F7}} \color[HTML]{000000} 0.02 & {\cellcolor[HTML]{DFEBF4}} \color[HTML]{000000} 0.03 & {\cellcolor[HTML]{E8F1F7}} \color[HTML]{000000} 0.02 & {\cellcolor[HTML]{EEF6FA}} \color[HTML]{000000} 0.01 \\
 &  & Vicuna 33B V1 & {\cellcolor[HTML]{EDF5F9}} \color[HTML]{000000} 0.01 & {\cellcolor[HTML]{D3E2EF}} \color[HTML]{000000} 0.04 & {\cellcolor[HTML]{EFF6FA}} \color[HTML]{000000} 0.01 & {\cellcolor[HTML]{E9F2F8}} \color[HTML]{000000} 0.02 \\
 &  & Wizard Vicuna 13B Uncensored HF & {\cellcolor[HTML]{EBF4F8}} \color[HTML]{000000} 0.02 & {\cellcolor[HTML]{F5FAFC}} \color[HTML]{000000} 0.01 & {\cellcolor[HTML]{E9F2F7}} \color[HTML]{000000} 0.02 & {\cellcolor[HTML]{F3F9FC}} \color[HTML]{000000} 0.01 \\
 \midrule
\multirow[c]{18}{*}{\thead{With\\ Colour}} & \multirow[c]{9}{*}{\thead{Without\\ Guidelines}} & Falcon 7B & {\cellcolor[HTML]{770C73}} \color[HTML]{F1F1F1} 0.19 & {\cellcolor[HTML]{8C91C4}} \color[HTML]{F1F1F1} 0.11 & {\cellcolor[HTML]{C2D5E7}} \color[HTML]{000000} 0.05 & {\cellcolor[HTML]{BFD3E6}} \color[HTML]{000000} 0.06 \\
 &  & Falcon 40B Instruct & {\cellcolor[HTML]{D8E6F0}} \color[HTML]{000000} 0.04 & {\cellcolor[HTML]{E4EEF5}} \color[HTML]{000000} 0.03 & {\cellcolor[HTML]{E4EFF6}} \color[HTML]{000000} 0.02 & {\cellcolor[HTML]{CEDFEC}} \color[HTML]{000000} 0.04 \\
 &  & ChatGPT 3.5 turbo & {\cellcolor[HTML]{EEF6FA}} \color[HTML]{000000} 0.01 & {\cellcolor[HTML]{F2F8FB}} \color[HTML]{000000} 0.01 & {\cellcolor[HTML]{D6E4F0}} \color[HTML]{000000} 0.04 & {\cellcolor[HTML]{DAE7F1}} \color[HTML]{000000} 0.03 \\
 &  & ChatGPT 4 & {\cellcolor[HTML]{E1EDF5}} \color[HTML]{000000} 0.03 & {\cellcolor[HTML]{E3EEF5}} \color[HTML]{000000} 0.03 & {\cellcolor[HTML]{D2E2EE}} \color[HTML]{000000} 0.04 & {\cellcolor[HTML]{DDEAF3}} \color[HTML]{000000} 0.03 \\
 &  & Mpt 7B & {\cellcolor[HTML]{D4E3EF}} \color[HTML]{000000} 0.04 & {\cellcolor[HTML]{95AAD0}} \color[HTML]{F1F1F1} 0.09 & {\cellcolor[HTML]{F6FBFC}} \color[HTML]{000000} 0.00 & {\cellcolor[HTML]{EBF4F8}} \color[HTML]{000000} 0.02 \\
 &  & Mpt 30B Chat & {\cellcolor[HTML]{E6F0F6}} \color[HTML]{000000} 0.02 & {\cellcolor[HTML]{E2EDF5}} \color[HTML]{000000} 0.03 & {\cellcolor[HTML]{F3F9FB}} \color[HTML]{000000} 0.01 & {\cellcolor[HTML]{C2D5E7}} \color[HTML]{000000} 0.05 \\
 &  & Vicuna 13B V1 & {\cellcolor[HTML]{F1F7FA}} \color[HTML]{000000} 0.01 & {\cellcolor[HTML]{EBF4F8}} \color[HTML]{000000} 0.02 & {\cellcolor[HTML]{F7FCFD}} \color[HTML]{000000} 0.00 & {\cellcolor[HTML]{EBF4F8}} \color[HTML]{000000} 0.02 \\
 &  & Vicuna 33B V1 & {\cellcolor[HTML]{F0F7FA}} \color[HTML]{000000} 0.01 & {\cellcolor[HTML]{EBF4F8}} \color[HTML]{000000} 0.02 & {\cellcolor[HTML]{F3F9FC}} \color[HTML]{000000} 0.01 & {\cellcolor[HTML]{EBF4F8}} \color[HTML]{000000} 0.02 \\
 &  & Wizard Vicuna 13B Uncensored HF & {\cellcolor[HTML]{F1F8FB}} \color[HTML]{000000} 0.01 & {\cellcolor[HTML]{F2F8FB}} \color[HTML]{000000} 0.01 & {\cellcolor[HTML]{F5FAFC}} \color[HTML]{000000} 0.00 & {\cellcolor[HTML]{F2F8FB}} \color[HTML]{000000} 0.01 \\
 \cmidrule{2-7}
 & \multirow[c]{9}{*}{\thead{With\\ Guidelines}} & Falcon 7B & {\cellcolor[HTML]{E8F1F7}} \color[HTML]{000000} 0.02 & {\cellcolor[HTML]{CFDFED}} \color[HTML]{000000} 0.04 & {\cellcolor[HTML]{EAF3F8}} \color[HTML]{000000} 0.02 & {\cellcolor[HTML]{E8F1F7}} \color[HTML]{000000} 0.02 \\
 &  & Falcon 40B Instruct & {\cellcolor[HTML]{7B0D76}} \color[HTML]{F1F1F1} 0.19 & {\cellcolor[HTML]{8C7AB8}} \color[HTML]{F1F1F1} 0.13 & {\cellcolor[HTML]{821580}} \color[HTML]{F1F1F1} 0.18 & {\cellcolor[HTML]{852B8F}} \color[HTML]{F1F1F1} 0.17 \\
 &  & ChatGPT 3.5 turbo & {\cellcolor[HTML]{DFEBF4}} \color[HTML]{000000} 0.03 & {\cellcolor[HTML]{DAE7F1}} \color[HTML]{000000} 0.03 & {\cellcolor[HTML]{D8E6F0}} \color[HTML]{000000} 0.04 & {\cellcolor[HTML]{D7E5F0}} \color[HTML]{000000} 0.04 \\
 &  & ChatGPT 4 & {\cellcolor[HTML]{D3E2EF}} \color[HTML]{000000} 0.04 & {\cellcolor[HTML]{D9E6F1}} \color[HTML]{000000} 0.03 & {\cellcolor[HTML]{CCDDEC}} \color[HTML]{000000} 0.04 & {\cellcolor[HTML]{DAE7F1}} \color[HTML]{000000} 0.03 \\
 &  & Mpt 7B & {\cellcolor[HTML]{EAF3F8}} \color[HTML]{000000} 0.02 & {\cellcolor[HTML]{D6E4F0}} \color[HTML]{000000} 0.04 & {\cellcolor[HTML]{000000}} \color[HTML]{F1F1F1} nan & {\cellcolor[HTML]{000000}} \color[HTML]{F1F1F1} nan \\
 &  & Mpt 30B Chat & {\cellcolor[HTML]{D6E4F0}} \color[HTML]{000000} 0.04 & {\cellcolor[HTML]{BCD1E5}} \color[HTML]{000000} 0.06 & {\cellcolor[HTML]{000000}} \color[HTML]{F1F1F1} nan & {\cellcolor[HTML]{000000}} \color[HTML]{F1F1F1} nan \\
 &  & Vicuna 13B V1 & {\cellcolor[HTML]{E2EDF5}} \color[HTML]{000000} 0.03 & {\cellcolor[HTML]{E5EFF6}} \color[HTML]{000000} 0.02 & {\cellcolor[HTML]{ECF4F9}} \color[HTML]{000000} 0.02 & {\cellcolor[HTML]{F6FBFC}} \color[HTML]{000000} 0.00 \\
 &  & Vicuna 33B V1 & {\cellcolor[HTML]{DCE9F2}} \color[HTML]{000000} 0.03 & {\cellcolor[HTML]{EEF5F9}} \color[HTML]{000000} 0.01 & {\cellcolor[HTML]{F5FAFC}} \color[HTML]{000000} 0.00 & {\cellcolor[HTML]{ECF4F9}} \color[HTML]{000000} 0.02 \\
 &  & Wizard Vicuna 13B Uncensored HF & {\cellcolor[HTML]{EEF6FA}} \color[HTML]{000000} 0.01 & {\cellcolor[HTML]{EDF5F9}} \color[HTML]{000000} 0.01 & {\cellcolor[HTML]{F2F8FB}} \color[HTML]{000000} 0.01 & {\cellcolor[HTML]{F3F9FB}} \color[HTML]{000000} 0.01 \\
\bottomrule
\end{tabular}
            
\end{table}

A comparison of the models' token distributions against the reference human distribution was also made. In this case, the Wasserstein distance \cite{wasserstein} was used as it is a common divergence metric. In Table \ref{tab:personal-narrative-elicitation-wasserstein} are reported the metrics calculated. Unfortunately, this coarse approach was unable to highlight real differences, and for most models, their distribution is very similar to the reference human distributions. Falcon models mark an exception, as their divergence is quite high. Deeper inspection reveals that, indeed, their distributions do not match at all the human distributions. 
\begin{figure}[!htbp]
    \centering
    \captionsetup[subfigure]{oneside,margin={0cm,2cm}}%
    \begin{subfigure}[t]{0.25\textwidth}
        \centering
        % [subfigure]}
        %,width=1\linewidth
        \includegraphics[height=18cm]{assets/imgs/dataset-test-set-top-50-answers-vertical.png}
        \caption{This Figure reports the top 50 most frequent tokens in the human crowdsourced elicitations.}
        \label{sub:persona-narrative-elicitation-comparison-distribution-human}
        % \includegraphics[height=10cm]{assets/imgs/tokens-vertical/token_distribution_no_color_no_guidelines_0_shot_mpt-7b.png}
    \end{subfigure}
    \hspace{-1.5cm}
    \captionsetup[subfigure]{oneside,margin={0cm,0cm}}
    \begin{subfigure}[t]{0.45\textwidth}
        \centering
        % \captionsetup{width=1\linewidth}%
        % \includegraphics[width=1\textwidth]{assets/imgs/dataset-test-set-top-30-answers-vertical.png}
        \includegraphics[height=18cm]{assets/imgs/tokens-vertical/no_color/no_guidelines/0_shot/token_distribution_no_color_no_guidelines_0_shot_falcon-7b.png}
        \caption{This Figure reports the top 50 most frequent tokens from the Falcon 7B model, prompted with no colour information, without guidelines and with 0 examples.}
        \label{sub:persona-narrative-elicitation-comparison-distribution-falcon}
    \end{subfigure}
    \hspace{-2cm}
    \captionsetup[subfigure]{oneside,margin={2.5cm,0cm}}
    \begin{subfigure}[t]{0.3\textwidth}
        \centering
        % \captionsetup{justification=raggedrigh, width=1\linewidth}%
        % \includegraphics[width=1\textwidth]{assets/imgs/dataset-test-set-top-30-answers-vertical.png}
        \includegraphics[height=18cm]{assets/imgs/tokens-vertical/with_color/with_guidelines/5_shot/token_distribution_with_color_with_guidelines_5_shot_gpt-4.png}
        \caption{This Figure reports the top 50 most frequent tokens from the ChatGPT 4 model, prompted with colour information, with guidelines and with 5 examples.}
            \label{sub:persona-narrative-elicitation-comparison-distribution-gpt-4}
    \end{subfigure}
    % \begin{subfigure}[b]{1\textwidth}
    %     \centering
    %     \includegraphics[width=1\textwidth]{assets/imgs/tokens/token_distribution_no_color_no_guidelines_0_shot_falcon-7b.png}
    % \end{subfigure}
    % \begin{subfigure}[b]{1\textwidth}
    %     \centering
    %     \includegraphics[width=1\textwidth]{assets/imgs/dataset-test-set-top-30-answers.png}
    % \end{subfigure}
     % \subfigure[]{\includegraphics[with=.5\linewidth]{assets/imgs/tokens/token_distribution_no_color_no_guidelines_0_shot_falcon-7b.png}}{}
    \caption{In Subfigure a), the top 50 most frequent token of the human crowdsourced elicitations. In Subfigures b) and c) two examples of distributions of two different models in different experimental settings. Notice how the distribution from Gpt-4 is much closer to the human one, in particular regarding to tokens like \emph{"dispiace"} and \emph{"figlia"}. All distributions are computed using Spacy.}
    \label{fig:persona-narrative-elicitation-comparison-distribution}
\end{figure}

In Figure \ref{fig:persona-narrative-elicitation-comparison-distribution} are represented the top 50 tokens of the reference human crowdsourced elicitations and two examples of the models in different experimental settings. Considering the origin of the dataset and the experimental setup in which it was gathered, we expect many tokens such as \emph{"dispiace"} and \emph{"capisco"} because they are used to convey empathy, which was one of the requirements in the guidelines. Other relevant tokens should regard topics of discussion present in the narrative. Due to the variety of the dataset, most tokens should appear only once. This is confirmed by the Subfigure \ref{sub:persona-narrative-elicitation-comparison-distribution-human}. 
In the Subfigure \ref{sub:persona-narrative-elicitation-comparison-distribution-gpt-4} it is possible to witness that ChatGPT 4 follows a similar distribution curve, with the most frequent tokens being words such as \emph{"dispiace"} and \emph{"sentire"} and then other tokens appear very rarely. 
In Subfigure \ref{sub:persona-narrative-elicitation-comparison-distribution-falcon} we can see that this model does not follow a similar distribution at all. This confirms our findings through the Wasserstein divergence from Table \ref{tab:personal-narrative-elicitation-wasserstein}.
\begin{table}[ht]
    \centering
    \caption{Averages Elicitations lengths and standard deviations across the tested models. Computed using the whitespace tokenizer}
    \label{tab:personal-narrative-elicitation-token-length-std}
\setlength{\tabcolsep}{4pt}
\begin{tabular}{l|l|l|rrrr|rrrr}
\toprule
\multicolumn{3}{r}{}  & \multicolumn{4}{c|}{\rotatebox[origin=l]{0}{\thead{Average \\token length}}} & \multicolumn{4}{c}{\rotatebox[origin=l]{0}{\thead{Standard deviation \\ token length}}} \\
\midrule
\multicolumn{3}{c|}{\thead{Human}} & \multicolumn{4}{c|}{{\cellcolor[HTML]{FEE9E6}} \color[HTML]{000000}  10.42 } & \multicolumn{4}{c}{{\cellcolor[HTML]{F5FBFC}} \color[HTML]{000000} 5.12} \\
\midrule
% \multirow{1}{*}{\rotatebox[origin=l]{0}{\thead{Colour}}} & \multirow{1}{*}{\rotatebox[origin=l]{0}{\thead{Guidelines}}} & \multirow{1}{*}{\rotatebox[origin=l]{0}{\thead{Model name}}} \\
\thead{Colour} & \thead{Guidelines} & \thead{Model name} & \thead{\rotatebox[origin=l]{0}{0-shot}} & \thead{\rotatebox[origin=l]{0}{1-shot}} & \thead{\rotatebox[origin=l]{0}{3-shot}} & \thead{\rotatebox[origin=l]{0}{5-shot}} & \thead{\rotatebox[origin=l]{0}{0-shot}} & \thead{\rotatebox[origin=l]{0}{1-shot}} & \thead{\rotatebox[origin=l]{0}{3-shot}} & \thead{\rotatebox[origin=l]{0}{5-shot}}  \\
\midrule
\multirow[c]{18}{*}{\rotatebox[origin=l]{0}{\thead{Without \\ Colour}}} & \multirow[c]{9}{*}{\rotatebox[origin=l]{0}{\thead{Without \\ Guidelines}}} & Falcon 7B & {\cellcolor[HTML]{650171}} \color[HTML]{F1F1F1} 59.60 & {\cellcolor[HTML]{E7489B}} \color[HTML]{F1F1F1} 39.34 & {\cellcolor[HTML]{F871A4}} \color[HTML]{F1F1F1} 33.53 & {\cellcolor[HTML]{FBBBBD}} \color[HTML]{000000} 22.36 & {\cellcolor[HTML]{00441B}} \color[HTML]{F1F1F1} 85.52 & {\cellcolor[HTML]{63C0A0}} \color[HTML]{000000} 45.50 & {\cellcolor[HTML]{9AD8CA}} \color[HTML]{000000} 34.04 & {\cellcolor[HTML]{CAEBE5}} \color[HTML]{000000} 24.43 \\
 &  & Falcon 40B Instruct  & {\cellcolor[HTML]{FDD8D5}} \color[HTML]{000000} 15.29 & {\cellcolor[HTML]{FCCBC6}} \color[HTML]{000000} 18.84 & {\cellcolor[HTML]{FCBFBE}} \color[HTML]{000000} 21.59 & {\cellcolor[HTML]{FDD3CF}} \color[HTML]{000000} 16.67 & {\cellcolor[HTML]{EBF7FA}} \color[HTML]{000000} 10.81 & {\cellcolor[HTML]{E0F3F5}} \color[HTML]{000000} 15.95 & {\cellcolor[HTML]{E2F4F7}} \color[HTML]{000000} 15.20 & {\cellcolor[HTML]{E6F5F9}} \color[HTML]{000000} 13.31 \\
 &  & Gpt 3 & {\cellcolor[HTML]{FDDEDB}} \color[HTML]{000000} 13.66 & {\cellcolor[HTML]{FDE1DE}} \color[HTML]{000000} 12.93 & {\cellcolor[HTML]{FCCECA}} \color[HTML]{000000} 18.03 & {\cellcolor[HTML]{FCC6C2}} \color[HTML]{000000} 20.10 & {\cellcolor[HTML]{F6FCFD}} \color[HTML]{000000} 4.22 & {\cellcolor[HTML]{F5FBFD}} \color[HTML]{000000} 4.65 & {\cellcolor[HTML]{F1FAFC}} \color[HTML]{000000} 7.02 & {\cellcolor[HTML]{F2FAFC}} \color[HTML]{000000} 6.65 \\
 &  & Gpt 4 & {\cellcolor[HTML]{FCD1CD}} \color[HTML]{000000} 17.24 & {\cellcolor[HTML]{FDD4D0}} \color[HTML]{000000} 16.41 & {\cellcolor[HTML]{FCC6C1}} \color[HTML]{000000} 20.34 & {\cellcolor[HTML]{FCD0CC}} \color[HTML]{000000} 17.50 & {\cellcolor[HTML]{F6FCFD}} \color[HTML]{000000} 4.17 & {\cellcolor[HTML]{F5FBFD}} \color[HTML]{000000} 4.71 & {\cellcolor[HTML]{F6FCFD}} \color[HTML]{000000} 4.37 & {\cellcolor[HTML]{F4FBFC}} \color[HTML]{000000} 5.64 \\
 &  & Mpt 7B & {\cellcolor[HTML]{FBB0BA}} \color[HTML]{000000} 24.43 & {\cellcolor[HTML]{FCC7C3}} \color[HTML]{000000} 19.83 & {\cellcolor[HTML]{FDE1DE}} \color[HTML]{000000} 13.14 & {\cellcolor[HTML]{FBBABD}} \color[HTML]{000000} 22.43 & {\cellcolor[HTML]{8DD3C0}} \color[HTML]{000000} 36.85 & {\cellcolor[HTML]{BAE5DC}} \color[HTML]{000000} 27.72 & {\cellcolor[HTML]{EBF7FA}} \color[HTML]{000000} 10.77 & {\cellcolor[HTML]{D1EEE9}} \color[HTML]{000000} 22.27 \\
 &  & Mpt 30B Chat & {\cellcolor[HTML]{FCCECA}} \color[HTML]{000000} 17.98 & {\cellcolor[HTML]{FDE1DE}} \color[HTML]{000000} 13.12 & {\cellcolor[HTML]{FDE6E2}} \color[HTML]{000000} 11.36 & {\cellcolor[HTML]{FCD1CD}} \color[HTML]{000000} 17.09 & {\cellcolor[HTML]{B0E1D6}} \color[HTML]{000000} 29.83 & {\cellcolor[HTML]{E0F3F5}} \color[HTML]{000000} 16.04 & {\cellcolor[HTML]{E8F6FA}} \color[HTML]{000000} 12.09 & {\cellcolor[HTML]{DEF2F4}} \color[HTML]{000000} 16.80 \\
 &  & Vicuna 13B V1 & {\cellcolor[HTML]{FDDDDA}} \color[HTML]{000000} 13.91 & {\cellcolor[HTML]{FDDFDC}} \color[HTML]{000000} 13.48 & {\cellcolor[HTML]{FDE0DD}} \color[HTML]{000000} 13.21 & {\cellcolor[HTML]{FDDDDA}} \color[HTML]{000000} 13.90 & {\cellcolor[HTML]{F2FAFC}} \color[HTML]{000000} 6.27 & {\cellcolor[HTML]{ECF8FA}} \color[HTML]{000000} 10.03 & {\cellcolor[HTML]{F0F9FB}} \color[HTML]{000000} 7.73 & {\cellcolor[HTML]{E9F7FA}} \color[HTML]{000000} 11.65 \\
 &  & Vicuna 33B V1 & {\cellcolor[HTML]{FDE5E2}} \color[HTML]{000000} 11.69 & {\cellcolor[HTML]{FEE9E6}} \color[HTML]{000000} 10.28 & {\cellcolor[HTML]{FDE5E2}} \color[HTML]{000000} 11.59 & {\cellcolor[HTML]{FDE3E0}} \color[HTML]{000000} 12.43 & {\cellcolor[HTML]{F7FCFD}} \color[HTML]{000000} 3.62 & {\cellcolor[HTML]{F7FCFD}} \color[HTML]{000000} 3.91 & {\cellcolor[HTML]{F2FAFC}} \color[HTML]{000000} 6.64 & {\cellcolor[HTML]{EBF7FA}} \color[HTML]{000000} 10.76 \\
 &  & Wizard Vicuna 13B  & {\cellcolor[HTML]{FDE1DE}} \color[HTML]{000000} 12.83 & {\cellcolor[HTML]{FDE5E2}} \color[HTML]{000000} 11.62 & {\cellcolor[HTML]{FDDDDA}} \color[HTML]{000000} 14.03 & {\cellcolor[HTML]{FDDBD7}} \color[HTML]{000000} 14.72 & {\cellcolor[HTML]{F5FBFC}} \color[HTML]{000000} 4.95 & {\cellcolor[HTML]{ECF8FA}} \color[HTML]{000000} 10.33 & {\cellcolor[HTML]{EFF9FB}} \color[HTML]{000000} 8.45 & {\cellcolor[HTML]{E7F6F9}} \color[HTML]{000000} 13.08 \\

 \cmidrule{2-11}
 & \multirow[c]{9}{*}{\rotatebox[origin=lenter]{0}{\thead{With \\ Guidelines}}}& Falcon 7B & {\cellcolor[HTML]{FCCDC9}} \color[HTML]{000000} 18.19 & {\cellcolor[HTML]{FBAFBA}} \color[HTML]{000000} 24.53 & {\cellcolor[HTML]{FBBABD}} \color[HTML]{000000} 22.47 & {\cellcolor[HTML]{FBB6BC}} \color[HTML]{000000} 23.10 & {\cellcolor[HTML]{D1EEEA}} \color[HTML]{000000} 21.96 & {\cellcolor[HTML]{CDECE7}} \color[HTML]{000000} 23.69 & {\cellcolor[HTML]{CEEDE8}} \color[HTML]{000000} 23.22 & {\cellcolor[HTML]{BFE7DE}} \color[HTML]{000000} 26.68 \\
 &  & Falcon 40B Instruct  & {\cellcolor[HTML]{E84A9B}} \color[HTML]{F1F1F1} 38.97 & {\cellcolor[HTML]{FCBEBE}} \color[HTML]{000000} 21.93 & {\cellcolor[HTML]{FCC4C0}} \color[HTML]{000000} 20.71 & {\cellcolor[HTML]{FCC2BF}} \color[HTML]{000000} 20.88 & {\cellcolor[HTML]{9DDACB}} \color[HTML]{000000} 33.39 & {\cellcolor[HTML]{C5E9E2}} \color[HTML]{000000} 25.56 & {\cellcolor[HTML]{DBF2F2}} \color[HTML]{000000} 17.81 & {\cellcolor[HTML]{CEEDE8}} \color[HTML]{000000} 23.29 \\
 &  & Gpt 3 & {\cellcolor[HTML]{FCD0CC}} \color[HTML]{000000} 17.67 & {\cellcolor[HTML]{FCCCC7}} \color[HTML]{000000} 18.72 & {\cellcolor[HTML]{FCC6C2}} \color[HTML]{000000} 20.07 & {\cellcolor[HTML]{FCC6C1}} \color[HTML]{000000} 20.17 & {\cellcolor[HTML]{F1FAFC}} \color[HTML]{000000} 7.09 & {\cellcolor[HTML]{F0F9FB}} \color[HTML]{000000} 7.53 & {\cellcolor[HTML]{EEF8FB}} \color[HTML]{000000} 8.90 & {\cellcolor[HTML]{F1FAFC}} \color[HTML]{000000} 7.44 \\
 &  & Gpt 4 & {\cellcolor[HTML]{FCC6C2}} \color[HTML]{000000} 20.03 & {\cellcolor[HTML]{FCC8C3}} \color[HTML]{000000} 19.57 & {\cellcolor[HTML]{FCC0BF}} \color[HTML]{000000} 21.31 & {\cellcolor[HTML]{FCCECA}} \color[HTML]{000000} 17.98 & {\cellcolor[HTML]{F5FBFD}} \color[HTML]{000000} 4.76 & {\cellcolor[HTML]{F5FBFD}} \color[HTML]{000000} 4.71 & {\cellcolor[HTML]{F5FBFD}} \color[HTML]{000000} 4.80 & {\cellcolor[HTML]{F5FBFC}} \color[HTML]{000000} 5.11 \\
 &  & Mpt 7B & {\cellcolor[HTML]{FDD6D2}} \color[HTML]{000000} 16.02 & {\cellcolor[HTML]{FCC9C4}} \color[HTML]{000000} 19.40 & {\cellcolor[HTML]{000000}} \color[HTML]{F1F1F1} nan & {\cellcolor[HTML]{000000}} \color[HTML]{F1F1F1} nan & {\cellcolor[HTML]{E1F4F6}} \color[HTML]{000000} 15.66 & {\cellcolor[HTML]{D9F1F0}} \color[HTML]{000000} 18.86 & {\cellcolor[HTML]{000000}} \color[HTML]{F1F1F1} nan & {\cellcolor[HTML]{000000}} \color[HTML]{F1F1F1} nan \\
 &  & Mpt 30B Chat & {\cellcolor[HTML]{FDD8D5}} \color[HTML]{000000} 15.21 & {\cellcolor[HTML]{FEE7E4}} \color[HTML]{000000} 11.10 & {\cellcolor[HTML]{FDD8D5}} \color[HTML]{000000} 15.34 & {\cellcolor[HTML]{FDD9D6}} \color[HTML]{000000} 15.12 & {\cellcolor[HTML]{E1F4F6}} \color[HTML]{000000} 15.53 & {\cellcolor[HTML]{EEF8FB}} \color[HTML]{000000} 8.78 & {\cellcolor[HTML]{E8F6FA}} \color[HTML]{000000} 12.57 & {\cellcolor[HTML]{E8F6FA}} \color[HTML]{000000} 12.47 \\
 &  & Vicuna 13B V1 & {\cellcolor[HTML]{FDD5D1}} \color[HTML]{000000} 16.19 & {\cellcolor[HTML]{FCCDC9}} \color[HTML]{000000} 18.26 & {\cellcolor[HTML]{FDD7D4}} \color[HTML]{000000} 15.50 & {\cellcolor[HTML]{FDE0DD}} \color[HTML]{000000} 13.16 & {\cellcolor[HTML]{E1F4F6}} \color[HTML]{000000} 15.51 & {\cellcolor[HTML]{D8F0EF}} \color[HTML]{000000} 19.03 & {\cellcolor[HTML]{E7F6F9}} \color[HTML]{000000} 12.96 & {\cellcolor[HTML]{E6F5F9}} \color[HTML]{000000} 13.38 \\
 &  & Vicuna 33B V1 & {\cellcolor[HTML]{FDE1DE}} \color[HTML]{000000} 12.81 & {\cellcolor[HTML]{FCCBC6}} \color[HTML]{000000} 18.81 & {\cellcolor[HTML]{FDDEDB}} \color[HTML]{000000} 13.83 & {\cellcolor[HTML]{FDD9D6}} \color[HTML]{000000} 15.12 & {\cellcolor[HTML]{E1F4F6}} \color[HTML]{000000} 15.64 & {\cellcolor[HTML]{DFF3F4}} \color[HTML]{000000} 16.66 & {\cellcolor[HTML]{EBF7FA}} \color[HTML]{000000} 10.82 & {\cellcolor[HTML]{E8F6FA}} \color[HTML]{000000} 12.27 \\
 &  & Wizard Vicuna 13B  & {\cellcolor[HTML]{FFF3F0}} \color[HTML]{000000} 7.26 & {\cellcolor[HTML]{FEE7E4}} \color[HTML]{000000} 10.95 & {\cellcolor[HTML]{FCD1CD}} \color[HTML]{000000} 17.00 & {\cellcolor[HTML]{FDE4E0}} \color[HTML]{000000} 12.07 & {\cellcolor[HTML]{EDF8FB}} \color[HTML]{000000} 9.33 & {\cellcolor[HTML]{E9F7FA}} \color[HTML]{000000} 11.77 & {\cellcolor[HTML]{E3F4F8}} \color[HTML]{000000} 14.68 & {\cellcolor[HTML]{EDF8FB}} \color[HTML]{000000} 9.31 \\

\midrule
\multirow[c]{18}{*}{\rotatebox[origin=l]{0}{\thead{With\\ Colour}}} & \multirow[c]{9}{*}{\rotatebox[origin=l]{0}{\thead{Without \\ Guidelines}}} & Falcon 7B & {\cellcolor[HTML]{49006A}} \color[HTML]{F1F1F1} 63.81 & {\cellcolor[HTML]{DC3397}} \color[HTML]{F1F1F1} 42.24 & {\cellcolor[HTML]{FBB8BC}} \color[HTML]{000000} 23.09 & {\cellcolor[HTML]{F98BAE}} \color[HTML]{000000} 30.31 & {\cellcolor[HTML]{006729}} \color[HTML]{F1F1F1} 76.58 & {\cellcolor[HTML]{6AC4A7}} \color[HTML]{000000} 43.73 & {\cellcolor[HTML]{C2E8E0}} \color[HTML]{000000} 26.12 & {\cellcolor[HTML]{B5E3D9}} \color[HTML]{000000} 28.61 \\
 &  & Falcon 40B Instruct  & {\cellcolor[HTML]{FCCECA}} \color[HTML]{000000} 17.93 & {\cellcolor[HTML]{FDD7D3}} \color[HTML]{000000} 15.81 & {\cellcolor[HTML]{FDD5D1}} \color[HTML]{000000} 16.22 & {\cellcolor[HTML]{FCCCC7}} \color[HTML]{000000} 18.72 & {\cellcolor[HTML]{DFF3F4}} \color[HTML]{000000} 16.56 & {\cellcolor[HTML]{E8F6FA}} \color[HTML]{000000} 12.21 & {\cellcolor[HTML]{EAF7FA}} \color[HTML]{000000} 11.22 & {\cellcolor[HTML]{DEF2F4}} \color[HTML]{000000} 16.97 \\
 &  & Gpt 3 & {\cellcolor[HTML]{FDDCD8}} \color[HTML]{000000} 14.34 & {\cellcolor[HTML]{FDE4E0}} \color[HTML]{000000} 12.24 & {\cellcolor[HTML]{FCC1BF}} \color[HTML]{000000} 21.28 & {\cellcolor[HTML]{FCC4C0}} \color[HTML]{000000} 20.69 & {\cellcolor[HTML]{F5FBFD}} \color[HTML]{000000} 4.60 & {\cellcolor[HTML]{F5FBFC}} \color[HTML]{000000} 5.10 & {\cellcolor[HTML]{F2FAFC}} \color[HTML]{000000} 6.51 & {\cellcolor[HTML]{F4FBFC}} \color[HTML]{000000} 5.65 \\
 &  & Gpt 4 & {\cellcolor[HTML]{FCD1CD}} \color[HTML]{000000} 17.17 & {\cellcolor[HTML]{FCD2CE}} \color[HTML]{000000} 16.78 & {\cellcolor[HTML]{FCC4C0}} \color[HTML]{000000} 20.71 & {\cellcolor[HTML]{FCD0CC}} \color[HTML]{000000} 17.62 & {\cellcolor[HTML]{F5FBFC}} \color[HTML]{000000} 4.96 & {\cellcolor[HTML]{F4FBFC}} \color[HTML]{000000} 5.47 & {\cellcolor[HTML]{F4FBFC}} \color[HTML]{000000} 5.34 & {\cellcolor[HTML]{F4FBFC}} \color[HTML]{000000} 5.52 \\
 &  & Mpt 7B & {\cellcolor[HTML]{FBADB9}} \color[HTML]{000000} 24.97 & {\cellcolor[HTML]{EA4D9C}} \color[HTML]{F1F1F1} 38.67 & {\cellcolor[HTML]{FDE3E0}} \color[HTML]{000000} 12.33 & {\cellcolor[HTML]{FCD2CE}} \color[HTML]{000000} 16.83 & {\cellcolor[HTML]{6FC6AA}} \color[HTML]{000000} 42.69 & {\cellcolor[HTML]{9CD9CA}} \color[HTML]{000000} 33.74 & {\cellcolor[HTML]{E8F6FA}} \color[HTML]{000000} 12.51 & {\cellcolor[HTML]{D8F0EF}} \color[HTML]{000000} 19.27 \\
 &  & Mpt 30B Chat & {\cellcolor[HTML]{FDD9D6}} \color[HTML]{000000} 15.16 & {\cellcolor[HTML]{FCCFCB}} \color[HTML]{000000} 17.78 & {\cellcolor[HTML]{FEE9E6}} \color[HTML]{000000} 10.41 & {\cellcolor[HTML]{FBB9BC}} \color[HTML]{000000} 22.81 & {\cellcolor[HTML]{D3EEEB}} \color[HTML]{000000} 21.39 & {\cellcolor[HTML]{DDF2F3}} \color[HTML]{000000} 17.07 & {\cellcolor[HTML]{ECF8FB}} \color[HTML]{000000} 9.88 & {\cellcolor[HTML]{DBF1F1}} \color[HTML]{000000} 18.23 \\
 &  & Vicuna 13B V1 & {\cellcolor[HTML]{FDE0DD}} \color[HTML]{000000} 13.33 & {\cellcolor[HTML]{FDD7D3}} \color[HTML]{000000} 15.83 & {\cellcolor[HTML]{FEE9E6}} \color[HTML]{000000} 10.40 & {\cellcolor[HTML]{FDD7D3}} \color[HTML]{000000} 15.78 & {\cellcolor[HTML]{F4FBFC}} \color[HTML]{000000} 5.43 & {\cellcolor[HTML]{E3F4F8}} \color[HTML]{000000} 14.56 & {\cellcolor[HTML]{F0F9FB}} \color[HTML]{000000} 7.99 & {\cellcolor[HTML]{E5F5F9}} \color[HTML]{000000} 13.90 \\
 &  & Vicuna 33B V1 & {\cellcolor[HTML]{FDDFDC}} \color[HTML]{000000} 13.59 & {\cellcolor[HTML]{FDD7D3}} \color[HTML]{000000} 15.69 & {\cellcolor[HTML]{FEEAE7}} \color[HTML]{000000} 10.03 & {\cellcolor[HTML]{FDDFDC}} \color[HTML]{000000} 13.40 & {\cellcolor[HTML]{EDF8FB}} \color[HTML]{000000} 9.06 & {\cellcolor[HTML]{E2F4F7}} \color[HTML]{000000} 15.28 & {\cellcolor[HTML]{E9F7FA}} \color[HTML]{000000} 11.88 & {\cellcolor[HTML]{E6F5F9}} \color[HTML]{000000} 13.24 \\
 &  & Wizard Vicuna 13B  & {\cellcolor[HTML]{FDDFDC}} \color[HTML]{000000} 13.47 & {\cellcolor[HTML]{FEE9E6}} \color[HTML]{000000} 10.31 & {\cellcolor[HTML]{FEE7E4}} \color[HTML]{000000} 10.91 & {\cellcolor[HTML]{FDDDDA}} \color[HTML]{000000} 13.97 & {\cellcolor[HTML]{F4FBFC}} \color[HTML]{000000} 5.47 & {\cellcolor[HTML]{F4FBFC}} \color[HTML]{000000} 5.60 & {\cellcolor[HTML]{EDF8FB}} \color[HTML]{000000} 9.67 & {\cellcolor[HTML]{EFF9FB}} \color[HTML]{000000} 8.73 \\

  \cmidrule{2-11}
 & \multirow[c]{9}{*}{\rotatebox[origin=l]{0}{\thead{With \\ Guidelines}}} & Falcon 7B & {\cellcolor[HTML]{FCD1CD}} \color[HTML]{000000} 17.26 & {\cellcolor[HTML]{FCBCBD}} \color[HTML]{000000} 22.14 & {\cellcolor[HTML]{FCD2CE}} \color[HTML]{000000} 16.78 & {\cellcolor[HTML]{FCD0CC}} \color[HTML]{000000} 17.50 & {\cellcolor[HTML]{D8F0EF}} \color[HTML]{000000} 19.28 & {\cellcolor[HTML]{CDECE6}} \color[HTML]{000000} 23.88 & {\cellcolor[HTML]{D8F0EF}} \color[HTML]{000000} 19.11 & {\cellcolor[HTML]{D1EEE9}} \color[HTML]{000000} 22.21 \\
 &  & Falcon 40B Instruct  & {\cellcolor[HTML]{EA4F9C}} \color[HTML]{F1F1F1} 38.47 & {\cellcolor[HTML]{FA9EB5}} \color[HTML]{000000} 27.67 & {\cellcolor[HTML]{F25D9F}} \color[HTML]{F1F1F1} 36.38 & {\cellcolor[HTML]{E94B9C}} \color[HTML]{F1F1F1} 38.79 & {\cellcolor[HTML]{94D6C5}} \color[HTML]{000000} 35.43 & {\cellcolor[HTML]{A4DCCF}} \color[HTML]{000000} 32.40 & {\cellcolor[HTML]{97D7C7}} \color[HTML]{000000} 34.87 & {\cellcolor[HTML]{A8DED2}} \color[HTML]{000000} 31.30 \\
 &  & Gpt 3 & {\cellcolor[HTML]{FCCAC5}} \color[HTML]{000000} 19.22 & {\cellcolor[HTML]{FCC6C2}} \color[HTML]{000000} 20.02 & {\cellcolor[HTML]{FCC6C1}} \color[HTML]{000000} 20.22 & {\cellcolor[HTML]{FCC4C0}} \color[HTML]{000000} 20.67 & {\cellcolor[HTML]{EBF7FA}} \color[HTML]{000000} 10.52 & {\cellcolor[HTML]{EBF7FA}} \color[HTML]{000000} 10.38 & {\cellcolor[HTML]{F1FAFC}} \color[HTML]{000000} 6.93 & {\cellcolor[HTML]{F1FAFC}} \color[HTML]{000000} 6.93 \\
 &  & Gpt 4 & {\cellcolor[HTML]{FCC6C2}} \color[HTML]{000000} 20.05 & {\cellcolor[HTML]{FCCAC5}} \color[HTML]{000000} 19.03 & {\cellcolor[HTML]{FCBEBE}} \color[HTML]{000000} 21.84 & {\cellcolor[HTML]{FCCAC5}} \color[HTML]{000000} 19.09 & {\cellcolor[HTML]{F5FBFD}} \color[HTML]{000000} 4.64 & {\cellcolor[HTML]{F6FCFD}} \color[HTML]{000000} 4.49 & {\cellcolor[HTML]{F6FCFD}} \color[HTML]{000000} 4.32 & {\cellcolor[HTML]{F5FBFD}} \color[HTML]{000000} 4.65 \\
 &  & Mpt 7B & {\cellcolor[HTML]{FDD6D2}} \color[HTML]{000000} 15.91 & {\cellcolor[HTML]{FCD1CD}} \color[HTML]{000000} 17.19 & {\cellcolor[HTML]{000000}} \color[HTML]{F1F1F1} nan & {\cellcolor[HTML]{000000}} \color[HTML]{F1F1F1} nan & {\cellcolor[HTML]{DDF2F3}} \color[HTML]{000000} 17.35 & {\cellcolor[HTML]{D8F0EF}} \color[HTML]{000000} 19.21 & {\cellcolor[HTML]{000000}} \color[HTML]{F1F1F1} nan & {\cellcolor[HTML]{000000}} \color[HTML]{F1F1F1} nan \\
 &  & Mpt 30B Chat & {\cellcolor[HTML]{FCC1BF}} \color[HTML]{000000} 21.17 & {\cellcolor[HTML]{FBBABD}} \color[HTML]{000000} 22.59 & {\cellcolor[HTML]{000000}} \color[HTML]{F1F1F1} nan & {\cellcolor[HTML]{000000}} \color[HTML]{F1F1F1} nan & {\cellcolor[HTML]{D5EFED}} \color[HTML]{000000} 20.56 & {\cellcolor[HTML]{D4EFEC}} \color[HTML]{000000} 21.14 & {\cellcolor[HTML]{000000}} \color[HTML]{F1F1F1} nan & {\cellcolor[HTML]{000000}} \color[HTML]{F1F1F1} nan \\
 &  & Vicuna 13B V1 & {\cellcolor[HTML]{FCD2CE}} \color[HTML]{000000} 16.95 & {\cellcolor[HTML]{FDDDDA}} \color[HTML]{000000} 14.03 & {\cellcolor[HTML]{FDE0DD}} \color[HTML]{000000} 13.28 & {\cellcolor[HTML]{FEECE9}} \color[HTML]{000000} 9.31 & {\cellcolor[HTML]{DEF2F4}} \color[HTML]{000000} 17.02 & {\cellcolor[HTML]{E2F4F7}} \color[HTML]{000000} 15.41 & {\cellcolor[HTML]{E2F4F7}} \color[HTML]{000000} 15.35 & {\cellcolor[HTML]{F2FAFC}} \color[HTML]{000000} 6.27 \\
 &  & Vicuna 33B V1 & {\cellcolor[HTML]{FDD6D2}} \color[HTML]{000000} 16.05 & {\cellcolor[HTML]{FDE6E2}} \color[HTML]{000000} 11.38 & {\cellcolor[HTML]{FFF2EE}} \color[HTML]{000000} 7.66 & {\cellcolor[HTML]{FFF7F3}} \color[HTML]{000000} 5.91 & {\cellcolor[HTML]{D9F1F0}} \color[HTML]{000000} 18.74 & {\cellcolor[HTML]{E3F4F7}} \color[HTML]{000000} 15.11 & {\cellcolor[HTML]{E8F6FA}} \color[HTML]{000000} 12.12 & {\cellcolor[HTML]{EBF7FA}} \color[HTML]{000000} 10.64 \\
 &  & Wizard Vicuna 13B  & {\cellcolor[HTML]{FEEEEB}} \color[HTML]{000000} 8.72 & {\cellcolor[HTML]{FEF1ED}} \color[HTML]{000000} 7.88 & {\cellcolor[HTML]{FDE3E0}} \color[HTML]{000000} 12.45 & {\cellcolor[HTML]{FEECE9}} \color[HTML]{000000} 9.48 & {\cellcolor[HTML]{EBF7FA}} \color[HTML]{000000} 10.63 & {\cellcolor[HTML]{E7F6F9}} \color[HTML]{000000} 12.68 & {\cellcolor[HTML]{EAF7FA}} \color[HTML]{000000} 11.14 & {\cellcolor[HTML]{E9F7FA}} \color[HTML]{000000} 11.89 \\

\bottomrule

\end{tabular}
\setlength{\tabcolsep}{6pt}
\end{table}

% \begin{table}[!htbp]
    \centering
    \caption{Standard deviation on Elicitations lengths across the tested models. Computed using the whitespace tokenizer.}
    \label{tab:personal-narrative-elicitation-token-std}
    \begin{tabular}{l|l|l|rrrr}
    \toprule
\multicolumn{3}{r}{\thead{Standard Deviation Reference Human \\ Elicitation Token Length}} & \multicolumn{4}{r}{{\cellcolor[HTML]{F5FBFC}} \color[HTML]{000000} 5.12} \\
    \midrule
    \thead{Colour} & \thead{Guidelines} & \thead{Model name} & \thead{0-shot} & \thead{1-shot} & \thead{3-shot} & \thead{5-shot} \\
    \midrule
    \multirow[c]{18}{*}{\thead{Without\\ Colour}} & \multirow[c]{9}{*}{\thead{Without\\ Guidelines}}  & Falcon 7B & {\cellcolor[HTML]{00441B}} \color[HTML]{F1F1F1} 85.52 & {\cellcolor[HTML]{63C0A0}} \color[HTML]{000000} 45.50 & {\cellcolor[HTML]{9AD8CA}} \color[HTML]{000000} 34.04 & {\cellcolor[HTML]{CAEBE5}} \color[HTML]{000000} 24.43 \\
 &  & Falcon 40B Instruct & {\cellcolor[HTML]{EBF7FA}} \color[HTML]{000000} 10.81 & {\cellcolor[HTML]{E0F3F5}} \color[HTML]{000000} 15.95 & {\cellcolor[HTML]{E2F4F7}} \color[HTML]{000000} 15.20 & {\cellcolor[HTML]{E6F5F9}} \color[HTML]{000000} 13.31 \\
 &  & Gpt 3 & {\cellcolor[HTML]{F6FCFD}} \color[HTML]{000000} 4.22 & {\cellcolor[HTML]{F5FBFD}} \color[HTML]{000000} 4.65 & {\cellcolor[HTML]{F1FAFC}} \color[HTML]{000000} 7.02 & {\cellcolor[HTML]{F2FAFC}} \color[HTML]{000000} 6.65 \\
 &  & Gpt 4 & {\cellcolor[HTML]{F6FCFD}} \color[HTML]{000000} 4.17 & {\cellcolor[HTML]{F5FBFD}} \color[HTML]{000000} 4.71 & {\cellcolor[HTML]{F6FCFD}} \color[HTML]{000000} 4.37 & {\cellcolor[HTML]{F4FBFC}} \color[HTML]{000000} 5.64 \\
 &  & Mpt 7B & {\cellcolor[HTML]{8DD3C0}} \color[HTML]{000000} 36.85 & {\cellcolor[HTML]{BAE5DC}} \color[HTML]{000000} 27.72 & {\cellcolor[HTML]{EBF7FA}} \color[HTML]{000000} 10.77 & {\cellcolor[HTML]{D1EEE9}} \color[HTML]{000000} 22.27 \\
 &  & Mpt 30B Chat & {\cellcolor[HTML]{B0E1D6}} \color[HTML]{000000} 29.83 & {\cellcolor[HTML]{E0F3F5}} \color[HTML]{000000} 16.04 & {\cellcolor[HTML]{E8F6FA}} \color[HTML]{000000} 12.09 & {\cellcolor[HTML]{DEF2F4}} \color[HTML]{000000} 16.80 \\
 &  & Vicuna 13B V1 & {\cellcolor[HTML]{F2FAFC}} \color[HTML]{000000} 6.27 & {\cellcolor[HTML]{ECF8FA}} \color[HTML]{000000} 10.03 & {\cellcolor[HTML]{F0F9FB}} \color[HTML]{000000} 7.73 & {\cellcolor[HTML]{E9F7FA}} \color[HTML]{000000} 11.65 \\
 &  & Vicuna 33B V1 & {\cellcolor[HTML]{F7FCFD}} \color[HTML]{000000} 3.62 & {\cellcolor[HTML]{F7FCFD}} \color[HTML]{000000} 3.91 & {\cellcolor[HTML]{F2FAFC}} \color[HTML]{000000} 6.64 & {\cellcolor[HTML]{EBF7FA}} \color[HTML]{000000} 10.76 \\
 &  & Wizard Vicuna 13B Uncensored HF & {\cellcolor[HTML]{F5FBFC}} \color[HTML]{000000} 4.95 & {\cellcolor[HTML]{ECF8FA}} \color[HTML]{000000} 10.33 & {\cellcolor[HTML]{EFF9FB}} \color[HTML]{000000} 8.45 & {\cellcolor[HTML]{E7F6F9}} \color[HTML]{000000} 13.08 \\

     \cmidrule{2-7}
     & \multirow[c]{9}{*}{\thead{With\\ Guidelines}} & Falcon 7B & {\cellcolor[HTML]{D1EEEA}} \color[HTML]{000000} 21.96 & {\cellcolor[HTML]{CDECE7}} \color[HTML]{000000} 23.69 & {\cellcolor[HTML]{CEEDE8}} \color[HTML]{000000} 23.22 & {\cellcolor[HTML]{BFE7DE}} \color[HTML]{000000} 26.68 \\
 &  & Falcon 40B Instruct & {\cellcolor[HTML]{9DDACB}} \color[HTML]{000000} 33.39 & {\cellcolor[HTML]{C5E9E2}} \color[HTML]{000000} 25.56 & {\cellcolor[HTML]{DBF2F2}} \color[HTML]{000000} 17.81 & {\cellcolor[HTML]{CEEDE8}} \color[HTML]{000000} 23.29 \\
 &  & Gpt 3 & {\cellcolor[HTML]{F1FAFC}} \color[HTML]{000000} 7.09 & {\cellcolor[HTML]{F0F9FB}} \color[HTML]{000000} 7.53 & {\cellcolor[HTML]{EEF8FB}} \color[HTML]{000000} 8.90 & {\cellcolor[HTML]{F1FAFC}} \color[HTML]{000000} 7.44 \\
 &  & Gpt 4 & {\cellcolor[HTML]{F5FBFD}} \color[HTML]{000000} 4.76 & {\cellcolor[HTML]{F5FBFD}} \color[HTML]{000000} 4.71 & {\cellcolor[HTML]{F5FBFD}} \color[HTML]{000000} 4.80 & {\cellcolor[HTML]{F5FBFC}} \color[HTML]{000000} 5.11 \\
 &  & Mpt 7B & {\cellcolor[HTML]{E1F4F6}} \color[HTML]{000000} 15.66 & {\cellcolor[HTML]{D9F1F0}} \color[HTML]{000000} 18.86 & {\cellcolor[HTML]{000000}} \color[HTML]{F1F1F1} nan & {\cellcolor[HTML]{000000}} \color[HTML]{F1F1F1} nan \\
 &  & Mpt 30B Chat & {\cellcolor[HTML]{E1F4F6}} \color[HTML]{000000} 15.53 & {\cellcolor[HTML]{EEF8FB}} \color[HTML]{000000} 8.78 & {\cellcolor[HTML]{E8F6FA}} \color[HTML]{000000} 12.57 & {\cellcolor[HTML]{E8F6FA}} \color[HTML]{000000} 12.47 \\
 &  & Vicuna 13B V1 & {\cellcolor[HTML]{E1F4F6}} \color[HTML]{000000} 15.51 & {\cellcolor[HTML]{D8F0EF}} \color[HTML]{000000} 19.03 & {\cellcolor[HTML]{E7F6F9}} \color[HTML]{000000} 12.96 & {\cellcolor[HTML]{E6F5F9}} \color[HTML]{000000} 13.38 \\
 &  & Vicuna 33B V1 & {\cellcolor[HTML]{E1F4F6}} \color[HTML]{000000} 15.64 & {\cellcolor[HTML]{DFF3F4}} \color[HTML]{000000} 16.66 & {\cellcolor[HTML]{EBF7FA}} \color[HTML]{000000} 10.82 & {\cellcolor[HTML]{E8F6FA}} \color[HTML]{000000} 12.27 \\
 &  & Wizard Vicuna 13B Uncensored HF & {\cellcolor[HTML]{EDF8FB}} \color[HTML]{000000} 9.33 & {\cellcolor[HTML]{E9F7FA}} \color[HTML]{000000} 11.77 & {\cellcolor[HTML]{E3F4F8}} \color[HTML]{000000} 14.68 & {\cellcolor[HTML]{EDF8FB}} \color[HTML]{000000} 9.31 \\

     \midrule
    \multirow[c]{18}{*}{\thead{With\\ Colour}} & \multirow[c]{9}{*}{\thead{Without\\ Guidelines}} & Falcon 7B & {\cellcolor[HTML]{006729}} \color[HTML]{F1F1F1} 76.58 & {\cellcolor[HTML]{6AC4A7}} \color[HTML]{000000} 43.73 & {\cellcolor[HTML]{C2E8E0}} \color[HTML]{000000} 26.12 & {\cellcolor[HTML]{B5E3D9}} \color[HTML]{000000} 28.61 \\
 &  & Falcon 40B Instruct & {\cellcolor[HTML]{DFF3F4}} \color[HTML]{000000} 16.56 & {\cellcolor[HTML]{E8F6FA}} \color[HTML]{000000} 12.21 & {\cellcolor[HTML]{EAF7FA}} \color[HTML]{000000} 11.22 & {\cellcolor[HTML]{DEF2F4}} \color[HTML]{000000} 16.97 \\
 &  & Gpt 3 & {\cellcolor[HTML]{F5FBFD}} \color[HTML]{000000} 4.60 & {\cellcolor[HTML]{F5FBFC}} \color[HTML]{000000} 5.10 & {\cellcolor[HTML]{F2FAFC}} \color[HTML]{000000} 6.51 & {\cellcolor[HTML]{F4FBFC}} \color[HTML]{000000} 5.65 \\
 &  & Gpt 4 & {\cellcolor[HTML]{F5FBFC}} \color[HTML]{000000} 4.96 & {\cellcolor[HTML]{F4FBFC}} \color[HTML]{000000} 5.47 & {\cellcolor[HTML]{F4FBFC}} \color[HTML]{000000} 5.34 & {\cellcolor[HTML]{F4FBFC}} \color[HTML]{000000} 5.52 \\
 &  & Mpt 7B & {\cellcolor[HTML]{6FC6AA}} \color[HTML]{000000} 42.69 & {\cellcolor[HTML]{9CD9CA}} \color[HTML]{000000} 33.74 & {\cellcolor[HTML]{E8F6FA}} \color[HTML]{000000} 12.51 & {\cellcolor[HTML]{D8F0EF}} \color[HTML]{000000} 19.27 \\
 &  & Mpt 30B Chat & {\cellcolor[HTML]{D3EEEB}} \color[HTML]{000000} 21.39 & {\cellcolor[HTML]{DDF2F3}} \color[HTML]{000000} 17.07 & {\cellcolor[HTML]{ECF8FB}} \color[HTML]{000000} 9.88 & {\cellcolor[HTML]{DBF1F1}} \color[HTML]{000000} 18.23 \\
 &  & Vicuna 13B V1 & {\cellcolor[HTML]{F4FBFC}} \color[HTML]{000000} 5.43 & {\cellcolor[HTML]{E3F4F8}} \color[HTML]{000000} 14.56 & {\cellcolor[HTML]{F0F9FB}} \color[HTML]{000000} 7.99 & {\cellcolor[HTML]{E5F5F9}} \color[HTML]{000000} 13.90 \\
 &  & Vicuna 33B V1 & {\cellcolor[HTML]{EDF8FB}} \color[HTML]{000000} 9.06 & {\cellcolor[HTML]{E2F4F7}} \color[HTML]{000000} 15.28 & {\cellcolor[HTML]{E9F7FA}} \color[HTML]{000000} 11.88 & {\cellcolor[HTML]{E6F5F9}} \color[HTML]{000000} 13.24 \\
 &  & Wizard Vicuna 13B Uncensored HF & {\cellcolor[HTML]{F4FBFC}} \color[HTML]{000000} 5.47 & {\cellcolor[HTML]{F4FBFC}} \color[HTML]{000000} 5.60 & {\cellcolor[HTML]{EDF8FB}} \color[HTML]{000000} 9.67 & {\cellcolor[HTML]{EFF9FB}} \color[HTML]{000000} 8.73 \\

     \cmidrule{2-7}
     & \multirow[c]{9}{*}{\thead{With\\ Guidelines}}& Falcon 7B & {\cellcolor[HTML]{D8F0EF}} \color[HTML]{000000} 19.28 & {\cellcolor[HTML]{CDECE6}} \color[HTML]{000000} 23.88 & {\cellcolor[HTML]{D8F0EF}} \color[HTML]{000000} 19.11 & {\cellcolor[HTML]{D1EEE9}} \color[HTML]{000000} 22.21 \\
 &  & Falcon 40B Instruct & {\cellcolor[HTML]{94D6C5}} \color[HTML]{000000} 35.43 & {\cellcolor[HTML]{A4DCCF}} \color[HTML]{000000} 32.40 & {\cellcolor[HTML]{97D7C7}} \color[HTML]{000000} 34.87 & {\cellcolor[HTML]{A8DED2}} \color[HTML]{000000} 31.30 \\
 &  & Gpt 3 & {\cellcolor[HTML]{EBF7FA}} \color[HTML]{000000} 10.52 & {\cellcolor[HTML]{EBF7FA}} \color[HTML]{000000} 10.38 & {\cellcolor[HTML]{F1FAFC}} \color[HTML]{000000} 6.93 & {\cellcolor[HTML]{F1FAFC}} \color[HTML]{000000} 6.93 \\
 &  & Gpt 4 & {\cellcolor[HTML]{F5FBFD}} \color[HTML]{000000} 4.64 & {\cellcolor[HTML]{F6FCFD}} \color[HTML]{000000} 4.49 & {\cellcolor[HTML]{F6FCFD}} \color[HTML]{000000} 4.32 & {\cellcolor[HTML]{F5FBFD}} \color[HTML]{000000} 4.65 \\
 &  & Mpt 7B & {\cellcolor[HTML]{DDF2F3}} \color[HTML]{000000} 17.35 & {\cellcolor[HTML]{D8F0EF}} \color[HTML]{000000} 19.21 & {\cellcolor[HTML]{000000}} \color[HTML]{F1F1F1} nan & {\cellcolor[HTML]{000000}} \color[HTML]{F1F1F1} nan \\
 &  & Mpt 30B Chat & {\cellcolor[HTML]{D5EFED}} \color[HTML]{000000} 20.56 & {\cellcolor[HTML]{D4EFEC}} \color[HTML]{000000} 21.14 & {\cellcolor[HTML]{000000}} \color[HTML]{F1F1F1} nan & {\cellcolor[HTML]{000000}} \color[HTML]{F1F1F1} nan \\
 &  & Vicuna 13B V1 & {\cellcolor[HTML]{DEF2F4}} \color[HTML]{000000} 17.02 & {\cellcolor[HTML]{E2F4F7}} \color[HTML]{000000} 15.41 & {\cellcolor[HTML]{E2F4F7}} \color[HTML]{000000} 15.35 & {\cellcolor[HTML]{F2FAFC}} \color[HTML]{000000} 6.27 \\
 &  & Vicuna 33B V1 & {\cellcolor[HTML]{D9F1F0}} \color[HTML]{000000} 18.74 & {\cellcolor[HTML]{E3F4F7}} \color[HTML]{000000} 15.11 & {\cellcolor[HTML]{E8F6FA}} \color[HTML]{000000} 12.12 & {\cellcolor[HTML]{EBF7FA}} \color[HTML]{000000} 10.64 \\
 &  & Wizard Vicuna 13B Uncensored HF & {\cellcolor[HTML]{EBF7FA}} \color[HTML]{000000} 10.63 & {\cellcolor[HTML]{E7F6F9}} \color[HTML]{000000} 12.68 & {\cellcolor[HTML]{EAF7FA}} \color[HTML]{000000} 11.14 & {\cellcolor[HTML]{E9F7FA}} \color[HTML]{000000} 11.89 \\

    \bottomrule
    \end{tabular}
                
\end{table}

Finally, in Table \ref{tab:personal-narrative-elicitation-token-length-std} are reported the statistics for average sentence length and standard deviation on the answers of the models, compared to the human respective statistics. Similarly to what happened to the divergence, through these statistics, it is not possible to determine if a model is performing well, but it is possible to exclude models whose statistics do not match the expected ones. Again, the statistics of Falcon models are very different from the human reference values. Interestingly, the statistics from Gpt models are also not in line with human data. 

\begin{table}[!htbp]
\centering
\caption{This table reports a few of the best eliciting questions for the models. In the first column the model in question, in the second column the narrative, and in the third column the eliciting question of the model for that specific narrative. The reference crowdsourced eliciting question is reported in the rightmost column. Notice how ChatGPT 3.5 and 4 predict almost word by word the same eliciting questions as crowdworkers. All the examples are obtained with 5 shots as examples, with guidelines and without colour information.}
\label{tab:personal-narrative-elicitation-best-bleu}
    \centering
    \begin{tabularx}{\linewidth}{l|X|X|X}
    \toprule
     \thead{Model name} & \thead{Narrative} & \thead{Model \\eliciting question} & \thead{Human eliciting questions}\\
    \midrule
     \arrayrulecolor{lightgray}
    %\multirow{6}{*}{Gpt-4}
    ChatGPT 4
    & Mi sento leggero. Sono in viaggio per andare a trovare mia figlia. & Oh, che bello! Dove vive tua figlia? & Mi fa piacere, dove abita tua figlia?\\
        \midrule
    ChatGPT 3 & Ciao. Oggi giornata abbastanza tranquilla. Non si lavora ma si hanno faccende domestiche e giardinaggio & Che tipo di faccende domestiche devi fare oggi? & Che bello, che faccende devi fare a casa?\\
    % lmsys/vicuna-13b-v1.3 & {'value': 0.5, 'model_answer': 'A) Cosa hai mangiato in libreria?', 'human_answer': ['hai fatto acquisti in libreria?']} \\
    % mosaicml/mpt-30b-chat & {'value': 0.4043537974357605, 'model_answer': 'Capisco, dove doveva andare tuo figlio?', 'human_answer': ['Non deve essere stato per niente bello. Questi attacchi di panico ti capitano spesso?', 'Oh cavolo, mi spiace. Dove doveva andare tuo figlio?']} \\
        \midrule
    Falcon 40B &  Mi sento leggero. Sono in viaggio per andare a trovare mia figlia & Come ti senti quando sei in viaggio? &  Mi fa piacere, dove abita tua figlia?\\
        \midrule
Vicuna 33B & Mi è dispiaciuto stamattina non aver avuto pazienza con mio padre. Mi sono resa conto che è davvero difficile, ma ripensandoci dopo, magari invece avrei potuto riuscirci &  Sì, capisco. Cosa ti ha fatto perdere pazienza? & Posso capirti. Cosa ti ha fatto perdere la pazienza? \\
    \midrule
    Wiz. Vic. & Mi è dispiaciuto stamattina non aver avuto pazienza con mio padre. Mi sono resa conto che è davvero difficile, ma ripensandoci dopo, magari invece avrei potuto riuscirci & Mi dispiace che tu ti sia sentito male dopo la conversazione. Mi hai spiegato cosa hai appreso da questo episodio? & Posso capirti. Cosa ti ha fatto perdere la pazienza?\\
    % tiiuae/falcon-7b & {'value': 0.19999998807907104, 'model_answer': 'Dovevi intervenire, è una cosa importante, perché lo hai fatto?', 'human_answer': ['Mi dispiace sentirlo, come lo hai difeso?']} \\
    \arrayrulecolor{black}
    \bottomrule
    \end{tabularx}
\end{table}

On a deeper layer of inspection, a small investigation on the best examples for each model reveals that at least those examples are indeed good elicitations. In Table \ref{tab:personal-narrative-elicitation-best-bleu} are reported the examples, with ChatGPT reporting an elicitation that is almost word by word the same elicitation as the human crowdsourced one.

Combining everything together we come to understand that the best examples are obtained with guidelines, without colour and in the 5 shot examples. % In the Appendix, are reported a table with examples for each experimental setup for one narrative and the respective tokens distributions.