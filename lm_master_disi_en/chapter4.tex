\chapter{Evaluation}
\label{cha:evaluation}
% \section{Evaluation}

In this section, an evaluation of the results obtained from the previous chapter \ref{cha:methodology} is conducted. In particular, the results obtained from section \ref{cha:methodology-personal-narrative-elicitation} and section \ref{cha:methodology-data-collection}. 
Previously, a dataset of narratives was crafted, for which elicitations were collected both through crowdsourcing and LLM prompting.
Two detailed analyses were conducted on both the crowdsourcing and LLM-generated data in sections \ref{cha:methodology-crowdsourcing-data-analysis} \ref{cha:methodology-personal-narrative-elicitation-results}. After that, it was determined that the best results were achieved through the experimental setup with guidelines, without colour information with 5-shot examples. 
From this experimental setup, four models elicitations were chosen to apply human evaluation:
\begin{itemize}
    \item Falcon 7B. This model has subpar performance, and it is going to be the lower bound.
    \item ChatGPT 3.5 turbo. Although ChatGPT 4 should perform better, we ultimately opted for ChatGPT 3.5 turbo as it displayed better behaviour with the increasing number of examples. Furthermore, its token distribution is slightly more similar to the human reference data. With all the analysis done so far, this model is expected to perform extremely well.
    \item Wizard Vicuna 13B Uncensored HF. This model performed very well compared to others, especially with few shots and considering its smaller size.
    \item Vicuna 33B V1. Last tested model. This model also performs well according to our analysis, although not as well as ChatGPT.
\end{itemize}
Alongside those four models, a fifth data point was added:
\begin{itemize}
    \item Human crowdsourced elicitations. This data is going to serve as the upper bound.
\end{itemize}
The human evaluation is done according to the human evaluation protocol presented in \cite{mousavi-etal-2022-evaluation}. It was decided to apply all 4 metrics presented in the human evaluation protocol: correctness, appropriateness, contextualisation and listening. Aside from correctness and appropriateness, which are derived from language and commonsense, we are interested in models that p
perform well in contextualisation and listening in particular, due to the task being eliciting continuation of personal narratives. Compared to the original human evaluation protocol there is one key difference, which lies in the fact that the evaluation is being applied to narratives instead of dialogues. 


For the actual evaluation process, the same UI presented in the human evaluation protocol was used. % A few examples of UI are presented in Figure \ref{fig:human}
Three workers were gathered, and they were assigned narratives to be evaluated.  
In order to prevent annotation fatigue, the narratives were split into batches. Each batch was done daily with an estimated annotation time of 40 minutes. Each of these batches contained a number from five to eight narratives. Each narrative contained the selected five elicitation questions: four from the models and the human elicitation. In the batches, the first one acted as a training example. For this reason, all narratives for all annotators were exactly the same in this initial batch. This fact allowed the computation of agreement metrics and determined if issues in the annotators' comprehension of the task were present. After this initial training task, the next batch contained only two narratives in common and successive batches only contained one narrative in common. In total four batches were evaluated.

Overall, this meant that of the 57 narratives, 285 total elicitations were annotated. Of those narratives, nine were evaluated by all three annotators, whereas the remaining 48 narratives were annotated by only one annotator. Considering this redundancy and the fact that there are four metrics for each elicitation, a total of 1500 data points were collected.


\subsection{Results}
Initially, the Fleiss-Kappa \cite{fleiss} metric was calculated for the five narratives contained in the first batch as soon as that annotation task was completed. This allowed to determine that there were no particular issues in the annotators' comprehension of the guidelines for the human evaluation protocol.
\begin{table}[!htbp]
\setlength{\tabcolsep}{3pt}
\centering
\caption{Fleiss-Kappa agreement metric computed over the different batches for all annotators. On the rightmost column the global agreement computed independently of metric, and on the bottom the overall metrics without accounting for batches. In the second column are also reported the number of narratives that are used to compute the metrics for each batch. Notice how correctness is lower than the other metrics due to the unbalance in distributions of correct and incorrect labels.}
\label{tab:human-evaluation-fleiss-kappa}
\begin{tabular}{l|c|rrrr|r}
\toprule
 \thead{Batch} & \thead{N° of narratives} & \thead{Correctness} & \thead{Appropriateness} & \thead{Contextualisation} & \thead{Listening} & \thead{Global} \\
\midrule
Training & 5 &{\cellcolor[HTML]{D3EEB3}} \color[HTML]{000000} 0.16 & {\cellcolor[HTML]{3EB3C4}} \color[HTML]{F1F1F1} 0.47 & {\cellcolor[HTML]{1D8DBE}} \color[HTML]{F1F1F1} 0.61 & {\cellcolor[HTML]{1D90C0}} \color[HTML]{F1F1F1} 0.60 & {\cellcolor[HTML]{216AAD}} \color[HTML]{F1F1F1} 0.70 \\
Batch 1 & 2 &{\cellcolor[HTML]{2195C0}} \color[HTML]{F1F1F1} 0.58 & {\cellcolor[HTML]{3FB4C4}} \color[HTML]{F1F1F1} 0.47 & {\cellcolor[HTML]{24409A}} \color[HTML]{F1F1F1} 0.83 & {\cellcolor[HTML]{2260A9}} \color[HTML]{F1F1F1} 0.73  & {\cellcolor[HTML]{23499E}} \color[HTML]{F1F1F1} 0.80 \\
Batch 2 & 1 &{\cellcolor[HTML]{FFFFD9}} \color[HTML]{000000} -0.07 & {\cellcolor[HTML]{42B6C4}} \color[HTML]{F1F1F1} 0.46 & {\cellcolor[HTML]{2B9FC2}} \color[HTML]{F1F1F1} 0.55 & {\cellcolor[HTML]{4AB9C3}} \color[HTML]{F1F1F1} 0.44 & {\cellcolor[HTML]{1E88BC}} \color[HTML]{F1F1F1} 0.62 \\
Batch 3 & 1 &{\cellcolor[HTML]{4AB9C3}} \color[HTML]{F1F1F1} 0.44 & {\cellcolor[HTML]{2355A4}} \color[HTML]{F1F1F1} 0.76 & {\cellcolor[HTML]{71C8BD}} \color[HTML]{000000} 0.36 & {\cellcolor[HTML]{081D58}} \color[HTML]{F1F1F1} 1.00 & {\cellcolor[HTML]{24459C}} \color[HTML]{F1F1F1} 0.81 \\
% \arrayrulecolor{white}
\midrule  
% \arrayrulecolor{black}
\textbf{Overall} & 9 &{\cellcolor[HTML]{D3EEB3}} \color[HTML]{000000} 0.16 & {\cellcolor[HTML]{3EB3C4}} \color[HTML]{F1F1F1} 0.47 & {\cellcolor[HTML]{1D8DBE}} \color[HTML]{F1F1F1} 0.61 & {\cellcolor[HTML]{1D90C0}} \color[HTML]{F1F1F1} 0.60 & {\cellcolor[HTML]{216AAD}} \color[HTML]{F1F1F1} 0.70 \\
\bottomrule
\end{tabular}
\setlength{\tabcolsep}{6pt}
\end{table}

Afterwards, the Fleiss-Kappa agreement metric was computed each time for both individual batches and overall metrics. In Table \ref{tab:human-evaluation-fleiss-kappa} are reported the results that were obtained. It is possible to observe that most of the annotators have a high agreement on all metrics, except for correctness. Upon further investigation, it was found that the low score is due to the fact that the Fleiss-Kappa metric also accounts for the distributions of the annotations labels. Since most models have correct answers, the very few instances of not correct values heavily penalise the resulting metric.
\begin{table}[!htbp]
\setlength{\tabcolsep}{3pt}
\centering
\caption{Percentage of overlap computed over the different batches for all annotators.  On the right the mean agreement, and on the bottom the overall metrics without accounting for batches. In the second column are also reported the number of narratives that are used to compute the metrics for each batch. Notice that mostly all annotators do agree on their evaluations.}
\label{tab:human-evaluation-overlap}
\begin{tabular}{l|c|rrrr|r}
\toprule
 \thead{Batch} & \thead{N° of narratives} & \thead{Correctness} & \thead{Appropriateness} & \thead{Contextualization} & \thead{Listening} & \thead{Mean} \\
 &  &  &  &  &  \\
\midrule
Training & 5&{\cellcolor[HTML]{707B90}} \color[HTML]{F1F1F1} 0.60 & {\cellcolor[HTML]{7A8A9A}} \color[HTML]{F1F1F1} 0.56 & {\cellcolor[HTML]{595C79}} \color[HTML]{F1F1F1} 0.68 & {\cellcolor[HTML]{656C84}} \color[HTML]{F1F1F1} 0.64 & {\cellcolor[HTML]{6A738A}} \color[HTML]{F1F1F1} 0.62 \\
Batch 1 & 2&{\cellcolor[HTML]{545574}} \color[HTML]{F1F1F1} 0.70 & {\cellcolor[HTML]{545574}} \color[HTML]{F1F1F1} 0.70 & {\cellcolor[HTML]{1C1C27}} \color[HTML]{F1F1F1} 0.90 & {\cellcolor[HTML]{37374D}} \color[HTML]{F1F1F1} 0.80 & {\cellcolor[HTML]{3F3F58}} \color[HTML]{F1F1F1} 0.78 \\
Batch 2 & 1&{\cellcolor[HTML]{FFFFFF}} \color[HTML]{000000} 0.20 & {\cellcolor[HTML]{707B90}} \color[HTML]{F1F1F1} 0.60 & {\cellcolor[HTML]{707B90}} \color[HTML]{F1F1F1} 0.60 & {\cellcolor[HTML]{707B90}} \color[HTML]{F1F1F1} 0.60 & {\cellcolor[HTML]{8CA2AC}} \color[HTML]{F1F1F1} 0.50 \\
Batch 3 & 1&{\cellcolor[HTML]{707B90}} \color[HTML]{F1F1F1} 0.60 & {\cellcolor[HTML]{37374D}} \color[HTML]{F1F1F1} 0.80 & {\cellcolor[HTML]{707B90}} \color[HTML]{F1F1F1} 0.60 & {\cellcolor[HTML]{000000}} \color[HTML]{F1F1F1} 1.00 & {\cellcolor[HTML]{454560}} \color[HTML]{F1F1F1} 0.75 \\
% \arrayrulecolor{white}
\midrule  
% \arrayrulecolor{black}
\textbf{Overall} & 9 &{\cellcolor[HTML]{707B90}} \color[HTML]{F1F1F1} 0.60 & {\cellcolor[HTML]{7A8A9A}} \color[HTML]{F1F1F1} 0.56 & {\cellcolor[HTML]{595C79}} \color[HTML]{F1F1F1} 0.68 & {\cellcolor[HTML]{656C84}} \color[HTML]{F1F1F1} 0.64 & {\cellcolor[HTML]{6A738A}} \color[HTML]{F1F1F1} 0.62 \\
\bottomrule
\end{tabular}
\setlength{\tabcolsep}{6pt}
\end{table}
In order to solve this issue, a decision to compute the overlap was made. This metric accounts for how many answers are shared across annotators. In Table \ref{tab:human-evaluation-overlap} are shown the percent of overlap. This metric did show that in fact, the annotators agree on correctness with a similar ratio to the other metrics measured. 



% We found that one annotator disagrees with the other two because some examples are ambiguous. In \ref{} are reported the problematic examples. It is possible to see that those examples have extra characters such as \emph{``} or \emph{a.} which for one annotator are marked as incorrect and the others as correct.

These two metrics were computed for each of the three pairs of annotators to highlight with more details if there were issues in their agreement. 
\begin{table}[!htbp]
\setlength{\tabcolsep}{3pt}
\centering
\caption{Fleiss-Kappa agreement metric computed over the different batches for all pairs of annotators. On the left are reported the batches while on the right are reported the Fleiss-Kappa agreements for each pair of annotators. A value of 1 for batch 3 under column user1-user2 for subcolumn app. means that both user 1 and user 2 totally agree on their evaluations for the appropriateness metric for all models in batch 3. The last row represents the values computed aggregating all batches together.}
\label{tab:human-evaluation-fleiss-kappa-pairs}
\begin{tabular}{l|rrrr|r|rrrr|r|rrrr|r}
\toprule
 \thead{Batch} & \multicolumn{5}{c|}{\thead{User 1-User 2}} & \multicolumn{5}{c|}{\thead{User 1-User 3}} & \multicolumn{5}{c}{\thead{User 2-User 3}} \\
 & \thead{Cor.} & \thead{App.} & \thead{Con.} & \thead{Lis.} & \thead{Glob.} & \thead{Cor.} & \thead{App.} & \thead{Con.} & \thead{Lis.} & \thead{Glob.} & \thead{Cor.} & \thead{App.} & \thead{Con.} & \thead{Lis.} & \thead{Glob.}\\
\midrule
Training & {\cellcolor[HTML]{63C3BF}} \color[HTML]{000000} 0.11 & {\cellcolor[HTML]{FFFFD9}} \color[HTML]{000000} 0.34 & {\cellcolor[HTML]{1F7BB6}} \color[HTML]{F1F1F1} 0.64 & {\cellcolor[HTML]{99D7B8}} \color[HTML]{000000} 0.68 & {\cellcolor[HTML]{C4E8B4}} \color[HTML]{000000} 0.67 & {\cellcolor[HTML]{34A9C3}} \color[HTML]{F1F1F1} 0.07 & {\cellcolor[HTML]{2355A4}} \color[HTML]{F1F1F1} 0.55 & {\cellcolor[HTML]{1D90C0}} \color[HTML]{F1F1F1} 0.60 & {\cellcolor[HTML]{FFFFD9}} \color[HTML]{000000} 0.59 & {\cellcolor[HTML]{1E86BB}} \color[HTML]{F1F1F1} 0.69 & {\cellcolor[HTML]{FFFFD9}} \color[HTML]{000000} 0.34 & {\cellcolor[HTML]{6FC7BD}} \color[HTML]{000000} 0.52 & {\cellcolor[HTML]{40B5C4}} \color[HTML]{F1F1F1} 0.57 & {\cellcolor[HTML]{61C2BF}} \color[HTML]{000000} 0.53 & {\cellcolor[HTML]{FFFFD9}} \color[HTML]{000000} 0.74 \\
Batch 1 & {\cellcolor[HTML]{081D58}} \color[HTML]{F1F1F1} 0.81 & {\cellcolor[HTML]{FFFFD9}} \color[HTML]{000000} 0.34 & {\cellcolor[HTML]{081D58}} \color[HTML]{F1F1F1} 0.75 & {\cellcolor[HTML]{EAF7B1}} \color[HTML]{000000} 0.59 & {\cellcolor[HTML]{2075B3}} \color[HTML]{F1F1F1} 0.80 & {\cellcolor[HTML]{081D58}} \color[HTML]{F1F1F1} 0.48 & {\cellcolor[HTML]{192B7C}} \color[HTML]{F1F1F1} 0.60 & {\cellcolor[HTML]{081D58}} \color[HTML]{F1F1F1} 0.75 & {\cellcolor[HTML]{37ACC3}} \color[HTML]{F1F1F1} 0.81 & {\cellcolor[HTML]{081D58}} \color[HTML]{F1F1F1} 0.80 & {\cellcolor[HTML]{E3F4B2}} \color[HTML]{000000} 0.45 & {\cellcolor[HTML]{FFFFD9}} \color[HTML]{000000} 0.44 & {\cellcolor[HTML]{081D58}} \color[HTML]{F1F1F1} 1.00 & {\cellcolor[HTML]{2163AA}} \color[HTML]{F1F1F1} 0.78 & {\cellcolor[HTML]{081D58}} \color[HTML]{F1F1F1} 0.80 \\
Batch 2 & {\cellcolor[HTML]{FFFFD9}} \color[HTML]{000000} -0.43 & {\cellcolor[HTML]{92D5B9}} \color[HTML]{000000} 0.57 & {\cellcolor[HTML]{216DAF}} \color[HTML]{F1F1F1} 0.66 & {\cellcolor[HTML]{FFFFD9}} \color[HTML]{000000} 0.52 & {\cellcolor[HTML]{FFFFD9}} \color[HTML]{000000} 0.60 & {\cellcolor[HTML]{FFFFD9}} \color[HTML]{000000} -0.43 & {\cellcolor[HTML]{FFFFD9}} \color[HTML]{000000} 0.29 & {\cellcolor[HTML]{FFFFD9}} \color[HTML]{000000} 0.35 & {\cellcolor[HTML]{FAFDCE}} \color[HTML]{000000} 0.60 & {\cellcolor[HTML]{FFFFD9}} \color[HTML]{000000} 0.49 & {\cellcolor[HTML]{081D58}} \color[HTML]{F1F1F1} 1.00 & {\cellcolor[HTML]{1E86BB}} \color[HTML]{F1F1F1} 0.57 & {\cellcolor[HTML]{2CA1C2}} \color[HTML]{F1F1F1} 0.63 & {\cellcolor[HTML]{FFFFD9}} \color[HTML]{000000} 0.17 & {\cellcolor[HTML]{3CB1C3}} \color[HTML]{F1F1F1} 0.77 \\
Batch 3 & {\cellcolor[HTML]{24439B}} \color[HTML]{F1F1F1} 0.60 & {\cellcolor[HTML]{081D58}} \color[HTML]{F1F1F1} 1.00 & {\cellcolor[HTML]{FFFFD9}} \color[HTML]{000000} 0.41 & {\cellcolor[HTML]{081D58}} \color[HTML]{F1F1F1} 1.00 & {\cellcolor[HTML]{081D58}} \color[HTML]{F1F1F1} 0.89 & {\cellcolor[HTML]{1E85BA}} \color[HTML]{F1F1F1} 0.17 & {\cellcolor[HTML]{081D58}} \color[HTML]{F1F1F1} 0.63 & {\cellcolor[HTML]{67C4BE}} \color[HTML]{000000} 0.52 & {\cellcolor[HTML]{081D58}} \color[HTML]{F1F1F1} 1.00 & {\cellcolor[HTML]{1D2E83}} \color[HTML]{F1F1F1} 0.77 & {\cellcolor[HTML]{BBE4B5}} \color[HTML]{000000} 0.52 & {\cellcolor[HTML]{081D58}} \color[HTML]{F1F1F1} 0.63 & {\cellcolor[HTML]{FFFFD9}} \color[HTML]{000000} 0.13 & {\cellcolor[HTML]{081D58}} \color[HTML]{F1F1F1} 1.00 & {\cellcolor[HTML]{2DA2C2}} \color[HTML]{F1F1F1} 0.77 \\
\midrule
\thead{Overall} & {\cellcolor[HTML]{63C3BF}} \color[HTML]{000000} 0.11 & {\cellcolor[HTML]{FFFFD9}} \color[HTML]{000000} 0.34 & {\cellcolor[HTML]{1F7BB6}} \color[HTML]{F1F1F1} 0.64 & {\cellcolor[HTML]{99D7B8}} \color[HTML]{000000} 0.68 & {\cellcolor[HTML]{C4E8B4}} \color[HTML]{000000} 0.67 & {\cellcolor[HTML]{34A9C3}} \color[HTML]{F1F1F1} 0.07 & {\cellcolor[HTML]{2355A4}} \color[HTML]{F1F1F1} 0.55 & {\cellcolor[HTML]{1D90C0}} \color[HTML]{F1F1F1} 0.60 & {\cellcolor[HTML]{FFFFD9}} \color[HTML]{000000} 0.59 & {\cellcolor[HTML]{1E86BB}} \color[HTML]{F1F1F1} 0.69 & {\cellcolor[HTML]{FFFFD9}} \color[HTML]{000000} 0.34 & {\cellcolor[HTML]{6FC7BD}} \color[HTML]{000000} 0.52 & {\cellcolor[HTML]{40B5C4}} \color[HTML]{F1F1F1} 0.57 & {\cellcolor[HTML]{61C2BF}} \color[HTML]{000000} 0.53 & {\cellcolor[HTML]{FFFFD9}} \color[HTML]{000000} 0.74 \\
\bottomrule
\end{tabular}
\setlength{\tabcolsep}{6pt}
\end{table}
\begin{table}[!htbp]
\setlength{\tabcolsep}{3pt}
\centering
\caption{Percentage of overlap computed over the different batches for all pairs of annotators. On the left are the batches and on the right are reported the metrics computed for each annotator pair. A value of 0.90 for Batch 1, under the column Cor. for User1-User2 means that User 1 and User 2 agree 90\% of the time on the Correctness metric for all models in Batch 2. The last row on the bottom represents the values computed aggregating all batches together.}
\label{tab:human-evaluation-overlap-pairs}
\begin{tabular}{l|@{\hspace{2\tabcolsep}}rrrr |@{\hspace{2\tabcolsep}}r@{\hspace{2\tabcolsep}}|@{\hspace{2\tabcolsep}}rrrr|@{\hspace{2\tabcolsep}}r@{\hspace{2\tabcolsep}}|@{\hspace{2\tabcolsep}}rrrr |@{\hspace{2\tabcolsep}}r}
\toprule
\multicolumn{16}{c}{\thead{\% of overlap over pairs of annotators}} \\
\midrule
 \thead{Batch} & \multicolumn{5}{c|@{\hspace{2\tabcolsep}}}{\thead{User 1-User 2}} & \multicolumn{5}{c|@{\hspace{2\tabcolsep}}}{\thead{User 1-User 3}} & \multicolumn{5}{c}{\thead{User 2-User 3}} \\
 & \thead{Cor.} & \thead{App.} & \thead{Con.} & \thead{Lis.} & \thead{Mean} & \thead{Cor.} & \thead{App.} & \thead{Con.} & \thead{Lis.} & \thead{Mean} & \thead{Cor.} & \thead{App.} & \thead{Con.} & \thead{Lis.} & \thead{Mean}\\
\midrule
Training & {\cellcolor[HTML]{535473}} \color[HTML]{F1F1F1} 0.64 & {\cellcolor[HTML]{FFFFFF}} \color[HTML]{000000} 0.60 & {\cellcolor[HTML]{FFFFFF}} \color[HTML]{000000} 0.80 & {\cellcolor[HTML]{FFFFFF}} \color[HTML]{000000} 0.80 & {\cellcolor[HTML]{ACCACA}} \color[HTML]{000000} 0.71 & {\cellcolor[HTML]{1A1A25}} \color[HTML]{F1F1F1} 0.64 & {\cellcolor[HTML]{595C79}} \color[HTML]{F1F1F1} 0.72 & {\cellcolor[HTML]{4A4A67}} \color[HTML]{F1F1F1} 0.80 & {\cellcolor[HTML]{FFFFFF}} \color[HTML]{000000} 0.76 & {\cellcolor[HTML]{4D4D6B}} \color[HTML]{F1F1F1} 0.73 & {\cellcolor[HTML]{595C79}} \color[HTML]{F1F1F1} 0.88 & {\cellcolor[HTML]{B9D2D2}} \color[HTML]{000000} 0.72 & {\cellcolor[HTML]{869AA6}} \color[HTML]{F1F1F1} 0.76 & {\cellcolor[HTML]{9DB9BC}} \color[HTML]{000000} 0.72 & {\cellcolor[HTML]{FFFFFF}} \color[HTML]{000000} 0.77 \\
Batch 1 & {\cellcolor[HTML]{000000}} \color[HTML]{F1F1F1} 0.90 & {\cellcolor[HTML]{A9C8C8}} \color[HTML]{000000} 0.70 & {\cellcolor[HTML]{000000}} \color[HTML]{F1F1F1} 0.90 & {\cellcolor[HTML]{FFFFFF}} \color[HTML]{000000} 0.80 & {\cellcolor[HTML]{42425D}} \color[HTML]{F1F1F1} 0.82 & {\cellcolor[HTML]{000000}} \color[HTML]{F1F1F1} 0.70 & {\cellcolor[HTML]{000000}} \color[HTML]{F1F1F1} 0.80 & {\cellcolor[HTML]{000000}} \color[HTML]{F1F1F1} 0.90 & {\cellcolor[HTML]{5D617D}} \color[HTML]{F1F1F1} 0.90 & {\cellcolor[HTML]{000000}} \color[HTML]{F1F1F1} 0.82 & {\cellcolor[HTML]{FFFFFF}} \color[HTML]{000000} 0.70 & {\cellcolor[HTML]{FFFFFF}} \color[HTML]{000000} 0.70 & {\cellcolor[HTML]{000000}} \color[HTML]{F1F1F1} 1.00 & {\cellcolor[HTML]{37374D}} \color[HTML]{F1F1F1} 0.90 & {\cellcolor[HTML]{000000}} \color[HTML]{F1F1F1} 0.82 \\
Batch 2 & {\cellcolor[HTML]{FFFFFF}} \color[HTML]{000000} 0.20 & {\cellcolor[HTML]{6F7A8F}} \color[HTML]{F1F1F1} 0.80 & {\cellcolor[HTML]{FFFFFF}} \color[HTML]{000000} 0.80 & {\cellcolor[HTML]{FFFFFF}} \color[HTML]{000000} 0.80 & {\cellcolor[HTML]{FFFFFF}} \color[HTML]{000000} 0.65 & {\cellcolor[HTML]{FFFFFF}} \color[HTML]{000000} 0.20 & {\cellcolor[HTML]{FFFFFF}} \color[HTML]{000000} 0.60 & {\cellcolor[HTML]{FFFFFF}} \color[HTML]{000000} 0.60 & {\cellcolor[HTML]{C6DADA}} \color[HTML]{000000} 0.80 & {\cellcolor[HTML]{FFFFFF}} \color[HTML]{000000} 0.55 & {\cellcolor[HTML]{000000}} \color[HTML]{F1F1F1} 1.00 & {\cellcolor[HTML]{000000}} \color[HTML]{F1F1F1} 0.80 & {\cellcolor[HTML]{6F7A8F}} \color[HTML]{F1F1F1} 0.80 & {\cellcolor[HTML]{FFFFFF}} \color[HTML]{000000} 0.60 & {\cellcolor[HTML]{666D85}} \color[HTML]{F1F1F1} 0.80 \\
Batch 3 & {\cellcolor[HTML]{201F2C}} \color[HTML]{F1F1F1} 0.80 & {\cellcolor[HTML]{000000}} \color[HTML]{F1F1F1} 1.00 & {\cellcolor[HTML]{FFFFFF}} \color[HTML]{000000} 0.80 & {\cellcolor[HTML]{000000}} \color[HTML]{F1F1F1} 1.00 & {\cellcolor[HTML]{000000}} \color[HTML]{F1F1F1} 0.90 & {\cellcolor[HTML]{2D2D3E}} \color[HTML]{F1F1F1} 0.60 & {\cellcolor[HTML]{000000}} \color[HTML]{F1F1F1} 0.80 & {\cellcolor[HTML]{4A4A67}} \color[HTML]{F1F1F1} 0.80 & {\cellcolor[HTML]{000000}} \color[HTML]{F1F1F1} 1.00 & {\cellcolor[HTML]{14141C}} \color[HTML]{F1F1F1} 0.80 & {\cellcolor[HTML]{95AEB5}} \color[HTML]{F1F1F1} 0.80 & {\cellcolor[HTML]{000000}} \color[HTML]{F1F1F1} 0.80 & {\cellcolor[HTML]{FFFFFF}} \color[HTML]{000000} 0.60 & {\cellcolor[HTML]{000000}} \color[HTML]{F1F1F1} 1.00 & {\cellcolor[HTML]{666D85}} \color[HTML]{F1F1F1} 0.80 \\
\midrule
\thead{Overall} & {\cellcolor[HTML]{535473}} \color[HTML]{F1F1F1} 0.64 & {\cellcolor[HTML]{FFFFFF}} \color[HTML]{000000} 0.60 & {\cellcolor[HTML]{FFFFFF}} \color[HTML]{000000} 0.80 & {\cellcolor[HTML]{FFFFFF}} \color[HTML]{000000} 0.80 & {\cellcolor[HTML]{ACCACA}} \color[HTML]{000000} 0.71 & {\cellcolor[HTML]{1A1A25}} \color[HTML]{F1F1F1} 0.64 & {\cellcolor[HTML]{595C79}} \color[HTML]{F1F1F1} 0.72 & {\cellcolor[HTML]{4A4A67}} \color[HTML]{F1F1F1} 0.80 & {\cellcolor[HTML]{FFFFFF}} \color[HTML]{000000} 0.76 & {\cellcolor[HTML]{4D4D6B}} \color[HTML]{F1F1F1} 0.73 & {\cellcolor[HTML]{595C79}} \color[HTML]{F1F1F1} 0.88 & {\cellcolor[HTML]{B9D2D2}} \color[HTML]{000000} 0.72 & {\cellcolor[HTML]{869AA6}} \color[HTML]{F1F1F1} 0.76 & {\cellcolor[HTML]{9DB9BC}} \color[HTML]{000000} 0.72 & {\cellcolor[HTML]{FFFFFF}} \color[HTML]{000000} 0.77 \\
\bottomrule
\end{tabular}
\setlength{\tabcolsep}{6pt}
\end{table}
Reported in Table \ref{tab:human-evaluation-fleiss-kappa-pairs} and Table \ref{tab:human-evaluation-overlap-pairs} are the resulting data. From both metrics, it is possible to understand that all annotators have an overall high agreement over all metrics.
Overall it was found that the annotators do mostly agree on their annotations.

\begin{table}[!htbp]
\setlength{\tabcolsep}{3pt}
\centering
\caption{Table reporting the percentage of positive, negative and uncertain values for the 4 models evaluated by 3 human annotators. Each subcolumn represent the evaluation for a different task, Correctness, Appropriateness, Contextualization, and Listening. A value of 0.90 in Positive Correctness means that for all narrative elicitations of that model, 90\% of them were evaluated by 3 human annotators as correct. On the top row the same metrics for the human crowdsourced elicitations. A value of 0.69 in Negative Listening means that 69\% of narratites of that model were evaluated as not listening. Notice how ChatGPT has a very similar behaviour to human crowdsourced elicitations.}
\label{tab:human-evaluation-scores}
\begin{tabular}{l|rrrr|r|rrrr|r|rrrr|r}
\toprule
 \thead{Model} & \multicolumn{5}{c|}{\thead{Positive}} & \multicolumn{5}{c|}{\thead{Negative}} & \multicolumn{5}{c}{\thead{Uncertain}} \\
 & \thead{Cor.} & \thead{App.} & \thead{Con.} & \thead{Lis.} & \thead{Mean} & \thead{Cor.} & \thead{App.} & \thead{Con.} & \thead{Lis.} & \thead{Mean} & \thead{Cor.} & \thead{App.} & \thead{Con.} & \thead{Lis.} & \thead{Mean}\\
\midrule
% \vspace{1em}
Human & {\cellcolor[HTML]{006737}} \color[HTML]{F1F1F1} 0.90 & {\cellcolor[HTML]{006737}} \color[HTML]{F1F1F1} 0.90 & {\cellcolor[HTML]{0A703A}} \color[HTML]{F1F1F1} 0.86 & {\cellcolor[HTML]{1C7E40}} \color[HTML]{F1F1F1} 0.81 & {\cellcolor[HTML]{096F3A}} \color[HTML]{F1F1F1} 0.87 & {\cellcolor[HTML]{FFF1A9}} \color[HTML]{000000} 0.07 & {\cellcolor[HTML]{FFF1A9}} \color[HTML]{000000} 0.07 & {\cellcolor[HTML]{FFEDA0}} \color[HTML]{000000} 0.09 & {\cellcolor[HTML]{FFF1A9}} \color[HTML]{000000} 0.07 & {\cellcolor[HTML]{FFF0A7}} \color[HTML]{000000} 0.07 & {\cellcolor[HTML]{FFF1AE}} \color[HTML]{000000} 0.03 & {\cellcolor[HTML]{FFF1AE}} \color[HTML]{000000} 0.03 & {\cellcolor[HTML]{FEE390}} \color[HTML]{000000} 0.05 & {\cellcolor[HTML]{F27D1B}} \color[HTML]{F1F1F1} 0.12 & {\cellcolor[HTML]{FED97C}} \color[HTML]{000000} 0.06 \\
% \midrule
% &&&&&&&&&&&&\\
\arrayrulecolor{white}
\midrule  
\arrayrulecolor{black}
ChatGPT 3.5 & {\cellcolor[HTML]{30954F}} \color[HTML]{F1F1F1} 0.74 & {\cellcolor[HTML]{0A703A}} \color[HTML]{F1F1F1} 0.86 & {\cellcolor[HTML]{004529}} \color[HTML]{F1F1F1} 1.00 & {\cellcolor[HTML]{056C39}} \color[HTML]{F1F1F1} 0.88 & {\cellcolor[HTML]{086E3A}} \color[HTML]{F1F1F1} 0.87 & {\cellcolor[HTML]{FEBA55}} \color[HTML]{000000} 0.24 & {\cellcolor[HTML]{FFF4B2}} \color[HTML]{000000} 0.05 & {\cellcolor[HTML]{FFFFCC}} \color[HTML]{000000} 0.00 & {\cellcolor[HTML]{FFF8BB}} \color[HTML]{000000} 0.03 & {\cellcolor[HTML]{FFEEA3}} \color[HTML]{000000} 0.08 & {\cellcolor[HTML]{FFFACA}} \color[HTML]{000000} 0.02 & {\cellcolor[HTML]{FEB643}} \color[HTML]{000000} 0.09 & {\cellcolor[HTML]{FFFFE5}} \color[HTML]{000000} 0.00 & {\cellcolor[HTML]{FEB643}} \color[HTML]{000000} 0.09 & {\cellcolor[HTML]{FEE799}} \color[HTML]{000000} 0.05 \\
Wiz. Vic. 13B & {\cellcolor[HTML]{A7DB8C}} \color[HTML]{000000} 0.48 & {\cellcolor[HTML]{8ED082}} \color[HTML]{000000} 0.53 & {\cellcolor[HTML]{4FB264}} \color[HTML]{F1F1F1} 0.66 & {\cellcolor[HTML]{BEE596}} \color[HTML]{000000} 0.43 & {\cellcolor[HTML]{92D183}} \color[HTML]{000000} 0.53 & {\cellcolor[HTML]{F74327}} \color[HTML]{F1F1F1} 0.45 & {\cellcolor[HTML]{FD8038}} \color[HTML]{F1F1F1} 0.36 & {\cellcolor[HTML]{FEAB49}} \color[HTML]{000000} 0.28 & {\cellcolor[HTML]{FC4F2A}} \color[HTML]{F1F1F1} 0.43 & {\cellcolor[HTML]{FD7435}} \color[HTML]{F1F1F1} 0.38 & {\cellcolor[HTML]{FECE65}} \color[HTML]{000000} 0.07 & {\cellcolor[HTML]{FE9829}} \color[HTML]{000000} 0.10 & {\cellcolor[HTML]{FECE65}} \color[HTML]{000000} 0.07 & {\cellcolor[HTML]{E1640E}} \color[HTML]{F1F1F1} 0.14 & {\cellcolor[HTML]{FEA736}} \color[HTML]{000000} 0.09 \\
Vicuna 33B V1 & {\cellcolor[HTML]{DFF3A8}} \color[HTML]{000000} 0.34 & {\cellcolor[HTML]{BEE596}} \color[HTML]{000000} 0.43 & {\cellcolor[HTML]{58B669}} \color[HTML]{F1F1F1} 0.64 & {\cellcolor[HTML]{8ED082}} \color[HTML]{000000} 0.53 & {\cellcolor[HTML]{A6DA8B}} \color[HTML]{000000} 0.49 & {\cellcolor[HTML]{C40524}} \color[HTML]{F1F1F1} 0.59 & {\cellcolor[HTML]{FD8038}} \color[HTML]{F1F1F1} 0.36 & {\cellcolor[HTML]{FEA446}} \color[HTML]{000000} 0.29 & {\cellcolor[HTML]{FD7435}} \color[HTML]{F1F1F1} 0.38 & {\cellcolor[HTML]{FC612F}} \color[HTML]{F1F1F1} 0.41 & {\cellcolor[HTML]{FECE65}} \color[HTML]{000000} 0.07 & {\cellcolor[HTML]{662506}} \color[HTML]{F1F1F1} 0.21 & {\cellcolor[HTML]{FECE65}} \color[HTML]{000000} 0.07 & {\cellcolor[HTML]{FEB643}} \color[HTML]{000000} 0.09 & {\cellcolor[HTML]{FB9225}} \color[HTML]{000000} 0.11 \\
Falcon 7B & {\cellcolor[HTML]{9FD788}} \color[HTML]{000000} 0.50 & {\cellcolor[HTML]{FAFDC9}} \color[HTML]{000000} 0.22 & {\cellcolor[HTML]{F3FAB6}} \color[HTML]{000000} 0.28 & {\cellcolor[HTML]{FFFFE5}} \color[HTML]{000000} 0.16 & {\cellcolor[HTML]{EFF9B3}} \color[HTML]{000000} 0.29 & {\cellcolor[HTML]{FC4F2A}} \color[HTML]{F1F1F1} 0.43 & {\cellcolor[HTML]{B00026}} \color[HTML]{F1F1F1} 0.62 & {\cellcolor[HTML]{E2191C}} \color[HTML]{F1F1F1} 0.52 & {\cellcolor[HTML]{800026}} \color[HTML]{F1F1F1} 0.69 & {\cellcolor[HTML]{CE0C22}} \color[HTML]{F1F1F1} 0.56 & {\cellcolor[HTML]{FECE65}} \color[HTML]{000000} 0.07 & {\cellcolor[HTML]{CB4B02}} \color[HTML]{F1F1F1} 0.16 & {\cellcolor[HTML]{662506}} \color[HTML]{F1F1F1} 0.21 & {\cellcolor[HTML]{CB4B02}} \color[HTML]{F1F1F1} 0.16 & {\cellcolor[HTML]{D65808}} \color[HTML]{F1F1F1} 0.15 \\
\bottomrule
\end{tabular}
\setlength{\tabcolsep}{6pt}
\end{table}
With the notion that all annotators do mostly agree, with those numbers, a scoring system was implemented. The scores were computed for each model, counting the times the models' elicitaons were evaluated as correct, appropriate, contextualised, and listening. The same scoring system was applied for incorrect, inappropriate, contextualised, not listening, and unsure labels. In the case of elicitations annotated by all 3 annotators, the majority vote was taken as valid. These results are shown in Table \ref{tab:human-evaluation-scores}.
Interestingly, only ChatGPT 3.5 turbo follows very similar characteristics to humans. All other models are significantly subpar compared to human annotators. 

% add examples of subjunctive
\begin{table}[!htbp]
\centering
\caption{Examples of eliciting questions of a narrative. On the first row is reported the narrative that is given as context. Following on each row are the models that were evaluated. In the second column are reported the eliciting questions that were evaluated and in the 4 rightmost columns are reported the correct or incorrect values for each of the 4 metrics that were evaluated ( correctness, appropriateness, contextualisation, and listening. Consider that even human answers are not 100\% accurate according to other humans, as in this example, the annotator was unsure whether the crowdsourced answer was listening. In this example the same annotator considers ChatGPT to be correct on all metrics. Wizard Vicuna has issues in the listening and context, but its eliciting question is grammatically correct. Falcon 7B is also reported as a lower bound.}
\label{tab:human-evaluation-examples}
    \centering
    \begin{tabularx}{\linewidth}{ l|X|c|c|c|c}
    % \begin{tabular}{p{1.5cm}|p{3cm}|p{5cm}|p{2.5cm}|p{2cm}}
        \toprule
        % \thead{Mode}
        % \midrule
       \thead{Narrative} & \multicolumn{5}{p{14cm}}{Ritornata dal lavoro mia figlia mi dice di aver chiamato il medico perché ha dei dolori alla testa, non ha il senso dell'olfatto e del gusto per cui ci siamo un po' allarmati.}\\
       \midrule
       \thead{Model name} & \thead{Eliciting question} &  \thead{Cor.} & \thead{App.} & \thead{Con.} & \thead{Lis.} \\
       \midrule
        \thead{Human} & Mi dispiace di sentirlo, da quanto tempo tua figlia ha questi sintomi? & \cmark & \cmark & \cmark & \textbf{Unsure}\\
        \arrayrulecolor{white}
        \midrule
        \arrayrulecolor{black}
        \thead{ChatGPT 3} & Mi dispiace sentire che tua figlia sta avendo questi sintomi preoccupanti. Cosa ha detto il medico riguardo ai suoi sintomi? & \cmark & \cmark & \cmark & \cmark \\
        
        \thead{Wiz. Vic.} & Mi dispiace molto, come ti senti ora? Hai avuto altri sintomi? Hai già fatto una visita al medico? & \cmark & \cmark & \xmark & \xmark \\
        \thead{Vic. 33B} & Speri che sia solo il stress del lavoro? & \xmark & \cmark & \cmark & \cmark \\
        \thead{Fal. 7B} & Sì, è normale dopo una caduta di questo genere. 'NARRATIVA: 'Mia moglie è al telefono con la nostra dottoressa. Ci sono un po' preoccupati perché la testa di mia figlia sanguina. Il medico ha detto però di non preoccuparci. 'DOMANDA:             'Mi dispiace sentirlo, però non si preoccupi tanto. 'NARRATIVA: 'Mia figlia ha la febbre. Le ho fatto il bagno e l'ho messa a letto. Non si è sve & \xmark & \xmark & \xmark & \xmark \\
        
        \bottomrule

    \end{tabularx}
\end{table}
\begin{table}[!htbp]
\centering
\caption{Examples of eliciting questions  a narrative. On the first row is reported the narrative that is given as context. Following on each row are the models that were evaluated. In the second column are reported the eliciting questions that were evaluated and in the 4 rightmost columns are reported the correct or incorrect values for each of the 4 metrics that were evaluated ( correctness, appropriateness, contextualisation, and listening. In this example, the annotators consider human crowdsouced, ChatGPT, and Wizard Vicuna eliciting questions to be correct on all metrics. The other two models have significant issues in their responses.}
\label{tab:human-evaluation-examples-1}
    \centering
    \begin{tabularx}{\linewidth}{ l|X|c|c|c|c}
    % \begin{tabular}{p{1.5cm}|p{3cm}|p{5cm}|p{2.5cm}|p{2cm}}
        \toprule
        % \thead{Mode}
        % \midrule
       \thead{Narrative} & \multicolumn{5}{p{14cm}}{ Questa mattina ero a casa con mia figlia e sono molto rilassata..}\\
       \midrule
       \thead{Model name} & \thead{Elicitation} &  \thead{Cor.} & \thead{App.} & \thead{Con.} & \thead{Lis.} \\
       \midrule
        \thead{Human} & Mi fa piacere, cosa avete fatto tu e tua figlia stamattina? & \cmark & \cmark & \cmark & \cmark\\
        \arrayrulecolor{white}
        \midrule
        \arrayrulecolor{black}
        \thead{ChatGPT 3} & Che cosa avete fatto insieme durante la mattina? & \cmark & \cmark & \cmark & \cmark \\
        
        \thead{Wiz. Vic.} & Come hai passato il tempo con la tua bambina? & \cmark & \cmark & \cmark & \cmark \\
        \thead{Vic. 33B} & Sembra di avere uno spazio dedicato per rispondere alle domande. Hai già deciso quale sarà il tuo obiettivo o hai bisogno di ulteriori informazioni per formulare la domanda? & \textbf{Unsure} & \xmark & \xmark & \xmark \\
        \thead{Fal. 7B} & Ah, allora la situazione è tranquilla.' ( 3 ) DOMANDA: 'Sì, stamattina ho avuto un'ora libera. 'DOMANDA: 'Ah, allora \u00e8 per questo motivo che la situazione è tranquilla, 'DOMANDA: 'sì sono contenta 'DOMANDA: 'Ah, allora \u00e8 per questo motivo che la situazione \u00e8 tranquilla, 'DOMANDA: 'sì sono contenta' ( 4 ) DOMANDA: 'Ah, allora è per questo motivo che & \xmark & \xmark & \xmark & \xmark \\
        
        \bottomrule

    \end{tabularx}
\end{table}
It is not surprising to see that human data is not 100 in all categories, because it was collected through crowdsourcing, and neither the guidelines nor the inspection considered the criteria present in the human evaluation protocol. 
% However, there is one metric that is mentioned which is correctness. Considering the examples that are deemed incorrect, as shown in Tables \ref{tab:human-evaluation-examples} and \ref{tab:human-evaluation-examples-1} we have to consider the subtleties of the Italian language and the subjunctive. This problem is experienced by both human annotators and ChatGPT 3. 

In the Tables \ref{tab:human-evaluation-examples} and \ref{tab:human-evaluation-examples-1} are reported two examples of narratives and respective elicitations, with their evaluations. It is possible to notice that ChatGPT 3 performs extremely well, whereas Falcon 7B is mostly incoherent. Wizard Vicuna is a surprisingly good model in the examples reported, although it does sometimes have issues with the logical coherence of its output to the narrative, which results in low listening scores. Vicuna 33B has similar issues, with occasional narratives that are incohesive. 

From these results, we can conclude that ChatGPT 3.5 turbo presents qualities that are very close to human, and human annotators do not evaluate its results differently from human real elicitations. 

