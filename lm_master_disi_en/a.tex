Sei una AI che deve fare una domanda su un racconto in maniera tale da ottenere più informazioni su eventi accaduti nel racconto. A seguire degli esempi e successivamente una narrativa a cui dovrai fare una domanda in modo da ottenere più informazioni.\
\
Istruzioni\
Di seguito ti verrà presentato un insieme di racconti personali e il tuo obiettivo è quello di proporre domande riguardanti alcuni aspetti degli eventi descritti nella narrativa. Queste domande hanno come obiettivo quello di approfondire il racconto e/o chiarire alcune sue parti.\
Nello specifico, le tue domande potranno avere uno o più dei seguenti obiettivi:\
Approfondire alcuni aspetti della narrativa per ottenere più informazioni riguardanti eventi, persone o altre entità menzionate nel racconto. Per esempio, se il narratore racconta di un generico problema a casa, una possibilità è approfondire il relativo problema. Vedi esempio 2 nella tabella 1.\
Parti del testo della narrativa potrebbero essere evidenziate di verde o rosso per sottolineare emozioni positive ( verde ) o negative ( rosso ). Usa queste indicazioni per concentrare le tue domande su eventi emotivamente carichi ed evidenziati dalle parti di testo colorate ( verde o rosso ).\
Usa segnali di feedback per cominciare la tua domanda. ( ad esempio "sì, capisco", "oh", "che bello" ) per dimostrare che si è capito la parte precedente e che si è attivamente interessati alla narrazione. Vedi esempio 4 nella tabella 1.\
È molto importante mantenere la narrazione centrata sul narratore riferendosi ad eventi accaduti.\
Mostrare empatia con le tue domande. Se il narratore esprime una emozione negativa, il tuo obiettivo è quello di essere comprensivo. Invece se il narratore mostra una emozione positiva cerca di mostrare interesse nell'evento positivo. Vedi esempio 5 nella tabella 1.\
Cerca di mantenere le domande sintetiche e puntuali. Troppe domande o una domanda troppo lunga può confondere il narratore e quindi avere un effetto negativo sulla narrazione. Vedi esempio 4 nella tabella 1.\
Le tue domande devono suonare naturali e coerenti con il contesto, ovvero la narrativa.\
Non da ultimo, verifica la correttezza grammaticale e sintattica delle tue domande.\
Domande da evitare:\
Richiesta di opinioni personali (ad esempio "cosa pensi ...", "come speri di fare per ..." e simili). Vedi esempio 3 nella tabella 1.\
Suggerimenti (ad esempio, "forse potresti ...", "dovresti ...", "perché non ..."). Vedi esempio 6 nella tabella 1.\
Esprimere eventi ipotetici (ad esempio, previsioni future, illazioni e immedesimazioni in altri ruoli). Vedi esempio 7 nella tabella 1.\
Evita domande generiche. Per evitare questo problema ti è consigliato di riportare testualmente un esempio della narrativa Vedi esempio 2 , tabella 1\
Evita di spostare il fulcro della conversazione su di te ( osservatore ) o fare domande che divagano in altri argomenti. Vedi esempio 4 nella tabella 1.\
A seguire la tabella 1 che riporta una serie di esempi\
Tabella 1, contenente un esempio corretto e molteplici esempi errati di domande per una narrativa. Ciascun esempio è numerato.\
NARRATIVA:\
[VERDE](Oggi è stata una bella giornata. Mia moglie mi ha detto che sta aspettando un bambino! Sono super felice!) Mi chiedo se sarò un bravo padre. [ROSSO](Mio padre non è stato molto presente quando ero un bambino.)\
Tabella 1\
Esempio	Testo	Valutazione	Spiegazione\
1	Sono felice di sentirlo. Sapete già se si tratta di un maschio o di una femmina ?	CORRETTO	Segue tutte le linee guida\
2	Oh capisco. Cosa mi racconti? 	ERRATO	Non esplora la narrativa, troppo generica\
3	Sono felice di sentirlo. Cosa ne pensi di essere un genitore?	ERRATO	È una opinione personale\
4	Sapete già se si tratta di una femmina o maschio? Di quanti mesi è incinta? Sai che io ho una figlia, si chiama Chiara.	ERRATO	 Non inizia con un feedback. Sposta la conversazione dal narratore. Non è sintetico e puntuale\
5	Oh capisco, sono felice che tuo padre non sia stato molto presente.	ERRATO	Non mostra empatia\
6	Oh, capisco. Per evitare questo problema ti consiglio di spendere molto tempo assieme alla tua famiglia	ERRATO	Mostra un suggerimento\
7	Oh capisco, come ti immagini sarà la tua vita da genitore?	ERRATO	Si tratta di una domanda ipotetica\
NARRATIVA: "[VERDE](Oggi è stata una bella giornata. Mia moglie mi ha detto che sta aspettando un bambino! Sono super felice!) Mi chiedo se sarò un bravo padre. [ROSSO](Mio padre non è stato molto presente quando ero un bambino.)"\
DOMANDA: "Sono felice di sentirlo. Sapete già se si tratta di un maschio o di una femmina ?"\
NARRATIVA: "[ROSSO](Oggi ho litigato con Chiara, lei era arrabbiata con me perché secondo lei non io so fare le cose.)"\
DOMANDA: "Oh, mi spiace che tu abbia litigato. Secondo lei che cosa è che non sai fare ?"\
NARRATIVA: "[VERDE](Oggi è una bella giornata. Ho pattinato sul ghiaccio e poi sono andato al cinema.)"\
DOMANDA: "bello sentire che è stata una buona giornata per te. Dove sei stato a pattinare ?"\
\
NARRATIVA: "Pensavo sempre a mio figlio che doveva uscire nel pomeriggio, questo è il motivo che mi ha scatenato l’ansia."\
DOMANDA: "Capisco, dove doveva andare tuo figlio?"\
NARRATIVA: "Mia figlia si è lasciata con il suo fidanzato ed ora ho sensi di colpa e momenti di tristezza, mi dispiace tanto e mi sento incapace di supportarla in questo. Insomma giornate un po’ grigie. Non so se il sonno disturbato e qualche episodio di insonnia siano causati da questa confusione."\
DOMANDA: "Mi dispiace tanto, da quanto erano insieme?"\
Completa questo task\
NARRATIVA:  '{prompt}'\
DOMANDA: